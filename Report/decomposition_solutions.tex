The distributed precoder designs for the proposed \ac{JSFRA} scheme are discussed in this section. The \review{convex} formulation in \eqref{eqn-9} or \eqref{eqn-mse-2} requires a centralized controller to perform the precoder design for all users belonging to the coordinating \acp{BS}. In order to design the precoders independently at each \ac{BS} with the minimal information exchange via backhaul, iterative decentralization methods are addressed. In particular, the \acl{PD} and the \ac{ADMM} based \acl{DD} approaches are considered.

Let us consider the \review{convex subproblem with the fixed receive beamformers \me{\mvec{w}{l,k,n}} presented} in \eqref{eqn-9} based on the Taylor series approximation for the nonconvex constraint. The following discussions are equally valid for the \ac{MSE} based solution outlined in \eqref{eqn-mse-2} as well. Since the objective of \eqref{eqn-9} can be decoupled across each \ac{BS}, the centralized problem can be equivalently written as
{\allowdisplaybreaks
\begin{subeqnarray} \label{eqn-decent-1}
\underset{\substack{\gamma_{l,k,n},\mvec{m}{l,k,n}, \beta_{l,k,n}}}{\text{minimize}} &\quad & \sum_{b \in \mc{B}} \| \tilde{\mbf{v}}_b \|_q \IEEEyessubnumber \slabel{eqn-decent-1a} \\
\text{subject to}&\quad& \eqref{eqn-9.1c} - \eqref{eqn-9.1e} \IEEEyessubnumber
\end{subeqnarray}}
where \me{\tilde{\mbf{v}}_b} denotes the vector of weighted queue deviation corresponding to users \me{k \in \mc{U}_b}.

To begin with, let \me{\bar{\mc{B}}_b} denote the set \me{\mc{B} \backslash \{b\}} and \me{\bar{\mc{U}}_b} represents the set \me{\mc{U} \backslash \mc{U}_b}. Following similar approach presented in \cite{pennanen2011decentralized,tolli2011decentralized}, the coupling constraint \eqref{eqn-9.1c} or \eqref{eqn-mse-1.3} can be expressed by grouping the interference from each \ac{BS} in \me{\mathcal{B}} as
\iftoggle{single_column}{
\begin{IEEEeqnarray}{rCl}\label{eqn-decent-3}
\beta_{l,k,n} & \geq & \sum_{\substack{j = 1\\j \neq l}}^L |\mvec{w}{l,k,n}^\herm \mvec{H}{b_k,k,n} \mvec{m}{j,k,n} |^2 \nonumber \\
&\quad& + \sum_{i \in \mc{U}_{b_k} \backslash \{k\}} \sum_{j = 1}^L |\mvec{w}{l,k,n}^\herm \mvec{H}{b_k,k,n} \mvec{m}{j,i,n} |^2 + \sum_{b \in \bar{\mc{B}}_{b_k}} \zeta_{l,k,n,b} \; + \; \enoise
\end{IEEEeqnarray}}{\allowdisplaybreaks
\begin{multline}\label{eqn-decent-3}
	\enoise + \sum_{\substack{j = 1,j \neq l}}^L |\mvec{w}{l,k,n}^\herm \mvec{H}{b_k,k,n} \mvec{m}{j,k,n} |^2 + \sum_{b \in \bar{\mc{B}}_{b_k}} \zeta_{l,k,n,b}  \\
	+ \sum_{i \in \mc{U}_{b_k} \backslash \{k\}} \sum_{j = 1}^L |\mvec{w}{l,k,n}^\herm \mvec{H}{b_k,k,n} \mvec{m}{j,i,n} |^2 \leq \beta_{l,k,n}
\end{multline}}
where \me{\zeta_{l,k,n,b}} is the total interference caused by the transmission of \ac{BS} \me{b} to user \me{k \in \mc{U}_{b_k}} in the spatial stream \me{l} and sub-channel \me{n}. It is given by the following upper bound as
\begin{equation} \label{inter_exp}
\zeta_{l,k,n,b} \geq \sum_{i \in \mc{U}_b} \sum_{j = 1}^L |\mvec{w}{l,k,n}^\herm \mvec{H}{b_i,k,n} \mvec{m}{j,i,n} |^2 \; \forall b \in \bar{\mc{B}}_{b_k}.
\end{equation}

The decentralization is achieved by decomposing the original convex problem in \eqref{eqn-decent-1} by a parallel iterative subproblems coordinated by either primal or \acl{DD} update. The coupling variables are updated in each iteration by exchanging limited information among the subproblems. Before proceeding further, let \me{\bar{\mbfa{\zeta}}_b} be the vector formed by stacking interference terms \eqref{inter_exp} from the neighboring \acp{BS} to the users of \ac{BS} \me{b} and \me{\hat{\mbfa{\zeta}}_{b}} be the stacked interference terms caused by \ac{BS} \me{b} to all users in the neighboring \acp{BS} \me{\bar{\mc{B}}_b}, represented as
\begin{IEEEeqnarray}{rCl}
	\bar{\mbfa{\zeta}}_{b} &=& \left [ {\zeta}_{l,k,n,\bar{\mc{B}}_b(1)}, \dotsc, {\zeta}_{l,k,n,\bar{\mc{B}}_b(|\bar{\mc{B}}_b|)} \right]^\tran, \forall k \in \mc{U}_{b} \eqsub \\
	\hat{\mbfa{\zeta}}_{b} &=& \left [ {\zeta}_{l,\bar{\mc{U}}_b(1),n,b}, {\zeta}_{l,\bar{\mc{U}}_b(2),n,b}, \dotsc, {\zeta}_{l,\bar{\mc{U}}_b(|\bar{\mc{U}}_b|),n,b} \right]^\tran. \eqsub
\end{IEEEeqnarray}
Let us define the vector \me{\mbfa{\zeta}_b}, formed by stacking the interference terms corresponding to the \ac{BS} \me{b} as
\begin{equation}
	\mbfa{\zeta}_{b} = \left [\hat{\mbfa{\zeta}}_{b}^\tran, \bar{\mbfa{\zeta}}_{b}^\tran \right ]^\tran.
	\label{if_vector}
\end{equation}
Since the decentralization solution is an iterative procedure, we represent the \me{\ith{i}} iteration index as \me{x^{(i)}}. Let \me{\mbfa{\zeta}_{b}(b_k)} denote the interference terms corresponding to the \ac{BS} \me{b_k} in \ac{BS} \me{b} as
\begin{equation}
	\mbfa{\zeta}_{b}(b_k) = \left [ {\zeta}_{l,\mc{U}_b(1),n,b_k},\dotsc, {\zeta}_{l,\mc{U}_b(|\mc{U}_b|),n,b_k} \right].
\end{equation}

To decentralize the problem in \eqref{eqn-decent-1}, the \ac{BS} specific vector \me{\mbfa{\zeta}_b} in \eqref{if_vector}, which are relevant for the \ac{BS} \me{b}, can either be fixed or treated as a variable in accordance to the decomposition method. To decouple the precoder design across \acp{BS}, the equivalent downlink channel \me{\mvec{w}{l,k,n}^\herm \mvec{H}{b,k,n}, \forall k \in {\mc{U}}} are to be known at each \ac{BS} \me{b} through the precoded uplink pilots from all the users in the system, where the precoders are the \ac{MMSE} receiver \me{\mvec{w}{l,k,n}} evaluated at the user. Similarly to update the \ac{MMSE} receivers at each user \me{k}, the equivalent channels \me{\mvec{H}{b,k,n} \mvec{m}{l,k^{\prime},n}, \forall k^{\prime} \in {\mc{U}_b}, \forall b \in \mc{B}} are to be known through the user specific precoded downlink pilots precoded with the updated transmit beamformers \me{\mvec{m}{l,k,n}, \forall k \in \mc{B}} evaluated at the \ac{BS} \me{b} by using the equivalent downlink channel that includes the updated \ac{MMSE} receivers of all the users, as in \cite{komulainen2013effective}.

%The subproblems with the \ac{BS} specific interference variables are controlled by a master subproblem, which determines the interference variable \me{\zeta_{l,k,n,b}^{\{b_k\}}} in accordance with the global objective.
