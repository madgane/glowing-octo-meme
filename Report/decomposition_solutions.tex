This section addresses the distributed precoder designs for the proposed \ac{JSFRA} scheme. The formulation in \eqref{eqn-9} requires a centralized controller to perform the precoder design for all users belonging to the coordinating \acp{BS}. In order to design the precoders independently at each \ac{BS} with the minimal information exchange via backhaul, iterative decentralization methods are considered. In particular, the \ac{PD} and the \ac{ADMM} based \ac{DD} approaches are addressed.

To begin with, let \me{\bar{\mc{B}}_b} denote the set \me{\set{\mc{B} \backslash \{b\}}} and \me{\bar{\mc{U}}_b} represents the set \me{\set{\mc{U} \backslash \mc{U}_b}}. The centralized problem defined by \eqref{eqn-9} can be equivalently written as
\begin{IEEEeqnarray}{rCl} \label{eqn-decent-1}
\underset{\substack{t_{l,k,n},\gamma_{l,k,n} \\ \mvec{m}{l,k,n}, \beta_{l,k,n}}}{\text{minimize}} &\quad & \sum_{b \in \mc{B}} \| \tilde{\mbf{v}}_b \|_q \IEEEyessubnumber \label{eqn-decent-1a} \\
\text{subject to}&\quad& \eqref{eqn-9.1c} - \eqref{eqn-9.1e} \IEEEyessubnumber,
\end{IEEEeqnarray}
where \me{\tilde{\mbf{v}}_b} denote the vector of of weighted queue deviation corresponding to the users \me{k \in \mc{U}_b}.

Following the similar approach as in \cite{pennanen2011decentralized,tolli2011decentralized} and \cite{kaleva2013primal}, the interference constraint given by \eqref{eqn-9.1c} can be expressed by grouping the interference contribution from each \acp{BS} in the system as
\begin{IEEEeqnarray}{RCL}
\beta_{l,k,n} & \geq & \sum_{\substack{j = 1\\j \neq l}}^L |\mvec{w}{l,k,n}^\herm \mvec{H}{b_k,k,n} \mvec{m}{j,k,n} |^2 \nonumber \\
&& + \sum_{i \in \mc{U}_{b_k} \backslash \{k\}} \sum_{j = 1}^L |\mvec{w}{l,k,n}^\herm \mvec{H}{b_k,k,n} \mvec{m}{j,i,n} |^2 + \underbrace{\sum_{b \in \bar{\mc{B}}_{b_k}} \sum_{i \in \mc{U}_b} \sum_{j = 1}^L |\mvec{w}{l,k,n}^\herm \mvec{H}{b_i,k,n} \mvec{m}{j,i,n} |^2}_{\text{neighboring \acp{BS} interference}} \; + \; N_0, \forall k \in \mc{U}.
\label{eqn-decent-2}
\end{IEEEeqnarray}
In \eqref{eqn-decent-2}, the neighboring \acp{BS} signals can be written explicitly using the relaxed expression as
\begin{IEEEeqnarray}{rCl}\label{eqn-decent-3}
\beta_{l,k,n} & \geq & \sum_{\substack{j = 1\\j \neq l}}^L |\mvec{w}{l,k,n}^\herm \mvec{H}{b_k,k,n} \mvec{m}{j,k,n} |^2 \nonumber \\
&\quad& + \sum_{i \in \mc{U}_{b_k} \backslash \{k\}} \sum_{j = 1}^L |\mvec{w}{l,k,n}^\herm \mvec{H}{b_k,k,n} \mvec{m}{j,i,n} |^2 + \sum_{b \in \bar{\mc{B}}_{b_k}} \zeta_{l,k,n,b} \; + \; N_0, \IEEEyessubnumber \label{eqn-decent-3a}
\end{IEEEeqnarray}
where \me{\zeta_{l,k,n,b}}, which is the total interference caused by the \ac{BS} \me{b} to the \me{\ith{l}} stream of user \me{k \in \mc{U}_{b_k}} on the \me{\ith{n}} sub-channel, is upper bounded by
\begin{equation}
\zeta_{l,k,n,b} \geq \sum_{i \in \mc{U}_b} \sum_{j = 1}^L |\mvec{w}{l,k,n}^\herm \mvec{H}{b_i,k,n} \mvec{m}{j,i,n} |^2, \forall b \in \bar{\mc{B}}_{b_k}.
\end{equation}

The expression in \eqref{eqn-decent-3} can be solved either by primal or by dual decomposition based on the usage of the coupling variable \me{\zeta_{l,k,n,b}} in the optimization problem. In both approaches, the distributed problem is solved by a two level optimization approach in which the local problems are solved at the \ac{BS} level and the master problem controls the variables involved in each local problem \cite{palomar2006tutorial}. 