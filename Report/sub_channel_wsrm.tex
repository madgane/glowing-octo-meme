The complexity of the \ac{JSFRA} algorithm scales quickly with the number of sub-channels, since the complexity of an interior point method, which is used to solve the problem, increases with the problem size. Thus, we can use the decomposition methods presented in \cite{palomar2006tutorial,boyd2011distributed} to overcome this complexity by designing precoders for each sub-channels independently with minimal information exchange.

As an alternative sub-optimal solution, we present a \acl{QM} \ac{SRA}, which solves for the precoders using \ac{JSFRA} formulation for a specific sub-channel \me{i} with a fixed transmit power \me{P_{\max,i}}. The power sharing can either be equal or based on some predetermined limits on each sub-channel as in partial frequency reuse as
\begin{equation}
\sum_{i=1}^N P_{\max,i} = P_{\max}.
\end{equation}
Even though \me{N} sub-channels are present at any given scheduling instant, precoders are computed for each sub-channel sequentially with \me{P_{\max,i}} and the residual number of backlogged packets. Let \me{Q_{k,i}} be the number of backlogged packets associated with user \me{k} while determining the precoders for the \eqn{\ith{i}} sub-channel. Since the precoder design is sequential, \textit{i.e}, the precoders are designed for sub-channels \me{[0,i-1]} before the \me{\ith{i}} sub-channel, the number of backlogged packets for the initial sub-channel is initialized as \me{Q_{k,1} = Q_k}. The queues associated with the consecutive sub-channels are given by
\begin{equation}	\label{eqn-weight}
	Q_{k,i+1} = \max \Big ( Q_k - \sum_{j = 1}^{i} \, \sum_{l = 1}^{L} \, t_{l,k,j} ,0 \Big ) \; \forall \; k \in \mathcal{U}
\end{equation}
where \me{t_{l,k,j}} is the \me{\ith{k}} user rate on sub-channel \me{j}. 

\review{For simplicity we use random sub-channel ordering in our paper, \textit{i.e.}, after finding the precoders for a current sub-channel, we can choose any previously unselected sub-channels as the next candidate sub-channel for which the precoders are identified using the updated backlogged packets. Additionally, we can also use greedy ordering by considering the channel norm between all users in the system and the corresponding serving \acp{BS}, but this comes at the cost of increased complexity. However, it is worth noting that, as the number of users in the system increases, the \ac{SRA} scheme will be insensitive to the sub-channel ordering due to the available multi-user diversity present in the system.}

%The algorithmic representation of the \ac{SRA} scheme is shown in Algorithm \ref{algo-5}
%\begin{algorithm}
% \SetAlgoLined
% \DontPrintSemicolon
% \BlankLine
% \SetKwInput{KwInit}{Initialize}
% \KwIn{\me{a_k, \, Q_k, \, \mvec{H}{b,k,n},\; \fall b \in \mathcal{B}, \, \fall k \in \mathcal{U}}}
% \KwIn{permute \me{\mathcal{N} \rightarrow \tilde{\mathcal{N}}}}
% \For{\me{n \leftarrow 1 } \KwTo \me{N}}{
% update \me{Q_{k,n}} using \eqref{eqn-weight} and let \me{\hat{n} = \tilde{\mathcal{N}}(n)}\;
% \KwOut{\me{\mvec{m}{l,k,\hat{n}}} and \me{\mvec{w}{l,k,\hat{n}} \fall l \in \set{1,2,\dotsc,L}}}
% \KwInit{\me{i=0} and \me{\tilde{\mbf{m}}_{l,k,\hat{n}}} randomly satisfying per sub-channel power constraint \me{P_{\max,\hat{n}}}}
% update \me{\mvec{w}{l,k,\hat{n}}} and \me{\tilde{\mbf{u}}_{l,k,\hat{n}}} using \eqref{eqn-10} and \eqref{eqn-8}\;
% \Repeat{Queue convergence or \me{i \geq I_{\max}}}{
% initialize \me{j = 0}\;
% \Repeat{\ac{SCA} convergence or \me{j \geq J_{\max}}}{
% solve for \me{\mvec{m}{l,k,\hat{n}}} using \eqref{eqn-9} with per sub-channel power constraint \me{P_{\max,\hat{n}}}\;
% update the constraint set \eqref{eqn-8} with \me{\tilde{u}_{l,k,\hat{n}}} and \me{\mvec{m}{l,k,\hat{n}}} using \eqref{eqn-wsrm-expr}\;
% $j = j + 1$\;
% }
% update \me{\mvec{w}{l,k,n}} using \eqref{eqn-10} with the updated precoders \me{\mvec{m}{l,k,n}}\;
% $i = i + 1$\;
% }
% }
% \caption{Algorithm of \acs{SRA} scheme}
% \label{algo-5}
%\end{algorithm}
%
