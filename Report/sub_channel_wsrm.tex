
The complexity involved in the \ac{JSFRA} scheme scales significantly with the increase in the number of sub-channels considered in the formulation. In addition to the increased complexity, the rate of convergence to the optimal precoders also degrades due to its dependency on the problem size. In order to mitigate this, we provide an alternative sub-optimal solution, in which the precoders are designed over each sub-channel independently in a sequential manner by taking the reamining number of queued bits in the formulation. It provides a sub-optimal solution to the optimal way of distributing the sub-channel wise precoder design via primal/dual decomposition methods discussed in \cite{palomar2006tutorial,boyd2011distributed}.

The proposed \acf{QM} \ac{SRA} scheme decouples the problem by fixing the power across each sub-channel to a constant value as compared to the global power constraint defined by \eqref{eqn-4.3}. In contrast to the decomposition based approach for the sub-channel wise resource allocation, where the primal/dual variables are exchanged, this method requires the update on the number of queued bits before each sub-channel level optimization. The number of queued bits for each user are updated by the difference between the total number of queued bits during the current slot to the total number of bits that are guaranteed by the earlier sub-channel wise allocations in the same slot as
\begin{equation}
Q_{k,n} = \max{\Big \lbrace Q_k - \sum_{j = 1}^{n-1} \, \sum_{l = 1}^{L} \, t_{l,k,j} ,0 \Big \rbrace }, \; \forall \; k \in \mathcal{U}
\label{eqn-weight}
\end{equation}
where \me{Q_{k,n}} is the total number of queued bits used in the optimization problem carried out for the sub-channel \me{n}. In the expression \eqref{eqn-weight}, \me{Q_k} denotes the total number of queued bits waiting to be transmitted for the user \me{k} during the current slot and \me{t_{l,k,j}} is the rate or guaranteed bits allocated over the sub-channel \me{j}. The current scheme is very sensitive to the order in which the sub-channels are selected for the optimization problem.
