
The complexity involved in the \ac{JSFRA} scheme scales up significantly based on the sub-channel over which the decision has to be made. Not only the complexity, but also the convergence speed depends on the size  and the number of decision variables involved in the optimization problem. In order to reduce the complexity involved in the \ac{JSFRA} scheme, we provide an alternative sub-optimal solution, where the precoders are designed over each sub-channel independently in a sequential manner by taking the remaining number of queued bits into consideration. It provides an alternative way to the parallel sub-channel wise precoder update using the decomposition techniques discussed in \cite{palomar2006tutorial,boyd2011distributed}.

The proposed \acf{QM} \ac{SRA} scheme decouples the fixing the power across each sub-channel to a constant value as compared to the global power constraint defined by \eqref{eqn-4.3}. In contrast to the decomposition based approach for sub-channel wise resource allocation, this method requires the update on the number of queued bits before each sub-channel level optimization. In this approach, the number of queued bits are updated by the difference between the total number of queued bits during the current slot and the total number of bits that are guaranteed by the earlier sub-channel wise allocations in the same slot as
\begin{equation}
Q_{k,n} = \max{\Big \lbrace Q_k - \sum_{j = 1}^{n-1} \, \sum_{l = 1}^{L} \, t_{l,k,j} ,0 \Big \rbrace }, \; \forall \; k \in \mathcal{U}
\label{eqn-weight}
\end{equation}
where \me{Q_{k,n}} is the total number of queued bits used in the optimization problem carried out for the sub-channel \me{n}. In the expression \eqref{eqn-weight}, \me{Q_k} denotes the total number of queued bits waiting to be transmitted for the user \me{k} during the current slot and \me{t_{l,k,j}} is the rate or guaranteed bits allocated over the sub-channel \me{j}. The current scheme is very sensitive to the order in which the sub-channels are selected for the optimization problem.
