In this paper, we consider the linear precoder design for the coordinated multi-cell \ac{MU} \ac{MIMO} technique with the \ac{OFDM} transmission for the downlink \acl{BC}. Each \ac{BS} serves the users associated with it, which causes interference to the neighboring \ac{BS} users. The design objective is to minimize the number of queued packets with all the coordinating \acp{BS} in the network, since the transmissions are guided by the backlogged packets. Traditionally, the problem is solved by weighing the transmission rates of the users by the corresponding length of the queued packets in the \ac{WSRM} method, \textit{i.e}, longer the queue, higher the priority. In this work, we approach the problem by formulating it as a nonconvex optimization problem and by applying \ac{SCA} method, we find the transmit and the receive beamformers in a coordinated manner. In the first part of the paper, we discuss the centralized solutions for the proposed problem in which the precoders are designed across the space-frequency resources jointly to minimize the number of backlogged packets, which is referred as the \ac{JSFRA} scheme. In the second part, we discuss the precoder design across the coordinating \acp{BS} in a distributed manner using primal and the \ac{ADMM} based approaches. In addition to that, we also propose an iterative algorithm by solving the \ac{KKT} conditions for the \ac{JSFRA} scheme based on the \ac{MSE} relaxation, which can be  solved in a distributed way across the coordinating \acp{BS}. The proposed solutions are compared to the traditional \ac{Q-WSRM} approaches mentioned above in terms of the rate of convergence and the backlogged packets remaining after a scheduling instant.
