We consider a downlink multi-cell \ac{MIMO} \ac{IBC} scenario using \ac{OFDM} with \acl{MU} contending for space-frequency resources in a given scheduling instant. The problem is to determine the transmit precoders by the \acp{BS} in a coordinated approach to minimize the total number of backlogged packets in the \acp{BS}, which are destined for the users in the system. Traditionally, it is solved using \ac{WSRM} objective with the number of backlogged packets as the corresponding weights, \textit{i.e}, longer the queue size, higher the priority. In contrast, we design the precoders jointly across the space-frequency resources by minimizing the total user queue deviations. The problem is nonconvex and therefore we employ \ac{SCA} technique to solve the problem by a sequence of convex subproblems using first order Taylor approximations. At first, we propose a centralized \ac{JSFRA} solution using two different formulations by employing \ac{SCA} technique, namely the sum rate formulation and the \ac{MSE} reformulation. We then introduce distributed precoder designs using primal and \acl{ADMM} method for the \ac{JSFRA} solutions. Finally, we propose a practical distributed iterative precoder design based on \ac{MSE} reformulation approach by solving the \acl{KKT} conditions with closed form expressions. Numerical results are used to compare the proposed algorithms with the existing solutions.

%we address the queue minimizing downlink precoder design as a joint nonconvex optimization problem over space-frequency resources. We employ \ac{SCA} technique to solve the problem by a sequence of convex subproblems using inner approximations. Initially, we discuss the centralized \acl{JSFRA} solutions based on \ac{SCA} as well as by \acl{MSE} reformulation. Then we extend the distributed precoder design for the centralized schemes using primal and \ac{ADMM} method. Finally, we discuss the distributed precoder design problem by solving the \ac{KKT} expressions to obtain the closed form solutions for the transmit and the receive precoders. Numerical results are shown to compare them.


