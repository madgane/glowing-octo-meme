
In this section, we discuss the \ac{ADMM} decomposition method, which is basically based on the \acl{DD}, but shows better convergence properties. In contrast to the \acl{PD} problem, the \ac{ADMM} method relaxes the interference constraints by including it in the objective function of each subproblem with a penalty pricing \cite{palomar2006tutorial,boyd2011distributed}. In order to decouple the problem \eqref{eqn-decent-1}, the coupling variables \me{\zeta_{l,k,n,b}} in \eqref{eqn-decent-3} are replaced by the respective local copies \me{\mbfa{\zeta}^{\set{b}}, \forall b \in \mc{B}}, which are then solved for an optimal solution. Now the sub problems are coupled by the global consensus vector \me{\mbfa{\zeta}} maintaining the complete stacked  interference profile of all users in the system as
\begin{IEEEeqnarray}{rCl}
\mbfa{\zeta} &=& \left [ \zeta_{1,\bar{\mc{U}}_{1}(1),1,1}, \dotsc, \zeta_{L,\bar{\mc{U}}_{1}(1),1,1}, \dotsc, \zeta_{L,\bar{\mc{U}}_{1}(|\bar{\mc{U}}_{1}|),1,1},  \right. \nonumber \\
&\quad& \left . \dotsc, \zeta_{L,\bar{\mc{U}}_{N_B}(|\bar{\mc{U}}_{N_B}|),1,N_B}, \dotsc, \zeta_{L,\bar{\mc{U}}_{N_B}(|\bar{\mc{U}}_{N_B}|),N,N_B} \right ] \IEEEyessubnumber \\
n_{b_k} &=& |\mbfa{\zeta}^{\set{b_k}}| = N L \sum_{b \in \mc{B}} \left | \bar{\mc{U}}_b\right | \IEEEyessubnumber
\end{IEEEeqnarray}

Let \me{\mbfa{\zeta}(b_k)} denotes the consensus entries corresponding to the \ac{BS} \me{b_k}. Let \me{\mbfa{\nu}^{\set{b_k}}} represents the stacked dual variables corresponding to the equality condition \me{\mbfa{\zeta}^{\set{b_k}} = \mbfa{\zeta}(b_k)} used in the subproblems. In order to limit the local interference assumptions \me{\zeta^{\set{b_k}}_{l,k,n,b}} at the \acp{BS} \me{b_k}, the \ac{ADMM} method augments a scaled quadratic penalty of the interference deviation between the local and consensus value for the interference from the \ac{BS} \me{b} as \me{\zeta_{l,k,n,b}} in the objective function. At optimality, the locally assumed and the consensus interference values will be equal,  providing no contribution to the objective function. The optimal step size used to update the dual variables is the scaling factor \me{\rho} used to scale the penalty term in the objective function \cite{boyd2011distributed,bertsekas1999nonlinear}. The equality constraint for the local and the consensus interference vector \me{\mbfa{\zeta}^{\set{b_k}} = \mbfa{\zeta}(b_k)} present in each subproblem is relaxed by the taking the partial Lagrangian. Now, the subproblem at the \ac{BS} \me{b} for the \me{\ith{i}} iteration is given by
\begin{IEEEeqnarray}{rCl} \label{eqn-dual-3}
\underset{\substack{\gamma_{l,k,n}, \mvec{W}{k,n},\mvec{M}{k,n} \\ \beta_{l,k,n}, \mbfa{\zeta}^{\set{b}(i)}}}{\text{minimize}} &\quad & \| \tilde{\mbf{v}}_{b} \|_q + \mbfa{\nu}^{{\set{b}(i-1)}T} \left ( \mbfa{\zeta}^{\set{b}(i)} - \mbfa{\zeta}^{(i-1)}(b) \right ) + \frac{\rho}{2} \Big \| \, \underbrace{\mbfa{\zeta}^{(i-1)}(b)}_{\text{consensus}} - \underbrace{\mbfa{\zeta}^{\set{b}(i)}}_{\text{locals}} \, \Big \|^2_2 \IEEEyessubnumber \label{eqn-dual-3a} \\
\text{subject to} && \beta_{l,k,n} \geq \sum_{\substack{j = 1\\j \neq l}}^L |\mvec{w}{l,k,n}^\herm \mvec{H}{{b},k,n} \mvec{m}{j,k,n} |^2 + \sum_{{\hat{b}} \in \bar{\mc{B}}_{b}} \zeta^{\set{b}(i-1)}_{l,k,n,{\hat{b}}} \nonumber \\
&& \qquad {} \qquad {} + \sum_{i \in \mc{U}_{b} \backslash \{k\}} \sum_{j = 1}^L |\mvec{w}{l,k,n}^\herm \mvec{H}{{b},k,n} \mvec{m}{j,i,n} |^2 + N_0\|\mvec{w}{l,k,n}\|^2 \IEEEyessubnumber \label{eqn-dual-1d} \\
&& \zeta^{\set{b}(i)}_{\pr{l},\pr{k},n,{b}} \geq \sum_{k \in \mc{U}_{\hat{b}}} \sum_{l = 1}^L |\mvec{w}{\pr{l},\pr{k},n}^\herm \mvec{H}{b,\pr{k},n} \mvec{m}{l,k,n} |^2, \; \forall \pr{k} \in \bar{\mc{U}}_{b}, \; \forall n \in \mc{C} \IEEEyessubnumber \label{eqn-dual-1e} \\
&\quad& \eqref{eqn-8} \; \text{and} \; \eqref{eqn-primal-1c} \IEEEyessubnumber \label{eqn-dual-1f}
\end{IEEEeqnarray}
where the superscript \me{i} represents the current iteration or the information exchange index and \me{\mbfa{\zeta}^{(i-1)}} denotes the updated global interference level from the \me{\ith{(i-1)}} information exchange of the local interference vector \me{\mbfa{\zeta}^{\set{b}(i-1)}, \forall b \in \mc{B}}.

Now, the local problem \eqref{eqn-dual-3} at each \ac{BS} \me{b} is solved either by the \ac{SCA} approach discussed in Section \ref{sec-3.2.1} or by using the \ac{MSE} reformulation approach outlined in Section \ref{sec-3.3}. Once the local problems are solved at each \ac{BS}, the new update for the global interference vector \me{\mbfa{\zeta}^{(i)}} and the dual variables \me{\mbfa{\nu}^{\set{b}(i)}} are performed at each \ac{BS} independently by exchanging the corresponding local copies of the interference vector \me{\mbfa{\zeta}^{\set{b}(i)}, \forall b \in \mc{B}}. Since the entries in \me{\mbfa{\zeta}^{(i)}} relates exactly two \acp{BS} only, each entry in the \me{\mbfa{\zeta}^{(i)}} can be updated by exchanging the local copies from the corresponding two \acp{BS} only. For instance, the entry \me{\zeta^{(i)}_{l,\mc{U}_{b_k}(1),n,b}} depends on the local interference value \me{\zeta^{\set{b_k}(i)}_{l,\mc{U}_{b_k}(1),n,b}} assumed by the \ac{BS} \me{b_k} and the actual interference caused by the \ac{BS} \me{b} as in \me{\zeta^{\set{b}(i)}_{l,\mc{U}_{b_k}(1),n,b}} as
\begin{equation}
\zeta_{l,\mc{U}_{b_k}(1),n,b}^{(i)} = \frac{1}{2} \, \left ( \, \zeta^{\set{b}(i)}_{l,\mc{U}_{b_k}(1),n,b} + \zeta^{\set{b_k}(i)}_{l,\mc{U}_{b_k}(1),n,b} \, \right )
\label{zeta_update}
\end{equation}
The dual variable entries in the vector \me{\mbfa{\nu}^{\set{b_k}}}, which is the stacked dual variables corresponding to the interference equality constraint at the \ac{BS} \me{b_k}, are updated using the subgradient as
\begin{equation}\label{dual-sg-update}
\nu^{\set{b_k}(i)}_{l,k,n,b} = \nu^{\set{b_k}(i-1)}_{l,k,n,b} + \rho \, \left (\zeta^{\set{b_k}(i)}_{l,k,n,b} - \zeta^{(i)}_{l,k,n,b} \right ), \forall b,b_k \in \mc{B}, \forall k \in \bar{\mc{U}}_{b}
\end{equation}

\subsubsection*{Convergence}
The convergence of the \ac{ADMM} method follows the same argument as the centralized algorithm if each distributed algorithm is allowed to converge to the optimal value for a fixed \ac{SCA} point. Since subproblem solved at each \ac{BS} is convex, the \ac{ADMM} method converges to the optimal value \cite{boyd2011distributed} for a given \ac{SCA} point. The receive beamformers are updated at each \ac{SCA} update provides monotonic increase in the objective function, since the \ac{MMSE} receive beamformers are optimal for the fixed transmit precoders obtained by solving the subproblems until convergence. The algorithmic representation of the \ac{ADMM} based approach for decentralization is given in Algorithm \ref{algo-3}.
\begin{algorithm}
 \SetAlgoLined
 \DontPrintSemicolon
 \BlankLine
 \SetKwInput{KwInit}{Initialize}
 \KwIn{\me{a_k, \, Q_k, \, \mvec{H}{b,k,n},\; \fall b \in \mathcal{B}, \, \fall k \in \mathcal{U}, \fall n \in \mathcal{N}}}
 \KwOut{\me{\mvec{m}{l,k,n}} and \me{\mvec{w}{l,k,n} \fall l \in \set{1,2,\dotsc,L}}}
 \KwInit{\me{i=0} and the transmit precoders \me{\tilde{\mbf{m}}_{l,k,n}} randomly satisfying the total power constraint \eqref{eqn-4.3}}
 update \me{\mvec{w}{l,k,n}} with \eqref{eqn-10} and \me{\tilde{\mbf{u}}_{l,k,n}} with \eqref{eqn-8} using \me{\tilde{\mbf{m}}_{l,k,n}} \;
 initialize the global interference vectors \me{\mbfa{\zeta}^{(0)} = \mbfa{0}^{\mathrm{T}}} \;
 initialize the interference threshold \me{\mbfa{\nu}^{\set{b}(0)} \forall b \in \mc{B} = 0} \;
 for each \ac{BS} \me{b \in \mc{B}}, perform the following procedure \;
 \Repeat{convergence or \me{i \geq I_{\max}}}{
    initialize \me{j=0} \;
    \Repeat{convergence or \me{j \geq J_{\max}}}{
        solve for the transmit precoders \me{\mvec{M}{k,n}} and the local interference \me{\mbfa{\zeta}^{\set{b}}} using \eqref{eqn-dual-3} \;
        exchange \me{\mbfa{\zeta}^{\set{b}(j)}} across the coordinating \acp{BS} in \me{\mc{B}} \;
		update the dual variables in \me{\mbfa{\nu}^{\set{b}(j+1)}} using \eqref{dual-sg-update} \;
        update the global interference vector \me{\mbfa{\zeta}^{(j+1)}} using \eqref{zeta_update} \;
		\me{j = j + 1} \;
    }
    update the receive beamformers \me{\mvec{w}{l,k,n}} using \eqref{eqn-10} by exchanging the recent precoders \me{\mvec{m}{l,k,n}} \;
    exchange the receive precoders \me{\mbf{W}_{k,n}} \me{\forall k \in \mc{U}_b} among the \acp{BS} in \me{\mc{B}} \;
    update \me{\tilde{p}_{l,k,n}}, \me{\tilde{q}_{l,k,n}} and \me{\tilde{\beta}_{l,k,n}} with the recent precoders using \eqref{eqn-wsrm-expr} and \eqref{eqn-6.3} for the \ac{SCA} approach (or)\;
    update \me{\tilde{\epsilon}_{l,k,n}} with the recent precoders using \eqref{eqn-mse-2.3} with equality for the \ac{MSE} formulation approach \;
    $i = i + 1$ \;
  }
 \caption{Decentralization via \ac{ADMM} for \acs{JSFRA} scheme}
  \label{algo-3}
\end{algorithm}


