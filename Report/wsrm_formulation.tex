
In this approach, we present an iterative algorithm to solve \eqref{eqn-3} locally based on the idea of alternating optimization and successive convex approximation. Using \eqref{eq:SINR}, we can reformulate the problem defined in \eqref{eqn-3} as
\begin{IEEEeqnarray}{RCL}\label{eqn-6}
\underset{\substack{\gamma_{l,k,n},\mvec{m}{l,k,n},\\\beta_{l,k,n},\mvec{w}{l,k,n}}}{\text{minimize}} & \quad & \|  \tilde{\mbf{v}}  \|_q \label{eqn-obj} \IEEEyessubnumber \\
\text{subject to}& \quad &\gamma_{l,k,n} \leq \dfrac{\mvec{w}{l,k,n}^\herm \mvec{H}{l,k,n} \mvec{m}{l,k,n}}{\beta_{l,k,n}} \triangleq f(\tilde{\mbf{u}}_{l,k,n}),\IEEEyessubnumber \label{eqn-6.2} \\
  & \quad & \beta_{l,k,n} \geq  N_0 \|\mvec{w}{l,k,n}\|^2 + \hspace{-0.75em} \sum_{(j,i) \neq (l,k)} \hspace{-0.75em} |\mvec{w}{l,k,n}^\herm \mvec{H}{b_i,k,n} \mvec{m}{j,i,n} |^2, \IEEEyessubnumber \label{eqn-6.3} \\
  & \quad &\eqref{eqn-4.3} \IEEEyessubnumber
\end{IEEEeqnarray}
where \me{\beta_{l,k,n}} denotes the total interference seen by the \me{l^\mathrm{th}} stream of the user \me{k} on the \me{n^\mathrm{th}} sub-channel and let \me{\tilde{\mbf{u}}_{l,k,n} \triangleq \{\mvec{w}{l,k,n}^\herm, \mvec{H}{b_k,k,n}, \mvec{m}{l,k,n},\beta_{l,k,n}\}} be the vector which needs to be identified for the optimal allocation. In this formulation, we relaxed the equality constraint in \eqref{eq:SINR} by the inequalities in \eqref{eqn-6.2} and \eqref{eqn-6.3}. However, this step is without loss of optimality leads to the same solution, since the inequalities in \eqref{eqn-6.2} and \eqref{eqn-6.3} are active for an optimal solution, following the same arguments as those in \cite{tran2012fast}. Intuitively, \eqref{eqn-6.2} denotes the \ac{SINR} constraint for \me{\gamma_{l,k,n}}, and \eqref{eqn-6.3} gives an upper bound for the interference seen by the user \me{k \in \mathcal{U}_b}. The problem in \eqref{eqn-6} is known to be NP-hard even for the single antenna case \cite{wmmse_shi,kaleva2013primal}. The reformulation in \eqref{eqn-6} allows a tractable solution as presented below. First, we note that the constraints \eqref{eqn-4.3} are convex with involved variables. Thus, we only need to deal with \eqref{eqn-6.2} and \eqref{eqn-6.3}. Towards this end, we resort to the traditional coordinate descent technique by fixing the transmit beamformers, and find the optimal linear receivers. The optimal linear receiver for the fixed transmit precoders \me{\mbf{M}_{i,n} \, \forall i \in \mc{U}, \, \forall n \in \mc{C}} is obtained by minimizing \eqref{eqn-3} w.r.t the \me{\mbf{w}_{l,k,n}}, and the closed form solution is given by
\begin{IEEEeqnarray}{rCl}\label{eqn-10}
\mvec{w}{l,k,n} &=& \mathbf{R}^{-1}_{l,k,n} \; \mvec{H}{b_k,k,n} \; \mvec{m}{l,k,n}, \IEEEyessubnumber \\
\mvec{R}{l,k,n} &=& \displaystyle \sum_{i\in \mc{U}} \sum_{j=1}^L \mvec{H}{b_i,k,n} \mvec{m}{j,i,n} \mvec{m}{j,i,n}^\herm \mvec{H}{b_i,k,n}^\herm + N_0 \, \mathbf{I}_{N_R}, \IEEEyessubnumber
\end{IEEEeqnarray}
which is the \ac{MMSE} receive beamformers \cite{christensen2008weighted,wmmse_shi,kaleva2013primal}.

The problem now is to find optimal transmit beamformers for a given set of linear receivers which is a challenging task. We note that for fixed $\mvec{w}{l,k,n}$, \eqref{eqn-6.3} can be written as a \ac{SOC} constraint. Thus, the difficulty is due to the non-convexity in \eqref{eqn-6.2}. To arrive at a tractable formulation, we adopt the \ac{SCA} method to handle \eqref{eqn-6.2} by replacing the original non-convex constraint by the series of convex constraints. Note that the function \me{f(\tilde{\mbf{u}}_{l,k,n})} in \eqref{eqn-6.2} is convex for fixed \me{\mvec{w}{l,k,n}} since it is in fact the ratio between a quadratic form (of \me{\mvec{m}{l,k,n}}) over an affine function (of \me{\beta_{l,k,n}}) \cite{boyd2004convex}. According to the \ac{SCA} method, we relax \eqref{eqn-6.2} to a convex constraint in each iteration of the iterative procedure. Since \me{f(\tilde{\mbf{u}}_{l,k,n})} is convex, a concave approximation of \eqref{eqn-6.2} can be easily found by considering the first order approximation of \me{f(\tilde{\mbf{u}}_{l,k,n})} around the current operation point. For this purpose, let the real and imaginary component of the complex number \me{\mvec{w}{l,k,n}^\herm \mvec{H}{b_k,k,n} \mvec{m}{l,k,n}} be represented by
\begin{subeqnarray} \label{eqn-wsrm-expr}
p_{l,k,n} &\triangleq& \Re \set{{\mvec{w}{l,k,n}^\herm \mvec{H}{b_k,k,n} \mvec{m}{l,k,n}}}, \\
q_{l,k,n} &\triangleq& \Im \set{{\mvec{w}{l,k,n}^\herm \mvec{H}{b_k,k,n} \mvec{m}{l,k,n}}},
\end{subeqnarray}
and hence \me{f(\tilde{\mbf{u}}_{l,k,n})=(p_{l,k,n}^2 + q_{l,k,n}^2)/\beta_{l,k,n}}. Note that \me{p_{l,k,n}} and \me{q_{l,k,n}} are just symbolic notation and not the newly introduced optimization variables. In CVX \cite{grant2008cvx}, for example,  we declare \me{p_{l,k,n}} and \me{q_{l,k,n}} with the `\emph{expression}' qualifier. Suppose that the current value of \me{p_{l,k,n}} and \me{q_{l,k,n}} at a specific iteration are \me{\tilde{p}_{l,k,n}} and \me{\tilde{q}_{l,k,n}}, respectively. Using the first order Taylor approximation around the local point \me{ [ \, \tilde{p}_{l,k,n},\tilde{q}_{l,k,n},\tilde{\beta}_{l,k,n} \, ]^T}, we can approximate \eqref{eqn-6.2} by the following linear inequality constraint
\begin{equation}
\label{eqn-8}
2 \dfrac{\tilde{p}_{l,k,n}}{\tilde{\beta}_{l,k,n}} \left ( p_{l,k,n} - \tilde{p}_{l,k,n} \right ) + 2 \dfrac{\tilde{q}_{l,k,n}}{\tilde{\beta}_{l,k,n}} \left ( q_{l,k,n} - \tilde{q}_{l,k,n} \right ) + \dfrac{\tilde{p}_{l,k,n}^2 + \tilde{q}^2_{l,k,n}}{\tilde{\beta}_{l,k,n}} \left (1 - \dfrac{\beta_{l,k,n} - \tilde{\beta}_{l,k,n}}{\tilde{\beta}_{l,k,n}} \right ) \geq \gamma_{l,k,n}.
\end{equation}
In summary, for the fixed linear receivers, the \ac{JSFRA} problem to find transmit beamformers is shown by
\begin{IEEEeqnarray}{rCl}\label{eqn-9}
\underset{\substack{\gamma_{l,k,n},p_{l,k,n},q_{l,k,n} \\ \mvec{m}{l,k,n}, \beta_{l,k,n}}}{\text{minimize}} &\quad & \| \tilde{\mbf{v}} \|_q \IEEEyessubnumber\label{eqn-9.1a} \\
\text{subject to} & \quad & \beta_{l,k,n} \geq  N_0\|\mvec{w}{l,k,n}\|^2 + \hspace{-0.75em} \sum_{(j,i) \neq (l,k)} \hspace{-0.75em} |\mvec{w}{l,k,n}^\herm \mvec{H}{b_i,k,n} \mvec{m}{j,i,n} |^2, \IEEEyessubnumber \label{eqn-9.1c} \\
& \quad&\sum_{n = 1}^N \sum_{k \in \mathcal{U}_b} \text{tr} \, (\mvec{M}{k,n} \mvec{M}{k,n}^\herm) \leq P_{{\max}}, \fall b, \IEEEyessubnumber \label{eqn-9.1d} \\
& \quad & \text{and } \eqref{eqn-8} \IEEEyessubnumber \label{eqn-9.1e}.
\end{IEEEeqnarray}

The proposed algorithm is referred as \ac{QM} \ac{JSFRA} scheme with a per \ac{BS} power constraint which is outlined in Algorithm \ref{algo-1}. The iterative procedure repeats until the improvement on the objective is less than a predetermined tolerance parameter or the maximum number of iterations is reached. Instead of initializing \me{\tilde{\mbf{u}}_{l,k,n}} arbitrarily to a feasible point, transmit precoders can also be initialized with any feasible point \me{\tilde{\mbf{m}}_{l,k,n}}, which is then used to find \me{\tilde{\mbf{u}}_{l,k,n}} in an efficient manner as briefed in Algorithm \ref{algo-1}. In Algorithm \ref{algo-1}, the \ac{SCA} iterations are carried until convergence or for maximum of \me{I_{\max}} iterations for the optimal \me{\mbf{w}_{l,k,n}} receive beamformers and the outer iterations are for the convergence of the number of queued bits, which is limited by the maximum of \me{J_{\max}} iterations.
\begin{algorithm}
 \SetAlgoLined
 \DontPrintSemicolon
 \BlankLine
 \SetKwInput{KwInit}{Initialize}
 \KwIn{\me{a_k, \, Q_k, \, \mvec{H}{b,k,n},\; \fall b \in \mathcal{B}, \, \fall k \in \mathcal{U}, \fall n \in \mathcal{C}}}
 \KwOut{\me{\mvec{m}{l,k,n}} and \me{\mvec{w}{l,k,n} \fall l \in \set{1,2,\dotsc,L}}}
 \KwInit{\me{i=0}, \me{j=0} and the transmit precoders \me{\tilde{\mbf{m}}_{l,k,n}} randomly satisfying the total power constraint \eqref{eqn-4.3}}
 update \me{\mvec{w}{l,k,n}} with \eqref{eqn-10} and \me{\tilde{\mbf{u}}_{l,k,n}} with \eqref{eqn-8} using \me{\tilde{\mbf{m}}_{l,k,n}}\;
 \Repeat{Queue convergence or \me{j \geq J_{\max}}}{
 \Repeat{\ac{SCA} convergence or \me{i \geq I_{\max}}}{
 solve for the transmit precoders \me{\mvec{m}{l,k,n}} using \eqref{eqn-9} \;
 update the constraint set \eqref{eqn-8} with \me{\tilde{p}_{l,k,n}}, \me{\tilde{q}_{l,k,n}} and \me{\tilde{\beta}_{l,k,n}} using \eqref{eqn-wsrm-expr} with the precoders \me{\mvec{m}{l,k,n}} obtained from the previous step \;
 $i = i + 1$ \;
 }
 update the receive beamformers \me{\mvec{w}{l,k,n}} using \eqref{eqn-10} with the recent precoders \me{\mvec{m}{l,k,n}} \;
 $j = j + 1$ \;
 }
 \caption{Algorithm of \acs{JSFRA} scheme}
 \label{algo-1}
\end{algorithm}

\subsubsection*{Convergence}

In the proposed solution, we replaced \eqref{eqn-6.2} by a convex constraint using the first order approximation, which basically means that we do not solve the problem exactly. According to the traditional \ac{BCDM}, we need to solve a sub problem when fixing a set of variables to the global optimum to ensure the convergence to a stationary point. If we just approximate the objective, then the convergence is guaranteed \cite{razaviyayn2013unified}. In our case, we solve the sub problem inexactly, so the convergence proof of \ac{BCDM} does not apply to our problem. In this problem, we used alternating optimization with \ac{SCA} method, which provides monotonic convergence since the objective is improved at each step \textit{i.e} \me{f^{(i)} \geq f^{(i+1)}}, assuming \me{f^{(i)}} is the objective function at the \me{i^\mathrm{th}} \ac{SCA} iteration. For a fixed receive beamformers, the problem is guaranteed to converge to a \ac{KKT} point at each \ac{SCA} points as discussed in \cite{marks1978technical}. Now, by alternating the transmit and the receive precoders, we perform \ac{BCDM}, which provides monotonic convergence of the objective. Since the \ac{SCA} method is also considered in our problem, the convergence is not straight forward. We can only speculate the convergence based on the monotonicity of the objective values. Since receive beamformers are based on the \ac{MMSE} objective, it improves the objective value for each update, providing a monotonic increase in the objective value.

