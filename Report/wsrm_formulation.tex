
We now present an algorithm to solve problem \eqref{eqn-3} by using alternating optimization technique in conjunction with \acf{SCA} \cite{marks1978technical}. To do this, first by using the \ac{SINR} expression in \eqref{eq:SINR}, we equivalently reformulate problem \eqref{eqn-3} as
\iftoggle{single_column}{
\begin{IEEEeqnarray}{CCl}\label{eqn-6}
\underset{\substack{\gamma_{l,k,n},\mvec{m}{l,k,n},\\ \beta_{l,k,n},\mvec{w}{l,k,n}}}{\text{minimize}} & \quad & \|  \tilde{\mbf{v}}  \|_q \label{eqn-obj} \IEEEyessubnumber \\
\text{subject to}& \quad &\gamma_{l,k,n} \leq \frac{ | \mvec{w}{l,k,n}^\herm \mvec{H}{l,k,n} \mvec{m}{l,k,n}  |^2}{\beta_{l,k,n}} \IEEEyessubnumber \label{eqn-6.2} \\
  & \quad & \beta_{l,k,n} \geq  \enoise + \hspace{-0.75em} \sum_{(j,i) \neq (l,k)} \hspace{-0.75em} |\mvec{w}{l,k,n}^\herm \mvec{H}{b_i,k,n} \mvec{m}{j,i,n} |^2 \IEEEyessubnumber \label{eqn-6.3} \\
  & \quad & \sum_{n = 1}^N \sum_{k \in \mathcal{U}_b} \sum_{l=1}^L \text{tr} \, (\mvec{m}{l,k,n} \mvec{m}{l,k,n}^\herm) \leq P_{{\max}}, \fall b. \IEEEyessubnumber \label{eqn-6.4}
\end{IEEEeqnarray}
}{\allowdisplaybreaks
\begin{IEEEeqnarray}{CCl}\label{eqn-6}
	\underset{\substack{\gamma_{l,k,n},\mvec{m}{l,k,n},\\\beta_{l,k,n},\mvec{w}{l,k,n}}}{\text{minimize}} &\quad& \|  \tilde{\mbf{v}}  \|_q \label{eqn-obj} \IEEEyessubnumber \vspace{-0.25cm}\\
	\text{subject to} &\quad& \gamma_{l,k,n} \leq \frac{ | \mvec{w}{l,k,n}^\herm \mvec{H}{l,k,n} \mvec{m}{l,k,n}|^2}{\beta_{l,k,n}} \IEEEyessubnumber \eqspace \label{eqn-6.2} \\
	&\quad& \beta_{l,k,n} \geq \enoise + \sum_{\mathclap{(j,i) \neq (l,k)}} |\mvec{w}{l,k,n}^\herm \mvec{H}{b_i,k,n} \mvec{m}{j,i,n} |^2 \IEEEyessubnumber \eqspace \label{eqn-6.3} \\
	&\quad& \sum_{n = 1}^N \sum_{k \in \mathcal{U}_b} \sum_{l=1}^L \text{tr} \, (\mvec{m}{l,k,n} \mvec{m}{l,k,n}^\herm) \leq P_{{\max}}, \fall b. \IEEEyessubnumber \eqspace \label{eqn-6.4}
\end{IEEEeqnarray}
}
In this formulation, we relaxed the \ac{SINR} expression \eqref{eq:SINR} by the inequalities in \eqref{eqn-6.2} and \eqref{eqn-6.3}. \review{Note that \eqref{eqn-6.2} is an under estimator for \ac{SINR} \me{\gamma_{l,k,n}}, and \eqref{eqn-6.3} provides an upper bound for the total interference seen by user \me{k \in \mathcal{U}_b}, denoted by variable \me{\beta_{l,k,n}}}. \review{The constraints are tight when the system is power limited or when the total number of backlogged packets is larger than the transmission rate.}

In order to find a tractable solution for problem \eqref{eqn-6}, we note that the constraints \eqref{eqn-6.4} are convex with involved variables. Thus, we only need to deal with \eqref{eqn-6.2} and \eqref{eqn-6.3}. Towards this end, we resort to the traditional coordinate descent technique by fixing the linear receivers, and finding the optimal transmit beamformers. \review{Note that the problem in \eqref{eqn-6} is similar to the \ac{WSRM} formulation, which is known to be NP-hard even for the single antenna case \cite{np_hard}.} % Recall that the coordinate descent method assumes that the optimization variables belong to disjoint sets and the problem is convex for a variable while all other variables are fixed \cite{xu2013block}.

For a fixed receivers \me{\mvec{w}{l,k,n}}, the problem now is to find the optimal transmit beamformers \me{\mvec{m}{l,k,n}} which is still a challenging task. We note that for fixed \me{\mvec{w}{l,k,n}}, \eqref{eqn-6.3} can be written as a \ac{SOC} constraint. Thus, the difficulty is due to the non-convexity in \eqref{eqn-6.2}. \review{To arrive at a tractable formulation, we adopt \ac{SCA} to handle \eqref{eqn-6.2} by replacing the original non-convex constraint by a series of convex constraints \cite{marks1978technical}}. Let us define a function, 
\begin{equation*}
f({\mbf{u}}_{l,k,n}) \triangleq \frac{ | \mvec{w}{l,k,n}^\herm \mvec{H}{l,k,n} \mvec{m}{l,k,n}|^2}{\beta_{l,k,n}},
\end{equation*}
where \me{{\mbf{u}}_{l,k,n} \triangleq \{\mvec{w}{l,k,n}, \mvec{m}{l,k,n},\beta_{l,k,n}\}} is the vector which needs to be identified for the optimal allocation. Note that the function \me{f({\mbf{u}}_{l,k,n})} is convex for fixed \me{\mvec{w}{l,k,n}}, since it is in fact the ratio between a quadratic form of \me{\mvec{m}{l,k,n}} over an affine function of \me{\beta_{l,k,n}} \cite{boyd2004convex}. Eq. \eqref{eqn-6.2} can be viewed as a \ac{DC} constraint. Now the convex function given by \me{f({\mbf{u}}_{l,k,n})} is approximated by considering the first order approximation around the operating point \me{\tilde{\mbf{u}}_{l,k,n}}.

For this purpose, let the real and imaginary component of the complex number \me{\mvec{w}{l,k,n}^\herm \mvec{H}{b_k,k,n} \mvec{m}{l,k,n}} be represented by
\begin{subeqnarray} \label{eqn-wsrm-expr}
p_{l,k,n} &\triangleq& \Re \set{{\mvec{w}{l,k,n}^\herm \mvec{H}{b_k,k,n} \mvec{m}{l,k,n}}} \\
q_{l,k,n} &\triangleq& \Im \set{{\mvec{w}{l,k,n}^\herm \mvec{H}{b_k,k,n} \mvec{m}{l,k,n}}}
\end{subeqnarray}
and hence \me{f({\mbf{u}}_{l,k,n})=(p_{l,k,n}^2 + q_{l,k,n}^2)/\beta_{l,k,n}}\footnote{Note that \me{p_{l,k,n}} and \me{q_{l,k,n}} are just symbolic notation and not the newly introduced optimization variables. In CVX \cite{grant2008cvx}, for example,  we declare \me{p_{l,k,n}} and \me{q_{l,k,n}} with the `\emph{expression}' qualifier}. Suppose that the current value of \me{p_{l,k,n}} and \me{q_{l,k,n}} at a specific iteration are \me{\tilde{p}_{l,k,n}} and \me{\tilde{q}_{l,k,n}}, respectively. Using the first order Taylor approximation around the local point \me{ [ \, \tilde{p}_{l,k,n},\tilde{q}_{l,k,n},\tilde{\beta}_{l,k,n} \, ]^T}, we can approximate \eqref{eqn-6.2} by the following linear inequality constraint as
\iftoggle{single_column}{
\begin{equation}\label{eqn-8}
	2 \frac{\tilde{p}_{l,k,n}}{\tilde{\beta}_{l,k,n}} \left ( p_{l,k,n} - \tilde{p}_{l,k,n} \right ) + 2 \frac{\tilde{q}_{l,k,n}}{\tilde{\beta}_{l,k,n}} \left ( q_{l,k,n} - \tilde{q}_{l,k,n} \right ) + \frac{\tilde{p}_{l,k,n}^2 + \tilde{q}^2_{l,k,n}}{\tilde{\beta}_{l,k,n}} \left (1 - \frac{\beta_{l,k,n} - \tilde{\beta}_{l,k,n}}{\tilde{\beta}_{l,k,n}} \right ) \geq \gamma_{l,k,n}.
\end{equation}
}{
\begin{multline}\label{eqn-8}
2 \frac{\tilde{p}_{l,k,n}}{\tilde{\beta}_{l,k,n}} \left ( p_{l,k,n} - \tilde{p}_{l,k,n} \right ) + 2 \frac{\tilde{q}_{l,k,n}}{\tilde{\beta}_{l,k,n}} \left ( q_{l,k,n} - \tilde{q}_{l,k,n} \right ) \\
+ \frac{\tilde{p}_{l,k,n}^2 + \tilde{q}^2_{l,k,n}}{\tilde{\beta}_{l,k,n}} \left (1 - \frac{\beta_{l,k,n} - \tilde{\beta}_{l,k,n}}{\tilde{\beta}_{l,k,n}} \right ) \geq \gamma_{l,k,n}.
\end{multline}
}
In summary, for the fixed linear receivers \review{\me{\mvec{w}{l,k,n}}}, the \ac{JSFRA} problem to find transmit beamformers is shown by
\begin{IEEEeqnarray}{CCl}\label{eqn-9}
\underset{\substack{\mvec{m}{l,k,n},\\ \gamma_{l,k,n},\beta_{l,k,n}}}{\text{minimize}} &\quad & \| \tilde{\mbf{v}} \|_q \IEEEyessubnumber\label{eqn-9.1a} \vspace{-0.15cm} \\
\text{subject to} & \quad & \beta_{l,k,n} \geq  \enoise + \hspace{-0.75em} \sum_{(j,i) \neq (l,k)} \hspace{-0.75em} |\mvec{w}{l,k,n}^\herm \mvec{H}{b_i,k,n} \mvec{m}{j,i,n} |^2 \IEEEyessubnumber \eqspace \label{eqn-9.1c} \\
& \quad&\sum_{n = 1}^N \sum_{k \in \mathcal{U}_b} \sum_{l=1}^L \text{tr} \, (\mvec{m}{l,k,n} \mvec{m}{l,k,n}^\herm) \leq P_{{\max}}, \fall b, \IEEEyessubnumber \label{eqn-9.1d} \\
& \quad & \text{and } \eqref{eqn-8}. \IEEEyessubnumber \label{eqn-9.1e}
\end{IEEEeqnarray}

Now, the optimal linear receivers for the fixed transmit precoders \me{\mbf{m}_{j,i,n} \, \forall i \in \mc{U}, \, \forall n \in \mc{C}} are obtained by minimizing \eqref{eqn-3} with respect to \me{\mbf{w}_{l,k,n}} as
\begin{IEEEeqnarray}{CCl}\label{eqn-9--1}
\underset{\substack{\gamma_{l,k,n},\\ \mvec{w}{l,k,n},\beta_{l,k,n}}}{\text{minimize}} &\quad & \| \tilde{\mbf{v}} \|_q \IEEEyessubnumber\label{eqn-9--1.1a} \\
\text{subject to} & \quad & \beta_{l,k,n} \geq  \enoise + \hspace{-0.75em} \sum_{(j,i) \neq (l,k)} \hspace{-0.75em} |\mvec{w}{l,k,n}^\herm \mvec{H}{b_i,k,n} \mvec{m}{j,i,n} |^2 \IEEEyessubnumber \eqspace \label{eqn-9--1.1c} \\
& \quad&\sum_{n = 1}^N \sum_{k \in \mathcal{U}_b} \sum_{l=1}^L \text{tr} \, (\mvec{m}{l,k,n} \mvec{m}{l,k,n}^\herm) \leq P_{{\max}}, \fall b, \IEEEyessubnumber \label{eqn-9--1.1d} \\
& \quad & \text{and } \eqref{eqn-8}. \IEEEyessubnumber \label{eqn-9--1.1e}
\end{IEEEeqnarray}
Solving \eqref{eqn-9--1} using the \ac{KKT} conditions, we obtain the following iterative expression for the receiver \me{\mvec{w}{l,k,n}} as
\begin{IEEEeqnarray}{rCl}
\mvec{\widetilde{R}}{l,k,n} &=& \displaystyle \sum_{\mathclap{(j,i)\neq (l,k)}} \mvec{H}{b_i,k,n} {\mbf{m}}_{j,i,n} {\mbf{m}}_{j,i,n}^\herm \mvec{H}{b_i,k,n}^\herm + N_0 \, \mathbf{I}_{N_R} \IEEEyessubnumber \\
\mvec{w}{l,k,n}^{(i)} &=& \left ( \tfrac{\tilde{\beta}_{l,k,n} {\mbf{m}}_{l,k,n}^\herm \mvec{H}{b_k,k,n}^\herm \mbf{w}_{l,k,n}^{(i-1)} }{\|\mvec{w}{l,k,n}^{(i-1)} \mvec{H}{b_k,k,n} {\mbf{m}}_{l,k,n} \|^2} \right )\mvec{\widetilde{R}}{l,k,n}^{-1}\mvec{H}{b_k,k,n} {\mbf{m}}_{l,k,n}, \IEEEyessubnumber \eqspace \label{opt-rx}
\end{IEEEeqnarray}
where \me{\mvec{w}{l,k,n}^{(i-1)}} is the receive beamformer from the earlier iteration, upon which the linear relaxation is performed for the nonconvex constraint in \eqref{eqn-9--1}. Eq. \eqref{opt-rx} is obtained by iterating over the fixed \me{\mvec{w}{l,k,n}^{(i-1)}} at each \ac{SCA} iteration until convergence or for fixed number of iterations. \review{Note that the optimal receiver has no explicit relation with the choice of the \me{q} value used in the objective function. The dependency is implied implicitly through the transmit precoders \me{\mvec{m}{l,k,n}}, which in deed depend on the \me{q} value.}

It can be seen that the optimal receiver expression in \eqref{opt-rx} is in fact a scaled version of the \ac{MMSE} receiver, which is given by
\begin{IEEEeqnarray}{rCl}
\mvec{R}{l,k,n} &=& \displaystyle \sum_{i\in \mc{U}} \sum_{j=1}^L \mvec{H}{b_i,k,n} \mvec{m}{j,i,n} \mvec{m}{j,i,n}^\herm \mvec{H}{b_i,k,n}^\herm + N_0 \, \mathbf{I}_{N_R} \IEEEyessubnumber \eqspace \\
\mvec{w}{l,k,n} &=& \mathbf{R}^{-1}_{l,k,n} \; \mvec{H}{b_k,k,n} \; \mvec{m}{l,k,n}. \IEEEyessubnumber \label{eqn-10}
\end{IEEEeqnarray}
\review{Since the scaling present in the optimal receiver \eqref{opt-rx} has no impact on the received \acp{SINR}, the \ac{MMSE} receiver in \eqref{eqn-10} can also be used without compromising the performance or the convergence behavior}.

The proposed algorithm is referred to as \acl{QM} \ac{JSFRA} scheme with a per \ac{BS} power constraint, and it is outlined in Algorithm \ref{algo-1}. The iterative procedure repeats until the improvement on the objective is less than a predetermined tolerance parameter or the maximum number of iterations is reached. Instead of initializing \me{{\mbf{u}}_{l,k,n}} arbitrarily to a feasible point, transmit precoders can also be initialized with any feasible point \me{\tilde{\mbf{m}}_{l,k,n}}, which is then used to find \me{{\mbf{u}}_{l,k,n}} as briefed in Algorithm \ref{algo-1}. For a fixed receive beamformer \me{\mvec{w}{l,k,n}}, the \ac{SCA} iteration is carried out until convergence or for the predefined iterations, say, \me{J_{\max}} for the optimal transmit precoders \me{\mvec{m}{l,k,n}}. Next, the receive beamformers are updated based on either \eqref{opt-rx} or \eqref{eqn-10} using the fixed transmit precoders \me{\mvec{m}{l,k,n}}. This procedure is carried out until convergence of the queue deviation or for fixed number of iterations by \me{I_{\max}} as outlined in Algorithm \ref{algo-1}.
{\allowdisplaybreaks
\begin{algorithm}
 \SetAlgoLined
 \DontPrintSemicolon
 \BlankLine
 \SetKwInput{KwInit}{Initialize}
 \KwIn{\me{a_k, \, Q_k, \, \mvec{H}{b,k,n},\; \fall b \in \mathcal{B}, \, \fall k \in \mathcal{U}, \fall n \in \mathcal{N}}}
 \KwOut{\me{\mvec{m}{l,k,n}} and \me{\mvec{w}{l,k,n} \fall l \in \set{1,2,\dotsc,L}}}
 \KwInit{\me{i=0} and transmit precoders \me{\tilde{\mbf{M}}_{k,n}} randomly satisfying the total power constraint \eqref{eqn-4.3}}
 update \me{\mvec{W}{k,n}} and \me{{\mbf{u}}_{l,k,n}} using \eqref{eqn-10} and \eqref{eqn-8} using \me{\tilde{\mbf{M}}_{k,n}}\;
 \Repeat{Queue convergence or \me{i \geq I_{\max}}}{
 initialize \me{j = 0}\;
 \Repeat{\ac{SCA} convergence or \me{j \geq J_{\max}}}{
 solve for the transmit precoders \me{\mvec{m}{l,k,n}} using \eqref{eqn-9}\;
 update the constraint set \eqref{eqn-8} with \me{{\mbf{u}}_{l,k,n}} and \me{\mvec{m}{l,k,n}} using \eqref{eqn-wsrm-expr}\;
 $j = j + 1$\;
 }
 update the receive beamformers \me{\mvec{w}{l,k,n}} using \eqref{eqn-9--1} or \eqref{eqn-10} with the updated precoders \me{\mvec{m}{l,k,n}}\;
 $i = i + 1$\;
 }
 \caption{Algorithm of \acs{JSFRA} scheme}
 \label{algo-1}
\end{algorithm}
}

\begin{comment}
\subsubsection*{Convergence}
In order to prove the convergence of the proposed iterative algorithm, following conditions are to be satisfied \cite{scutari}
\begin{itemize}
	\item convergence of the \ac{SCA} subproblem	
	\item uniqueness of the transmit and the receive beamformers
	\item monotonic convergence of the objective function
\end{itemize}
In the proposed solution, we replaced \eqref{eqn-6.2} by a convex constraint using the first order approximation, which is majorized by the quadratic-over-linear function in \eqref{eqn-6.2} from below around a fixed point \me{\tilde{\mbf{u}}^{(i)}_{l,k,n}}. Since the \ac{SCA} method is adopted in the proposed algorithm, the constraint approximation satisfies the following conditions as in \cite{marks1978technical}
\begin{subeqnarray} \label{sca-req}
	f(\tilde{\mbf{u}}_{l,k,n}) &\leq& \bar{f}(\tilde{\mbf{u}}_{l,k,n},\tilde{\mbf{u}}^{(i)}_{l,k,n}) \\
	f(\tilde{\mbf{u}}^{(i)}_{l,k,n}) &=& \bar{f}(\tilde{\mbf{u}}^{(i)}_{l,k,n},\tilde{\mbf{u}}^{(i)}_{l,k,n}) \\
	\nabla f(\tilde{\mbf{u}}^{(i)}_{l,k,n}) &=& \nabla \bar{f}(\tilde{\mbf{u}}^{(i)}_{l,k,n},\tilde{\mbf{u}}^{(i)}_{l,k,n}),
\end{subeqnarray}
where \me{\bar{f}(\mbf{x},\mbf{x}^{(i)})} is the approximate function of \me{f(\mbf{x})} around the point \me{\mbf{x}^{\ast(i)}}. The stationary point of the relaxed convex problem satisfies the \ac{KKT} conditions of the original nonconvex problem, which can be obtained by using conditions in \eqref{sca-req}. It can be seen that the  \ac{SCA} relaxed formulation converges to a local stationary point at each iteration.

The uniqueness of the transmit and the receive beamformers can be justified by forcing one antenna to be real valued to exclude the phase ambiguity arising from the complex precoders. The monotonic convergence of the objective function can be justified by the following arguments. At each \ac{SCA} iteration, the relaxed subproblem is solved for the locally optimal transmit precoders to minimize the objective function. Since the \ac{SCA} subproblem is relaxed around the \me{\ith{i-1}} optimal point, \textit{i.e},  \me{\mbf{x}^{\ast(i-1)}} for the \me{\ith{i}} iteration, the domain of the problem in the \me{\ith{i}} step includes optimal point from the  \me{\ith{i-1}} iteration as well. Therefore, at each \ac{SCA} step, the objective function can either be equal to or smaller than the previous value, thereby leading to the monotonic convergence of the objective function.

Once the problem is converged to a stationary transmit precoders, the receive beamformers are updated based on the receivers in \eqref{opt-rx} or \eqref{eqn-10}. The monotonic nature of the objective function is preserved by the receive beamformer update, since the receiver minimizes the objective value for the fixed transmit precoders, and hence the proposed \ac{JSFRA} scheme is guaranteed to converge to a stationary point of the original nonconvex problem.

\end{comment}