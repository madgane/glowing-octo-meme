
In this paper, we address the problem of resource allocation across the \ac{MIMO} \ac{OFDM} framework, which are contended by the users served by multiple \acp{BS}. In order to minimize the number of queued packets, the precoders are designed to utilize the available space-frequency resources efficiently by limiting the over utilization of the same. In order to achieve this, we use the queue deviation utility as compared to the traditional \ac{WSRM} objective discussed extensively in the literature \cite{christensen2008weighted,wmmse_shi}. For practical and tractability reasons, we impose a constraint that the maximum number of transmitted bits for the user \me{k} is limited by the total backlogged packets available at the transmitter. As a result, the number of backlogged packets \me{v_k} remaining in the system for the user \me{k} is given by
\begin{equation}
v_k =  Q_k - \sum_{n = 1}^N \sum_{l = 1}^{L} \log_2(1+\gamma_{l,k,n}) \geq 0 \fall k \in \mathcal{U}.
\label{eqn-4.2}
\end{equation}
The above positivity constraint need to be satisified by \me{v_k} to avoid the excessive allocation of the resources.

The objective function for the precoder design problem is given by minimizing the absolute value of the queue deviation in \eqref{eqn-4.2} raised to the exponent \me{q} as \me{\text{minimize} \: \sum_{k \in \mc{U}} \, | v_k |^q}. With this objective, the problem of weighted queued packet minimization is given by the $q$-norm minimization as
\begin{IEEEeqnarray}{rCl}\label{eqn-3}
\underset{\substack{\mvec{M}{k,n},\gamma_{l,k,n}\\\mvec{W}{k,n}}}{\text{minimize}} &\quad& \|  \tilde{\mbf{v}}  \|_q\IEEEyessubnumber \\
\text{subject to} & \quad&\sum_{n = 1}^N \sum_{k \in \mathcal{U}_b} \text{tr} \, (\mvec{M}{k,n} \mvec{M}{k,n}^\herm) \leq P_{{\max}}, \fall b, \IEEEyessubnumber \label{eqn-4.3}
\end{IEEEeqnarray}
where \me{\tilde{v}_k \triangleq a_k^{1/{q}}v_k}, \me{a_k} is the weighting factor which is incorporated to control user priority based on their respective \ac{QoS}, \me{\gamma_{l,k,n}} is defined in \eqref{eq:SINR}, \me{\mbf{M}_{k,n} \triangleq [ \, \mvec{m}{1,k,n} \, \mvec{m}{2,k,n} \dotsc \mvec{m}{L,k,n} \,]} comprises the beamformers associated with the user \me{k} for \me{n^\mathrm{th}} sub-channel transmission, and \me{\mbf{W}_{k,n} \triangleq [ \, \mvec{w}{1,k,n} \, \mvec{w}{2,k,n} \dotsc \mvec{w}{L,k,n} \,]} stacks the receive beamformers respectively. It can be easily extended for user specific streams \me{L_k} instead of using the common \me{L} streams for all users. In \eqref{eqn-4.3}, we consider the sum power constraint for each \ac{BS} across all sub-channels. Before discussing the solutions in detail, we discuss the existing algorithm to solve the issue of minimizing the number of backlogged packets with additional constraints required by problem.
