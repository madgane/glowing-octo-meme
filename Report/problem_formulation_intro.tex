To minimize the total number of backlogged packets, we consider minimizing the weighted \me{\ell_{q}}-norm of the queue deviation objective as
\begin{equation}
v_k =  Q_k - t_k = Q_k - \sum_{n = 1}^N \sum_{l = 1}^{L} \log_2(1+\gamma_{l,k,n})
\label{eqn-4.2}
\end{equation}
where \me{\gamma_{l,k,n}} is given by \eqref{eq:SINR} and the optimization variables are the transmit precoders \me{\mvec{m}{l,k,n}} and the receivers \me{\mvec{w}{l,k,n}}.

Explicitly, the objective of the problem considered is given as \me{\sum_{k \in \mc{U}} \, a_k | v_k |^q}. Thus the formulation becomes
\begin{IEEEeqnarray}{rCl}\label{eqn-3}
\underset{\substack{\mvec{m}{l,k,n},\mvec{w}{l,k,n}}}{\text{minimize}} &\quad& \|  \tilde{\mbf{v}}  \|_q\IEEEyessubnumber \label{eqn-3-1.a} \\
\text{subject to} & \quad&\sum_{n = 1}^N \sum_{k \in \mathcal{U}_b} \sum_{l=1}^L \trace \, (\mvec{m}{l,k,n} \mvec{m}{l,k,n}^\herm) \leq P_{{\max}} \fall b \IEEEyessubnumber \eqspace \label{eqn-4.3}
\end{IEEEeqnarray}
where \me{\tilde{v}_k \triangleq a_k^{1/{q}}v_k} is the element of vector \me{\tilde{\mbf{v}}}, and \me{a_k} is the weighting factor, which is used to alter the user priority based on the \ac{QoS} constraints such as packet delay requirements and packet waiting time, since they are proportional to the corresponding number of backlogged packets. The \ac{BS} specific power constraint for all sub-channels is considered in \eqref{eqn-4.3}.

For practical reasons, we impose a constraint on the maximum number of transmitted bits for the user \me{k}, since it is limited by the total number of backlogged packets available at the transmitter. As a result, the number of backlogged packets \me{v_k} for user \me{k} remaining in the system is given by
\begin{equation} \label{rate_constraint_a}
v_k =  Q_k - \sum_{n = 1}^N \sum_{l = 1}^{L} \log_2(1+\gamma_{l,k,n}) \geq 0.
\end{equation}
The above positivity constraint need to be satisfied by \me{v_k} to avoid the excessive allocation of the resources.

Before proceeding further, we show that the constraint in \eqref{rate_constraint_a} is handled implicitly by the definition of norm \me{\ell_q} in the objective of \eqref{eqn-3}. Suppose that \me{t_k>Q_k} for certain \me{k} at the optimum, i.e., \me{-{v}_k=t_k-Q_k>0}. Then there exists \me{\delta_k>0} such that \me{-{v}^{\prime}_k=t^{\prime}_k-Q_k<-{v}_k} where \me{t^{\prime}_k=t_k-\delta_k}. Since \me{\|\tilde{\mbf{v}}\|_q=\| |\tilde{\mbf{v}}| \|_q=\||-\tilde{\mbf{v}}|\|_q}, this means that the newly created vector \me{\mbf{t}^{\prime}} achieves a smaller objective which contradicts with the fact that the optimal solution has been obtained. The choice of the norm \me{\ell_q} used in the objective function \cite{berry2004cross,qps_cioffi} alters the priorities for the queue deviation function as follows.
\begin{itemize}
\item \me{\ell_{1}} results in greedy allocation \textit{i.e}, emptying the queue of users with good channel conditions before considering the users with worse channel conditions. As a special case, it is easy to see that \eqref{eqn-3} reduces to the \ac{WSRM} problem when the queue size is large enough for all users.
\item \me{\ell_{2}} prioritizes users with a higher number of queued packets before considering the users with a smaller number of backlogged packets. For example, it could be more ideal for the delay limited scenario when the packet arrival rates of the users are similar, since the number of backlogged packets is proportional to the delay in the transmission following the Little's law \cite{neely2010stochastic}.
\item \(\ell_{\infty} \) minimizes the maximum number of queued packets among users with the current transmission, thereby providing queue fairness by allocating the resources proportional to the number of backlogged packets.
\end{itemize} 