The problem defined in \eqref{q_gen_sum} ignores the second order term arising from the Lyapunov drift minimization objective by the limiting it to a constant value. In fact, Eq. \eqref{eqn-4.2} provides similar expression when the exponent is set to be \me{\ell_{q=2}} as
\begin{equation}\label{pf-1}
\underset{t_k}{\text{minimize}} \, \sum_k \, v_k^2 = \underset{t_k}{\text{minimize}} \, \sum_k \, Q_k^2 - 2 \, Q_k t_k + t_k^2.
\end{equation}

It is evident that \eqref{pf-1} is equivalent to \eqref{backpressure-approx} if the second order terms are ignored. Limiting \me{t_k^2} by a constant value, the \ac{Q-WSRM} formulation requires the explicit rate constraint \eqref{eqn-3.1.4} to avoid the resource wastage in the form of over allocation. In the proposed queue deviation formulation, the explicit rate constraint is not needed, since it is handled by the objective function itself. This makes the problem simpler and allows us to employ efficient algorithms to distribute the precoder design problem across each \acp{BS} independently by exchanging minimal information exchange \cite{boyd2011distributed}. In contrast to the \ac{WSRM} formulation, the \ac{JSFRA} and the \ac{Q-WSRM} problems include the sub-channels jointly to achieve an efficient allocation by identifying the optimal space-frequency resource for each user in the system. The queue deviation objective provides an alternative approach to perform the resource allocation without the additional rate constraints as in \ac{Q-WSRME} approach.
