In order to minimize the number of queued packets at the \acp{BS}, the precoders \me{\mbf{M}_{k,n}} are designed to distribute the resources in an efficient manner to the users by utilizing both spatial and frequency dimension. The problem defined in \eqref{q_gen_sum} raises a question, namely, the linearity of the weights used in the objective function. It can also be a quadratic function, providing more emphasis on the users with large number queued packets as compared to the users with the small queue sizes or some other function of \me{\mbf{Q}}. In order to answer this question, we discuss the \ac{JSFRA} problem formulation, which restricts the solution only to the power functions. For practical and tractability reasons, we impose a constraint that the maximum number of transmitted bits for the user \me{k} is limited by the packets available at the transmitter. As a result, the number of backlogged packets remaining in the system is given by
\begin{equation}
v_k =  Q_k - \sum_{n = 1}^N \sum_{l = 1}^{L} \log_2(1+\gamma_{l,k,n}) \geq 0 \fall k \in \mathcal{U}.
\label{eqn-4.2}
\end{equation}

The precoder design problem can be given minimizing the absolute value of the deviation in \eqref{eqn-4.2} raised to the exponent \me{q} as
\me{\text{minimize} \: \sum_{k \in \mc{U}} \, | v_k |^q}. The exponent \me{q} plays a vital role in the final allocation, which will be discussed later. Now, the problem of weighted queued packet minimization formulated as a $q$-norm minimization, is given by
\begin{IEEEeqnarray}{rCl}\label{eqn-3}
\underset{\substack{\mvec{m}{l,k,n},\\\mvec{w}{l,k,n}}}{\text{minimize}} &\quad& \|  \tilde{\mbf{v}}  \|_q\IEEEyessubnumber \\
\text{subject to} & \quad&\sum_{n = 1}^N \sum_{k \in \mathcal{U}_b} \text{tr} \, (\mvec{M}{k,n} \mvec{M}{k,n}^\herm) \leq P_{{\max}}, \fall b, \IEEEyessubnumber \label{eqn-4.3}
\end{IEEEeqnarray}
where \me{\tilde{v}_k \triangleq a_k^{1/{q}}v_k}, \me{a_k} is the weighting factor which is incorporated to control user priority based on their respective \ac{QoS}, \me{\gamma_{l,k,n}} is defined in \eqref{eq:SINR}, and \me{\mbf{M}_{k,n} \triangleq [ \, \mvec{m}{1,k,n} \, \mvec{m}{2,k,n} \dotsc \mvec{m}{L,k,n} \,]} comprises the beamformers associated with the user \me{k} for \me{n^\mathrm{th}} sub-channel transmission. It can be easily extended for user specific streams \me{L_k} instead of using the common \me{L} streams for all users. The expression for \me{t_{l,k,n}} is due to the assumption of Gaussian signaling, so that the maximum achievable rate is \me{\log_2(1 + \gamma_{l,k,n})} for a given \ac{SINR} \me{\gamma_{l,k,n}}. In \eqref{eqn-4.3}, we consider the sum power constraint for each \ac{BS} across all sub-channels.
