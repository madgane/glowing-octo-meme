The problem defined in \eqref{q_gen_sum} ignores the second order term arising from the Lyapunov drift minimization objective by limiting it to a constant value. \review{In fact, using \me{\ell_{q=2}} in \eqref{eqn-4.2}, we obtain the following objective
\begin{equation}\label{pf-1}
\underset{t_k}{\text{minimize}} \, \sum_k \, v_k^2 = \underset{t_k}{\text{minimize}} \, \sum_k \, Q_k^2 - 2 \, Q_k t_k + t_k^2
\end{equation}
which is similar to the objective in \eqref{q_gen_sum}. It is achieved either by removing \me{t_k^2} from \eqref{pf-1} or when the total number of queued packets is large for all users such that \me{t_k^2} has no impact on the objective function.}

By limiting \me{t_k^2} with a constant value, the \ac{Q-WSRM} formulation requires an explicit rate constraint \eqref{eqn-3.1.4} to avoid over-allocation of the available resources. In the proposed queue deviation formulation, the explicit rate constraint is not needed, since it is handled by the objective function \eqref{eqn-4.2} itself. It makes the problem simpler and allows us to employ efficient algorithms to distribute the precoder design problem across each \ac{BS} independently with minimal information exchange \cite{boyd2011distributed}. In contrast to the \ac{WSRM} formulation, the \ac{JSFRA} and the \ac{Q-WSRME} problems handle the sub-channels jointly to obtain an efficient allocation by identifying the optimal space-frequency resources for the users.
