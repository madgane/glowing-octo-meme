
The problem defined in \eqref{q_gen_sum} ignores the second order term arising from the Lyapunov drift minimization objective by the limiting it to a constant value. Eq. \eqref{eqn-4.2} provides similar expression when the exponent is set to be \me{\ell_{q=2}} as
\begin{equation}\label{pf-1}
\underset{t_k}{\text{minimize}} \, \sum_k \, v_k^2 = \underset{t_k}{\text{minimize}} \, \sum_k \, Q_k^2 - 2 \, Q_k t_k + t_k^2.
\end{equation}
It is evident that \eqref{pf-1} is equivalent to \eqref{backpreassure} when we ignore the second order terms from \eqref{pf-1}. Limiting \me{t_k^2} by a constant value, imposes the additional rate constraint \eqref{eqn-3.1.4} in the \ac{Q-WSRM} formulation to avoid the resource wastage in the form of over allocation. In the current queue deviation formulation, the explicit rate constraint is not needed, since it is handled by the objective function itself. In contrast to the \ac{WSRM} formulation, the \ac{JSFRA} and the \ac{Q-WSRM} problems include the sub-channels jointly to achive an efficient allocation by identifying the optimal space-frequency resource for each user in the system. The queue deviation objective provides an alternate and a more efficient way to perform the resource allocation as compared to the \eqref{q_gen_sum} approach.

Before proceeding further, we note that the constraint in \eqref{eqn-4.2} is handled implicitly by the definition of the \me{\ell_q} in the objective of \eqref{eqn-3}. As a proof, suppose that \me{t_k>Q_k} for a certain \me{k} at optimum, i.e., \me{-{v}_k=t_k-Q_k>0}. Then there exists \me{\delta_k>0} such that \me{-{v}^{\prime}_k=t^{\prime}_k-Q_k<-{v}_k} where \me{t^{\prime}_k=t_k-\delta_k}. Since \me{\|\tilde{\mbf{v}}\|_q=\| |\tilde{\mbf{v}}| \|_q=\||-\tilde{\mbf{v}}|\|_q}, this means that the newly created vector \me{\mbf{t}^{\prime}} achieves a smaller objective which contradicts with the fact that an optimal solution has been obtained. The choice of the norm \me{\ell_q} used in the objective function \cite{berry2004cross,qps_cioffi} alters the priorities for the queue deviation function as
\begin{itemize}
\item With \me{\ell_{q = 1}}, the objective results in greedy allocation \textit{i.e}, emptying the queue of users with good channel condition before considering the users with worse channel conditions. As a special case, it is easy to see that \eqref{eqn-3} reduces to the \ac{WSRM} problem \eqref{gen_sum} when the queue size is large enough for all users.
\item With \me{\ell_{q = 2}}, the objective prioritizes users with higher number of queued packets before considering the users with a smaller number of backlogged packets. For example, it could be more ideal for the delay limited scenario when the packet arrival rates of the users are similar, since the backlogged packets is proportional to the delay in the transmission following the Little's law \cite{neely2010stochastic}.
\item With \(\ell_{q = \infty} \), the objective minimizes the maximum number of queued packets among users with the current transmission, thereby providing queue fairness by allocating the resources proportional to the number of backlogged packets.
\end{itemize}