The problem defined in \eqref{q_gen_sum} ignores the second order term arising from the Lyapunov drift minimization objective by limiting it to a constant value. In fact, by using \me{\ell_{q=2}} norm in \eqref{eqn-4.2}, we obtain the objective function, similar to \eqref{q_gen_sum} as
\begin{equation}\label{pf-1}
\underset{t_k}{\text{minimize}} \, \sum_k \, v_k^2 = \underset{t_k}{\text{minimize}} \, \sum_k \, Q_k^2 - 2 \, Q_k t_k + t_k^2
\end{equation}
The equivalence is achieved by either removing \me{t_k^2} from \eqref{pf-1} or when the number of queued packets is large enough.

By ignoring \me{t_k^2} from \eqref{pf-1}, the \ac{Q-WSRM} scheme requires an explicit rate constraint \eqref{eqn-3.1.4} to avoid over-allocation of the resources. In the proposed queue deviation approach, explicit rate constraints are not needed, since they are handled by the objective function \eqref{eqn-4.2} itself. In contrast to the \ac{WSRM} formulation, the \ac{JSFRA} and the \ac{Q-WSRME} problems handle the sub-channels jointly to obtain an efficient allocation by identifying the optimal space-frequency resources for the users.
