In order for the constraints \eqref{eqn-6.2} and \eqref{eqn-6.3} to be active, there should be at least one user per \ac{BS} with large enough backlogged packets that cannot be satisfied with the given power budget. To prove this, let us consider a single \ac{BS} \me{N_T \times 1} system with \me{|U|_b} users each with \me{Q_k} backlogged packets. Let us consider the problem for a single \ac{BS} \me{b} as 
\begin{IEEEeqnarray}{rCl}
	\underset{\gamma_{k},\mvec{m}{k},\beta_{k}}{\text{minimize}} &\quad& \sum_{k \in \mc{U}_b} \big | Q_k - \log_2(1 + \gamma_k) \big |^q  \eqsub \\
	\text{subject to} &\quad& \gamma_{k} \leq \frac{ | \mvec{H}{k} \mvec{m}{k}|^2}{\beta_{k}} \IEEEyessubnumber \label{ae-1} \\
	&\quad& \beta_{k} \geq N_0 + \sum_{\mathclap{i \neq k}} |\mvec{H}{k} \mvec{m}{i} |^2 \IEEEyessubnumber \label{ae-2} \\
	&\quad& \sum_{k \in \mc{U}_b} \text{tr} \, (\mvec{m}{k} \mvec{m}{k}^\herm) \leq P_{{\max}}. \IEEEyessubnumber \label{ae-3}
\end{IEEEeqnarray}
Let \me{N_T > |\mc{U}_b|} and the precoders are designed for all users. Let us consider two users, say \me{i} and \me{j}, whose precoders are given by \me{\mvec{m}{i}} and \me{\mvec{m}{j}}. In order to show that the constraint \eqref{ae-1} and \eqref{ae-2} are tight for all users at optimal when the objective has non zero value at the final solution. Let us assume by contradiction that \eqref{ae-1} and \eqref{ae-2} are not tight for user \me{i} only. Let us also consider that \me{\gamma_k} for all users \me{k \in \mc{U}_b \backslash j} are identified to satisfy the backlogged packets. 

Now, in order to minimize the objective of the \me{\ith{i}} user, \me{\gamma_i} should be \me{2^{Q_i} - 1} to have the minimum objective. Since the constraints \eqref{ae-1} and \eqref{ae-2} are not tight for the \me{\ith{i}} user, the actual \ac{SINR} on the r.h.s of \eqref{ae-1} is greater than \me{\gamma_i} of the user \me{i}. It is attributed to the excess power in the transmit precoder of user \me{i} as \me{\mvec{m}{i} = \bar{\mbf{m}}_{i} + \mbf{V}_i \mbf{e}}, where \me{\mbf{V}_i = \mc{N}([\mbf{H}^{\mathrm{T}}_0, \dotsc, \mbf{H}^{\mathrm{T}}_{i-1},\mbf{H}^{\mathrm{T}}_{i+1}, \dotsc,\mbf{H}^{\mathrm{T}}_{|\mc{U}_b| - 1}]^{\mathrm{T}})} denotes the null space corresponding to the other user channel vectors and \me{\mbf{e}} is any random vector with power \me{\Delta P}. Note that this extra power has no impact on the interference constraint \eqref{ae-2} of other users. 

Since the power is not utilized efficiently, the objective will force the excess \me{\Delta P} to be used for the starving user \me{j} with the non zero objective value with the current precoder \me{\mvec{m}{j}}. In order to minimize the objective, we can find a vector \me{\mvec{m}{j} = \bar{\mbf{m}}_{j} + \mbf{V}_j \mbf{e}}, which has no impact on the interference or the optimal precoders of other users and their objective value, still minimize the objective function. It is in contradiction to our original assumption that the power is utilized for the user \me{i} making the constraints loose. It can be seen that the constraints are tight as long as there is one user with unserviced backlogged packets at each \ac{BS} with the given power budget. Since the objective is not to minimize the power, when the power budget is more than sufficient to service the backlogged packets of all users, the \ac{JSFRA} scheme is not guaranteed to find the minimum power precoders to minimize the current backlogged packets in the system.
