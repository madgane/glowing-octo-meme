\begin{comment}
For the constraints \eqref{eqn-6.2} and \eqref{eqn-6.3} to be active, there should be at least one user in each \ac{BS} with enough backlogged packets that cannot be served with the given power budget. To prove this, let us consider a \ac{BS} \me{b} with \me{N_T \times 1} system serving \me{|U|_b} users each with \me{Q_k} backlogged packets. The \ac{JSFRA} formulation can be written as
\begin{IEEEeqnarray}{rCl} \label{tight}
	\underset{\gamma_{k},\mvec{m}{k},\beta_{k}}{\text{minimize}} &\quad& \sum_{k \in \mc{U}_b} \big | Q_k - \log_2(1 + \gamma_k) \big |^q  \eqsub \\
	\text{subject to} &\quad& \gamma_{k} \leq \frac{ | \mvec{H}{k} \mvec{m}{k}|^2}{\beta_{k}} \IEEEyessubnumber \label{ae-1} \\
	&\quad& \beta_{k} \geq N_0 + \sum_{\mathclap{i \neq k}} |\mvec{H}{k} \mvec{m}{i} |^2 \IEEEyessubnumber \label{ae-2} \\
	&\quad& \sum_{k \in \mc{U}_b} \trace \, (\mvec{m}{k} \mvec{m}{k}^\herm) \leq P_{{\max}}. \IEEEyessubnumber \label{ae-3}
\end{IEEEeqnarray}
Let \me{N_T > |\mc{U}_b|} and the precoders are found by solving \eqref{tight} with a nonzero objective value at the optimal point. Let us consider two users, say \me{i} and \me{j} with precoders \me{\mvec{m}{i}} and \me{\mvec{m}{j}}. In order to show the tightness of \eqref{ae-1} and \eqref{ae-2}, let us assume by contradiction that \eqref{ae-1} and \eqref{ae-2} are not tight for user \me{i} only. Let us also consider that \me{\gamma_k} for all users \me{k \in \mc{U}_b \backslash j} are identified to satisfy the backlogged packets. 

Now, in order to minimize the objective of the \me{\ith{i}} user, \me{\gamma_i} should be \me{2^{Q_i} - 1} to have the minimum objective. Since the constraints \eqref{ae-1} and \eqref{ae-2} are not tight for the \me{\ith{i}} user, the actual \ac{SINR} on the r.h.s of \eqref{ae-1} is greater than \me{\gamma_i}. It is attributed to the excess power in the transmit precoder of user \me{i} as \me{\mvec{m}{i} = \bar{\mbf{m}}_{i} + \mbf{V}_i \mbf{e}}, where \me{\mvec{V}{i} = \mc{N}([\mbf{H}^{\mathrm{T}}_0, \dotsc, \mbf{H}^{\mathrm{T}}_{i-1},\mbf{H}^{\mathrm{T}}_{i+1}, \dotsc,\mbf{H}^{\mathrm{T}}_{|\mc{U}_b| - 1}]^{\mathrm{T}})} denotes the null space corresponding to the other user channel vectors and \me{\mbf{e}} be any random vector with power \me{\Delta P}. Note that the extra power has no impact on the interference constraint \eqref{ae-2} of other users. 

Since the allocated rate of user \me{j} is less than the queued packets, we can find a precoder \me{\mvec{m}{j}} with the additional power of \me{\Delta P} as \me{\mvec{m}{j}^\prime = \bar{\mbf{m}}_{j} + \mvec{V}{j} \mbf{e}^\prime}. Note that the new precoder has no impact on the interference constraint of other users. The newly found precoder \me{\mvec{m}{j}^\prime} minimize the queue of user \me{j} without affecting the rates of other user, there by reduces the objective further. It is in contradiction to our original assumption that the precoders are optimal and the objective is minimum. 

It can be seen that the constraints are tight as long as there is one user with unserviced backlogged packets at each \ac{BS} with the given power budget. Since the objective is not to minimize power, when the power budget is more than sufficient to service the backlogged packets of all users, the \ac{JSFRA} scheme is not guaranteed to find the minimum power precoders to empty the current backlogged packets in the system. The objective of the \ac{JSFRA} problem can be regularized by including the power term to find the minimum power precoders as
\begin{equation*}
\| \tilde{\mbf{v}} \|_q + \varphi \sum_{k \in \mc{U}} \sum_{n = 1}^{N} \sum_{l=1}^{L} \mathrm{tr} \left ( \mvec{m}{l,k,n} \mbf{m}^\herm_{l,k,n} \right ),
\end{equation*}
for a small \me{\varphi} without affecting the optimal solution. Note that the above objective is guaranteed to make the constraints \eqref{eqn-6.2} and \eqref{eqn-6.3} active by relaxing the power constraint \eqref{eqn-6.4}.
\end{comment}

For the constraints \eqref{eqn-6.2} and \eqref{eqn-6.3} to be active, there should be at least one user in each \ac{BS} with enough backlogged packets that cannot be served with the given power budget. In order to make the constraints active for all scenarios, the objective of the \ac{JSFRA} formulation should be regularized with the transmit power without affecting the solution as
\begin{equation*}
\| \tilde{\mbf{v}} \|_q + \varphi \sum_{k \in \mc{U}} \sum_{n = 1}^{N} \sum_{l=1}^{L} \mathrm{tr} \left ( \mvec{m}{l,k,n} \mbf{m}^\herm_{l,k,n} \right ),
\end{equation*}
where \me{\varphi \approx 0}. Note that the modified objective will relax the power constraint by making the constraints \eqref{eqn-6.2} and \eqref{eqn-6.3} active at the stationary point.

