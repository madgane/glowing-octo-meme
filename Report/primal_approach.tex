
In \acl{PD}, the convex problem in \eqref{eqn-decent-1} is solved for the optimal transmit precoders in an iterative manner for a fixed \ac{BS} specific interference terms \me{\zeta_{l,k,n,b}^{\{b_k\}}} using master-slave model \cite{pennanen2011decentralized}. The slave subproblem is solved in each \ac{BS} for the optimal transmit precoders for the associated users only by assuming a fixed interference \me{\zeta_{l,k,n,b}^{\{b_k\}}} from other \acp{BS} \me{b \in \mc{B} \backslash b_k} in each iteration. Upon finding the optimal associated transmit precoders by each slave subproblems, the master problem is used to update the \ac{BS} specific interference terms \me{\zeta_{l,k,n,b}^{\{b_k\}}} to a new value by using dual variables corresponding to the interference constraint \eqref{eqn-decent-3} as discussed in \cite{pennanen2011decentralized}. In this manner, the interference variables are updated until the global consensus is obtained. The primal approach is similar to the minimum power precoder design presented in \cite{pennanen2011decentralized}. Note that the master problem considers \me{\zeta_{l,k,n,b}^{\{b_k\}} \forall \{b,b_k\} \in \mc{B}} as a variable and the slave subproblems assume this to be a constant in each iteration that will be updated by the master problem.

\begin{comment}
The \acl{PD} approach decomposes the problem by fixing the interference variables \me{\zeta_{l,k,n,b} \forall k,b} in order to perform the precoder design independently across each \ac{BS}. Once the optimal precoders are designed at each \ac{BS} with the fixed interference constraints \eqref{eqn-decent-3}, the dual variables corresponding to the interference constraints are exchanged between the cooperating \acp{BS} in \me{\mc{B}} to update the interference variables \me{\zeta_{l,k,n,b}} for the next iteration until convergence. The primal approach is discussed extensively for the min-power problem in \cite{pennanen2011decentralized} and much of the current work follows similar approach. 
%Much of the details are provided in the Appendix \ref{a-2}.

\subsubsection*{Convergence}
The convergence of the primal decomposition is similar to that of the centralized problem if the interference variables \me{\zeta_{l,k,n,b}} are allowed to converge to a stationary point. In practice, we can limit the number of exchanges to \me{J_{\max}} after which the \ac{SCA} update is performed until convergence or for \me{I_{\max}} times. The update of \me{\tilde{p}_{l,k,n}, \tilde{q}_{l,k,n}} and \me{\tilde{\beta}_{l,k,n}} can be made in conjunction with the receiver update \me{\mvec{W}{k,n}}. The receiver update can be made by using the precoded pilot transmission from each user as in \cite{komulainen2013effective}. 
\end{comment}