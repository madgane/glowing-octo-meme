
The \acl{PD} approach decomposes the problem by fixing the interference variables \me{\zeta_{l,k,n,b} \forall k,b} in order to perform the precoder design independently across each \ac{BS}. Once the optimal precoders are designed at each \ac{BS} with the fixed interference constraints \eqref{eqn-decent-3}, the dual variables corresponding to the interference constraints are exchanged between the cooperating \acp{BS} in \me{\mc{B}} to update the interference variables \me{\zeta_{l,k,n,b}} for the next iteration until convergence. The primal approach is discussed extensively for the min-power problem in \cite{pennanen2011decentralized} and much of the current work follows similar approach. 
%Much of the details are provided in the Appendix \ref{a-2}.

\subsubsection*{Convergence}
The convergence of the primal decomposition is similar to that of the centralized problem if the interference variables \me{\zeta_{l,k,n,b}} are allowed to converge to a stationary point. In practice, we can limit the number of exchanges to \me{J_{\max}} after which the \ac{SCA} update is performed until convergence or for \me{I_{\max}} times. The update of \me{\tilde{p}_{l,k,n}, \tilde{q}_{l,k,n}} and \me{\tilde{\beta}_{l,k,n}} can be made in conjunction with the receiver update \me{\mvec{W}{k,n}}. The receiver update can be made by using the precoded pilot transmission from each user as in \cite{komulainen2013effective}. 