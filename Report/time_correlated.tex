The performance of the \ac{JSFRA} scheme for different \me{\ell_q} over multiple transmission slots are discussed here. It is compared to the existing \ac{Q-WSRME} scheme by varying the average arrival rate \me{A_k} of all users. Fig. \ref{fig-time-analysis} plots the performances of the centralized algorithms for different \me{\ell_q} values. Even though \me{A_k}'s are constant for all users, the instantaneous arrivals are random and it follows the Poisson process. The \ac{PL} is modeled as a uniform random variable \me{[0,-6]} dB.

Fig. \subref*{fig-review} compares the average number of backlogged packets present in the system after each transmission slot for various schemes discussed. The performance of the \me{\ell_2} \ac{JSFRA} scheme is similar to the \ac{Q-WSRME} approach in terms of the average number of residual packets. The explicit rate constraints in the \ac{Q-WSRME} scheme is the reason for the equivalence in the stable region, \textit{i.e}, when the average number of backlogged packets is bounded. Both \ac{Q-WSRM} and \ac{Q-WSRME} performs similar to \me{\ell_2} \ac{JSFRA} scheme when the system is unstable. It also includes the \me{\ell_1} \ac{JSFRA} scheme in both uncoordinated and in \ac{TDM} mode with the total power fixed to \me{P_{\max}} in each transmission. The uncoordinated \ac{JSFRA} performs better than \ac{TDM} mode of operation due to the high \ac{SINR} variations among the users. Due to the greedy allocation of the \me{\ell_1} \ac{JSFRA} scheme, the total number of backlogged packets is significantly less compared to the other schemes in Fig. \ref{fig-time-analysis}. Also the \me{\ell_{\infty}} \ac{JSFRA} scheme performs equally well compared to the other objectives when the system is stable, whereas it is inferior in the unstable region, since the fairness is not effective when the number of backlogged packets is large. Fig. \subref*{fig-review-time} compares the number of backlogged packets at each instant for the schemes discussed.

%Fig. \subref*{fig-review} also includes the performance of the \ac{JSFRA} scheme using \me{\ell_1} norm in the uncoordinated precoder design and in \ac{TDM} mode. The total power constraint is fixed at \me{P_{\max}} on each transmission in the \ac{TDM} mode.
