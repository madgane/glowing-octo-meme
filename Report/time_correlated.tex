
We discuss the performance of the \ac{JSFRA} algorithm for different \me{\ell_q} over multiple transmission slots. It is compared to the existing \ac{Q-WSRME} scheme by varying the average arrival rate \me{A_k} of all users. Fig. \ref{fig-time-analysis} plots the performances of the centralized algorithms for different \me{\ell_q} values. Even though \me{A_k}'s are constant for all users, the instantaneous arrivals are random and they follow the Poisson process. The \ac{PL} is modeled as a uniform random variable \me{[0,-6]} dB.

Fig. \subref*{fig-review} compares various schemes with the average number of backlogged packets present in the system after each transmission slot. The performance of the \ac{JSFRA} scheme using \me{\ell_2} is similar to the \ac{Q-WSRME} approach in the average number of residual packets after each transmission slot. The additional rate constraints in the \ac{Q-WSRME} scheme is the reason for the equivalence in the stable region, \textit{i.e}, when the average number of backlogged packets is bounded. Both \ac{Q-WSRM} and \ac{Q-WSRME} performs similar to \me{\ell_2} \ac{JSFRA} scheme when the arrival rates are larger than the actual transmissions. Fig. \ref{fig-time-analysis} shows that the number of backlogged packets are noticeably less for the \me{\ell_1} \ac{JSFRA} scheme due to the greedy allocation of serving users with better channel. It also highlights that the \me{\ell_{\infty}} \ac{JSFRA} scheme performs equally well compared to the other objectives when the system is stable. On contrary, the performance is inferior in the unstable region, since the fairness is not effective when the number of backlogged packets is large for all users. 


%Fig. \subref*{fig-review} also includes the performance of the \ac{JSFRA} scheme using \me{\ell_1} norm in the uncoordinated precoder design and in \ac{TDM} mode. The total power constraint is fixed at \me{P_{\max}} on each transmission in the \ac{TDM} mode.
