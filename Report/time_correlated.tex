
We discuss performance of the \ac{JSFRA} algorithm for different values of \me{\ell_q} over multiple transmission slots. It is compared with the existing \ac{Q-WSRME} scheme by varying the average arrival rate \me{A_k} of all users. Fig. \ref{fig-time-analysis} demonstrates the performance of the centralized algorithms for different \me{\ell_q} values. Even though \me{A_k}'s are constant for all users, the instantaneous arrivals are random and is based on Poisson arrival process. We considered a \me{4 \times 1} \ac{MIMO} system with \me{N = 4} sub-channels and \me{N_B = 2} \acp{BS}. The path loss is modeled using a uniform random variable \me{[0,-3]} dB with the maximum \ac{SINR} seen by any user is \me{6} dB.

Fig. \ref{fig-review} compares various schemes with the average number of backlogged packets present in the system after each transmission slot. The performance of the \ac{JSFRA} scheme using \me{\ell_2} and \ac{Q-WSRME} approach are similar in the average number of residual packets after each transmission slot. Note that the additional rate constraints in the \ac{Q-WSRME} scheme is the reason for the equivalence. Both \ac{Q-WSRM} and \ac{Q-WSRME} performs similar to \me{\ell_2} \ac{JSFRA} scheme when the arrival rates are significantly greater than the actual transmissions. It can be seen from Fig. \ref{fig-review} and Fig. \ref{fig-review-time} that the number of backlogged packets are noticeably less for the \me{\ell_1} \ac{JSFRA} formulation due to the greedy allocation by serving users with better channel conditions. Fig. \ref{fig-time-analysis} shows that the \me{\ell_{\infty}} \ac{JSFRA} scheme performs worst in terms of the average number of backlogged packets due to the instantaneous fairness constraints.
