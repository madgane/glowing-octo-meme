
The distributed solutions via primal and \ac{ADMM} approaches depend on the subgradient update by using a step size parameter for the coupling variables, which affects the speed of convergence to the optimal value.
In this method, we provide an alternative approach to decentralize the \ac{MSE} equivalent problem considered in \cite{christensen2008weighted,wmmse_shi} by solving the \ac{KKT} conditions. Simimar work has been considered for the \ac{WSRM} problem with the rate constraints in \cite{kaleva2012weighted}. When the queues are involved, the maximum rate constraint imposed by the number of queued packets at the \ac{BS} poses a nonconvex constraint, which makes the problem difficult to solve in an iterative approach as in \cite{kaleva2012weighted}.

Even though the rate constraints are implicitly present in the objective function, we cannot formulate the \ac{KKT} conditions readily due to the non-differentiable objective function. The non-differentiability of the objective function is due to the absolute operator present in the norm function. In order to make the objective function differentiable, we consider the following case for which the absolute operator can be ignored without affecting the optimal solution, namely,
\begin{itemize}
\item when the exponent \me{q} is even or,
\item when the number of backlogged packets of each user is large enough \me{Q_k \gg \sum_{n=1}^N \sum_{l=1}^L {t}_{l,k,n}} to ignore the absolute operator.
\end{itemize}

With the assumption of either one of the above conditions to be true, the problem in \eqref{eqn-mse-2} can be written as
\begin{IEEEeqnarray}{rCl}\label{kkt-mse-1}
\underset{\substack{t_{l,k,n},\mvec{M}{k,n}, \alpha_{l,k,n}, \\ \delta_{b},\sigma_{l,k,n},\mvec{W}{k,n}}}{\text{minimize}} &\hspace{0.5cm}& \sum_{b \in \mc{B}} \sum_{k \in \mc{U}_b} \, a_k \, \Big ( Q_k - \sum_{n = 1}^N \sum_{l = 1}^{L} t_{l,k,n} \Big )^q \IEEEyessubnumber \label{kkt-mse-1.1} \\
\text{subject to} & \hspace{0.5cm} & \nonumber \\
\alpha_{l,k,n} : & \hspace{0.5cm} & \epsilon_{l,k,n} = \left | 1 - \mvec{w}{l,k,n}^\herm \mvec{H}{b_k,k,n} \mvec{m}{l,k,n} \right |^2 + \sum_{(x,y) \neq (l,k)} \left | \mvec{w}{l,k,n}^\herm \mvec{H}{b_y,y,n} \mvec{m}{x,y,n} \right |^2 \\ \nonumber
&\hspace{0.5cm}& \quad {} + N_0 \, \|\mvec{w}{l,k,n}\|^2 \IEEEyessubnumber \label{kkt-mse-1.2} \\
\sigma_{l,k,n} : & \hspace{0.5cm} & - \log(\tilde{\epsilon}_{l,k,n}) - \frac{\left ( {\epsilon}_{l,k,n} - \tilde{\epsilon}_{l,k,n} \right ) }{\tilde{\epsilon}_{l,k,n}} \geq t_{l,k,n} \, \log(2) \IEEEyessubnumber \label{kkt-mse-1.3} \\
\delta_b : & \hspace{0.5cm} & \sum_{n = 1}^N \sum_{k \in \mathcal{U}_b} \text{tr} \, (\mvec{M}{k,n} \mvec{M}{k,n}^\herm) \leq P_{{\max}}, \fall b, \IEEEyessubnumber \label{kkt-mse-1.4}
\end{IEEEeqnarray}
where \me{\alpha_{l,k,n},\sigma_{l,k,n}} and \me{\delta_b} are the dual variables corresponding to the constraints defined in \eqref{kkt-mse-1.2}, \eqref{kkt-mse-1.3} and \eqref{kkt-mse-1.4}. The equality of \eqref{kkt-mse-1.2} is due to the equivalence of the \ac{MSE} expression with the transmit and the receive precoders.

By forming the Lagrangian of \eqref{kkt-mse-1} with the corresponding dual variables as shown in \eqref{kkt-mse-1}, we can obtain the \ac{KKT} expressions by differentiating with respect to the variables present in the problem as detailed in Appendix \ref{a-1}. Upon solving the \ac{KKT} expressions, the iterative solution \me{\forall k \in \mc{U}}, \me{\forall n \in \set{1,\ldots,N}} and \me{\forall l \in \set{1,\ldots,L}} is given by
\begin{IEEEeqnarray}{rCl} \label{kkt-mse-4}
\mvec{m}{l,k,n}^{(i)} &=& \Big ( \sum_{x \in \mc{U}} \sum_{y=1}^L \alpha_{y,x,n}^{(i-1)} \mvec{H}{b_k,x,n}^\herm \mvec{w}{y,x,n}^{(i-1)} \mvec{w}{y,x,n}^{\herm \, {(i-1)}} \mvec{H}{b_k,x,n} + \delta_b \mbf{I}_{N_T} \Big )^{-1} \alpha^{(i-1)}_{l,k,n} \mvec{H}{b_k,k,n}^\herm \mvec{w}{l,k,n}^{(i-1)} \IEEEyessubnumber \label{kkt-mse-4.3} \\
\mvec{w}{l,k,n}^{(i)} &=& \Big ( \sum_{x\in\mc{U}}\sum_{y=1}^L \mvec{H}{b_x,k,n} \mvec{m}{y,x,n}^{(i)} \mvec{m}{y,x,n}^{\herm \, (i)} \mvec{H}{b_{x},k,n}^\herm + \mathbf{I}_{N_R} \Big ) ^{-1} \; \mvec{H}{b_k,k,n} \; \mvec{m}{l,k,n}^{(i)}, \IEEEyessubnumber \label{kkt-mse-4.6} \\
\epsilon_{l,k,n}^{(i)} &=& \left | 1 - \mvec{w}{l,k,n}^{\herm \, (i)} \mvec{H}{b_k,k,n} \mvec{m}{l,k,n}^{(i)} \right |^2 + \sum_{(x,y) \neq (l,k)} \left | \mvec{w}{l,k,n}^{\mathrm{H} \, (i)} \mvec{H}{b_y,y,n} \mvec{m}{x,y,n}^{(i)} \right |^2 + N_0 \, \|\mvec{w}{l,k,n}^{(i)}\|^2, \IEEEyessubnumber \label{kkt-mse-4.4} \\
t_{l,k,n}^{(i)} &=&  -\log_2(\epsilon_{l,k,n}^{(i-1)}) - \frac{\left ( \epsilon_{l,k,n}^{(i)} - \epsilon_{l,k,n}^{(i-1)} \right ) }{\log(2) \, \epsilon_{l,k,n}^{(i-1)}}, \IEEEyessubnumber \label{kkt-mse-4.5} \\
\sigma_{l,k,n}^{(i)} &=& q \,a_k \, \log_2(e)  \, \Big (Q_k - \sum_{n = 1}^N \sum_{l=1}^L t_{l,k,n}^{(i)} \Big )^{(q-1)}, \IEEEyessubnumber \label{kkt-mse-4.2} \\
\alpha^{(i)}_{l,k,n} &=& \alpha^{(i-1)}_{l,k,n} + \rho \Big ( \rfrac{\sigma_{l,k,n}^{(i)}}{\epsilon_{l,k,n}^{(i)}} - \alpha^{(i-1)}_{l,k,n} \Big ). \IEEEyessubnumber \label{kkt-mse-4.1}
\end{IEEEeqnarray}
The dual variable \me{\alpha^{(i)}} is updated with memory as in \eqref{kkt-mse-4.1} to avoid abrupt oscillations due to \me{\sigma < 0} from \eqref{kkt-mse-4.2} when \me{Q_k < \sum_{n=1}^N \sum_{l=1}^L {t}_{l,k,n}}. The parameter \me{\rho \in [0,1]} provides a linear combination of the previous to the current value of the dual variable, dictating system reaction for the excess allocation.

The \ac{KKT} solutions provided in \eqref{kkt-mse-4} are solved in an iterative manner by initializing the transmit and the receive precoders \me{\mvec{M}{k,n},\mvec{W}{k,n}} with the single user beamforming and the \ac{MMSE} vectors. The dual variable \me{\alpha}'s corresponding to precoder weights are initialized with ones to provide equal priorities to all streams. Now, the closed form expressions in \eqref{kkt-mse-4} are evaluated sequentially until convergence or to a certain accuracy. In \eqref{kkt-mse-4}, all expressions are in closed form except the transmit precoders \eqref{kkt-mse-4.3}, which depends on the \ac{BS} specific dual variable \me{\delta_b}. It can be solved efficiently by the bisection method satisfying the power constraint \eqref{kkt-mse-1.4}. After each iteration instant, the transmit and the receive precoders are exchanged across the coordinating \acp{BS} in \me{\mc{B}} to obtain the next operating point.

The receive beamformers from the users can be informed to the coordinating \acp{BS} by using the precoded uplink pilot signaling, where the precoders used for the uplink pilots are the receive beamformers \me{\mvec{W}{k,n}} evaluated at the receivers. Upon receiving the uplink precoded pilots by the \ac{BS}, the effective channel \me{\mvec{W}{k,n}^{\herm \, (i-1)} \mvec{H}{b,k,n}} can be used in the expression \eqref{kkt-mse-4.3} to update the transmit precoders at the respective \acp{BS} \cite{komulainen2013effective}. The algorithmic representation of the \ac{KKT} based scheme is shown in Algorithm. \ref{algo-4}.
\begin{algorithm}
 \SetAlgoLined
 \DontPrintSemicolon
 \BlankLine
 \SetKwInput{KwInit}{Initialize}
 \KwIn{\me{a_k, \, Q_k, \, \mvec{H}{b,k,n},\; \fall b \in \mathcal{B}, \, \fall k \in \mathcal{U}, \fall n \in \mathcal{C}}}
 \KwOut{\me{\mvec{m}{l,k,n}} and \me{\mvec{w}{l,k,n} \fall l \in \set{1,2,\dotsc,L}}}
 \KwInit{\me{i=1} and the receive beamformers \me{\mbf{w}^{(0)}_{l,k,n}} randomly}
 \KwInit{\me{\epsilon_{l,k,n}^{(0)}} randomly and the dual variables \me{{\alpha}_{l,k,n}^{(0)} = 1}}
 set the maximum iteration counter \me{I_{\max}} to a valid number \;
 for each \ac{BS} \me{b \in \mc{B}}, perform the following procedure \;
 \Repeat{convergence or \me{i \geq I_{\max}}}{
	update the transmit precoders \me{\mvec{M}{k,n}^{(i)}} with \me{\delta_b} using \eqref{kkt-mse-4.3} by the bisection method satisfying \eqref{kkt-mse-1.4} \;
	evaluate the receive beamforming vector \me{\mvec{W}{k,n}^{(i)}} by \eqref{kkt-mse-4.6} \;
	update the \ac{MSE} variable \me{\epsilon_{l,k,n}^{(i)}} and the throughput variable \me{t_{l,k,n}^{(i)}} using \eqref{kkt-mse-4.4} and \eqref{kkt-mse-4.5} \;
	solve for the dual variables \me{\alpha_{l,k,n}^{(i)}} and \me{\sigma_{l,k,n}^{(i)}} using \eqref{kkt-mse-4.1} and \eqref{kkt-mse-4.2} \;
	exchange the transmit and the receive precoders \me{\mvec{M}{k,n}^{(i)}, \mvec{W}{k,n}^{(i)}} across the coordinating \acp{BS} in \me{\mc{B}} \;
	$i = i + 1$ \;
  }
 \caption{\ac{KKT} approach for the \ac{JSFRA} scheme}
  \label{algo-4}
\end{algorithm}

