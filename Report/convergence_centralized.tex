To prove the convergence of the transmit, receive beamformers, and the objective values of the centralized algorithms in \eqref{eqn-6} and \eqref{eqn-mse-1}, we need to show that the following conditions are satisfied by the centralized formulations.
\begin{itemize}
	\item[(a)] The function should be coercive and bounded below
	\item[(b)] The feasible set should be a compact set
	\item[(c)] The sequence of the objective values should be strictly decreasing in each iteration
	\item[(d)] Uniqueness of the minimizer, \textit{i.e}, the transmit and the receive beamformers should be unique in each iteration.
\end{itemize}
Using \cite[Prop. A.8]{bertsekas1999nonlinear}, the existence of a global minimizer in the feasible set can be guaranteed if the conditions (a) and (b) are satisfied. Since the feasible set is not fixed in each iteration, we require the conditions (c) and (d) to prove the global convergence of the objective function and the corresponding arguments namely, the transmit and the receive precoders. Assuming the conditions (a), (b) and (c) are satisfied, using \cite[Theorem 3.14]{rudin1964principles}, we can show that the bounded monotonically decreasing objective value sequence has a unique minimum. Finally, we use the discussions in \cite{marks1978technical,lanckriet2009convergence,scutari_1} to show that the limiting point of the iterative algorithm is indeed a stationary point of the nonconvex problem in \eqref{eqn-6} and \eqref{eqn-mse-1}.

\subsection{Bounded Objective Function and Compact Feasible Set}
The feasible set of the problem \eqref{eqn-6} and \eqref{eqn-mse-1} are bounded and closed, which can be verified by the total power constraint on the transmit precoders \eqref{eqn-6.4}, and therefore, a compact set. 

The minimum value of the norm operator in the objective is zero, \textit{i.e}, \me{\|x\|_q > -\infty}, therefore it is bounded below. The objective function is Lipschitz continuous over the feasible set, and therefore, it is bounded from above as well, since the feasible set is bounded. The objective function in \eqref{eqn-6} and \eqref{eqn-mse-1} is continuous and approaches \me{\infty} as \me{\mbf{t}_k \rightarrow \infty}, therefore it is coercive. Using conditions (a) and (b), we can guarantee the existence of a minimizer to the problem in \eqref{eqn-6} and \eqref{eqn-mse-1}.

\subsection{Monotonicity} \label{mcity}
Let us express the centralized problem in \eqref{eqn-6} and \eqref{eqn-mse-1} as
\begin{subeqnarray} \label{con}
	\underset{\mx,\my,\mz}{\text{minimize}} &\quad& f(\mx,\my,\mz) \eqsub \slabel{con-obj} \\
	\text{subject to} &\quad& h(\mz) - g_0(\mx,\my) \leq 0 \eqsub \slabel{con-dc} \\
	&\quad& g_1(\mx,\my) \leq 0 \eqsub \slabel{con-cvx-blk} \\
	&\quad& g_2(\mx) \leq 0 \eqsub \slabel{con-cvx}
\end{subeqnarray}
where \me{g_2,f} are convex and \me{h} is a linear function. Let \me{g_0,g_1} be convex in either \me{\mx} or \me{\my} but not on both. The constraint in \eqref{con-dc} corresponds to \eqref{eqn-6.2} or \eqref{eqn-mse-1.2} and the constraint \eqref{con-cvx-blk} correspond to \eqref{eqn-6.3} or \eqref{eqn-mse-1.3}. Other convex constraints are addressed by \eqref{con-cvx} and the feasible set of \eqref{con} is given by 
\iftoggle{single_column}{
\begin{equation}
\mc{F} = \{ \; \mx,\my,\mz \; \big | h(\mz) - g_0(\mx,\my) \leq 0, g_1(\mx,\my) \leq 0, g_2(\mx) \leq 0 \}.
\end{equation}
}{
\begin{align}
\mc{F} = \{ \; \mx,\my,\mz \; \big | & \quad h(\mz) - g_0(\mx,\my) \leq 0, \nonumber \\
								 & \quad g_1(\mx,\my) \leq 0, g_2(\mx) \leq 0 \}.
\end{align}}

To solve \eqref{con}, we adopt \ac{AO} by fixing a block of varibles and optimize for others \cite{bezdek2002some}. In \eqref{con}, even after fixing the variable \me{\my}, the problem is nonconvex due to the \ac{DC} constraint \eqref{con-dc}. We adopt \ac{SCA} presented in \cite{lipp2014variations,lanckriet2009convergence,scutari_1} by relaxing the nonconvex set by a sequence of convex subsets. Since it involves two nested iterations, we denote the \ac{AO} iteration index by a superscript \me{(i)} and the \ac{DC} constraint relaxations by a subscript \me{k}. Let \me{\iter{\mc{X}}{k}{i}} be the feasible set for the \eqn{\ith{i}} \ac{AO} iteration and the \me{\ith{k}} \ac{SCA} point for a fixed \me{\my} and \me{\iter{\mc{Y}}{k}{i}} denotes the feasible set for a fixed \me{\mx}. Since the \ac{SCA} iterations are performed until convergence, let \me{\iter{\mx}{\ast}{i}} denote the converged point of \me{\mx} in the \me{\ith{i}} \ac{AO} iteration. Let \me{\iter{\mz}{\ast|\my}{i}} be the optimal value of \me{\mz} obtained in the \me{\ith{i}} \ac{AO} iterate for a fixed \me{\my}.

To begin with, let us consider the variable \me{\my} is fixed for the \ac{AO} \me{i} with the optimal value achieved from the previous iteration \me{i-1} as \me{\iter{\my}{\ast}{i-1}}. In order to solve for \me{\mx} in the \ac{SCA} iteration \me{k}, we linearize the nonconvex function \me{g_0} using previous \ac{SCA} iterate of \me{\mx} as
\iftoggle{single_column}{
\begin{equation}\label{con-relax}
\hat{g}_o(\mx,\iter{\my}{\ast}{i-1};\iter{\mx}{k}{i}) = {g}_0(\iter{\mx}{k}{i},\iter{\my}{\ast}{i-1}) + \nabla g_0(\iter{\mx}{k}{i},\iter{\my}{\ast}{i-1})^{\mathrm{T}} (\mx - \iter{\mx}{k}{i}).
\end{equation}}{
\begin{multline} \label{con-relax}
\hat{g}_o(\mx,\iter{\my}{\ast}{i-1};\iter{\mx}{k}{i}) = {g}_0(\iter{\mx}{k}{i},\iter{\my}{\ast}{i-1}) \\ + \nabla g_0(\iter{\mx}{k}{i},\iter{\my}{\ast}{i-1})^{\mathrm{T}} (\mx - \iter{\mx}{k}{i}).
\end{multline}}
Using \eqref{con-relax}, the convex subproblem for \me{\ith{i}} \ac{AO} iteration and \me{\ith{k}} \ac{SCA} point for the variable \me{\mx} and \me{\mz} is given by
\begin{subeqnarray} \label{con-m}
	\underset{\mx,\mz}{\text{minimize}} &\quad& f(\mx,\iter{\my}{\ast}{i-1},\mz) \eqsub \slabel{con-obj-m} \\
	\text{subject to} &\quad& h(\mz) - \hat{g}_0(\mx,\iter{\my}{\ast}{i-1};\iter{\mx}{k}{i}) \leq 0 \eqsub \slabel{con-dc-m} \\
	&\quad& g_1(\mx,\iter{\my}{\ast}{i-1}) \leq 0 \eqsub \slabel{con-cvx-blk-m} \\
	&\quad& g_2(\mx) \leq 0. \eqsub \slabel{con-cvx-m}
\end{subeqnarray}
Let the feasible set defined by the problem in \eqref{con-m} be represented as \me{\iter{\mc{X}}{k}{i} \subset \mc{F}}. In order to prove the convergence of the convex subproblem \eqref{con-m} for a fixed \me{\my = \iter{\my}{\ast}{i-1}} operating at \me{\iter{\mx}{k}{i}}, let us consider that \eqref{con-m} yields \me{\iter{\mx}{k+1}{i}} and \me{\iter{\mz}{k+1}{i}} as the solution for the \me{\ith{k}} iteration. Note that the point \me{\iter{\mx}{k+1}{i}} and \me{\iter{\mz}{k+1}{i}}, which minimizes the objective function, is also feasible for \eqref{con-m} using the following inequality
\iftoggle{single_column}{
\begin{equation}\label{con-sub-set}\allowdisplaybreaks
h(\iter{\mz}{k+1}{i}) - {g}_0(\iter{\mx}{k+1}{i},\iter{\my}{\ast}{i-1}) \leq - \hat{g}_0(\iter{\mx}{k+1}{i},\iter{\my}{\ast}{i-1};\iter{\mx}{k}{i}) + h(\iter{\mz}{k+1}{i}) \leq h(\iter{\mz}{k}{i}) - \hat{g}_0(\iter{\mx}{k}{i},\iter{\my}{\ast}{i-1};\iter{\mx}{k}{i}) \leq 0. 
\end{equation}}{
\begin{multline}\label{con-sub-set}\allowdisplaybreaks
h(\iter{\mz}{k+1}{i}) - {g}_0(\iter{\mx}{k+1}{i},\iter{\my}{\ast}{i-1}) \leq - \hat{g}_0(\iter{\mx}{k+1}{i},\iter{\my}{\ast}{i-1};\iter{\mx}{k}{i}) \\ + h(\iter{\mz}{k+1}{i}) \leq h(\iter{\mz}{k}{i}) - \hat{g}_0(\iter{\mx}{k}{i},\iter{\my}{\ast}{i-1};\iter{\mx}{k}{i}) \leq 0. 
\end{multline}}
Using \eqref{con-sub-set}, we can show that the solution \me{\iter{\mx}{k+1}{i}} and \me{\iter{\mz}{k+1}{i}} are feasible, since the initial point of \me{\mx = \iter{\mx}{\ast}{i-1}} was chosen to be feasible from the earlier \ac{AO} iteration \me{i-1}. At each \ac{SCA} update, the feasible set includes the limiting point from the previous iteration as \me{\{\iter{\mx}{k+1}{i},\iter{\my}{\ast}{i-1},\iter{\mz}{k+1}{i}\} \in \iter{\mc{X}}{k+1}{i} \subset \mc{F}}, therefore, it ensures monotonic decrease in the objective values \cite{lanckriet2009convergence,scutari_1,quoc2011sequential} as
\iftoggle{single_column}{
\begin{equation} \label{con-convergence}
f(\iter{\mx}{0}{i},\iter{\my}{\ast}{i-1},\iter{\mz}{0}{i}) \geq f(\iter{\mx}{k}{i},\iter{\my}{\ast}{i-1},\iter{\mz}{k}{i}) \geq f(\iter{\mx}{k+1}{i},\iter{\my}{\ast}{i-1},\iter{\mz}{k+1}{i}) \geq f(\iter{\mx}{\ast}{i},\iter{\my}{\ast}{i-1},\iter{\mz}{\ast|\my}{i}). 
\end{equation}}{
\begin{multline} \label{con-convergence}
f(\iter{\mx}{0}{i},\iter{\my}{\ast}{i-1},\iter{\mz}{0}{i}) \geq f(\iter{\mx}{k}{i},\iter{\my}{\ast}{i-1},\iter{\mz}{k}{i}) \\ \geq f(\iter{\mx}{k+1}{i},\iter{\my}{\ast}{i-1},\iter{\mz}{k+1}{i}) \geq f(\iter{\mx}{\ast}{i},\iter{\my}{\ast}{i-1},\iter{\mz}{\ast|\my}{i}). 
\end{multline}}
Thus the sequence \me{f(\iter{\mx}{k}{i},\iter{\my}{\ast}{i-1},\iter{\mz}{k}{i})} is nonincreasing and approaches a limiting point as \me{k \rightarrow \infty}. The feasible point \me{\{\iter{\mx}{\ast}{i}, \iter{\my}{\ast}{i-1},\iter{\mz}{\ast|\my}{i}\}} need not be a stationary point of \eqref{con}, since it is a minimizer in the set \me{\iter{\mc{X}}{\ast}{i} \subset \mc{F}}, for a fixed \me{\my}.

Once the solution is found for a fixed \me{\my}, we fix \me{\mx} as \me{\iter{\mx}{\ast}{i}} and optimize for \me{\my}. Even after treating \me{\mx} as a constant, the problem is still nonconvex due to the \ac{DC} constraint. Following similar approach, we can find the minimizer \me{\iter{\my}{k}{i}} and \me{\iter{\mz}{k}{i}} for a similar convex subproblem \eqref{con-m} at each iteration \me{{k}}. Note that \me{\iter{\mz}{k}{i}} is reused since the variable \me{\mx} is fixed for the \me{\ith{i}} \ac{AO} iteration. The convergence and the nonincreasing behavior of the problem follows similar arguments as above\footnote{Note that we can also use the \ac{MMSE} receiver in \eqref{eqn-10} instead of performing the \ac{SCA} updates until convergence for the optimal receiver.}. Now, the solution of the converged subproblems with \me{\my} as variable are \me{\iter{\my}{\ast}{i}} and \me{\iter{\mz}{\ast}{i}}. Note that the limiting point \me{\{\iter{\mx}{\ast}{i}, \iter{\my}{\ast}{i},\iter{\mz}{\ast|\mx}{i}\}} is a minimizer in the set \me{\iter{\mc{Y}}{\ast}{i}}.

Finally, to prove the global convergence of the iterative algorithm, we need to show the nonincreasing behavior of the objective function in between the \ac{AO} updates, \textit{i.e}, 
\iftoggle{single_column}{
\begin{equation}\allowdisplaybreaks
f(\iter{\mx}{\ast}{i},\iter{\my}{\ast}{i},\iter{\mz}{\ast|\mx}{i}) \leq f(\iter{\mx}{\ast}{i},\iter{\my}{0}{i},\iter{\mz}{0}{i}) \leq f(\iter{\mx}{\ast}{i},\iter{\my}{\ast}{i-1},\iter{\mz}{\ast|\my}{i}).
\end{equation}}{
\begin{multline}\allowdisplaybreaks
f(\iter{\mx}{\ast}{i},\iter{\my}{\ast}{i},\iter{\mz}{\ast|\mx}{i}) \leq f(\iter{\mx}{\ast}{i},\iter{\my}{0}{i},\iter{\mz}{0}{i}) \\
\leq f(\iter{\mx}{\ast}{i},\iter{\my}{\ast}{i-1},\iter{\mz}{\ast|\my}{i}).
\end{multline}}
Let us consider an \ac{AO} iteration \me{i} in which the optimal value for \eqn{\mx} and \eqn{\mz} are obtained as \me{\iter{\mx}{\ast}{i}} and \me{\iter{\mz}{\ast|\my}{i}} using fixed \eqn{\my = \iter{\my}{\ast}{i-1}}. To find \me{\iter{\my}{0}{i}}, we fix \eqn{\mx} as \me{\iter{\mx}{\ast}{i}} and optimize for \me{\my}. Since the convex function is linearized in \eqref{con-dc}, the fixed operating point is also included in the feasible set \eqn{\{\iter{\mx}{\ast}{i},\iter{\my}{\ast}{i-1},\iter{\mz}{\ast|\my}{i}\} \in \iter{\mc{Y}}{0}{i}} by following \eqref{con-sub-set}. Using this, we can show the monotonicity of the objective value sequence as
\begin{equation}
f(\iter{\mx}{\ast}{i},\iter{\my}{0}{i},\iter{\mz}{0}{i}) \leq f(\iter{\mx}{\ast}{i},\iter{\my}{\ast}{i-1},\iter{\mz}{\ast|\mx}{i}).
\end{equation}
The \ac{AO} update follows \eqn{\{\iter{\mx}{\ast}{i},\iter{\my}{\ast}{i-1},\iter{\mz}{\ast|\my}{i}\} \in \{ \iter{\mc{X}}{\ast}{i} \cap \iter{\mc{Y}}{0}{i} \}}. %Using the conditions (a),(b) and (c), we can show that the objective values of the iterative algorithm converges to a limiting point or a local optimal point of the original nonconvex problem \eqref{con}.

\subsection{Convergence of the Beamformer iterates}
Let \me{\mbf{\ma} \triangleq [\mx,\my,\mz]} be the stacked vector of the optimization variables and \me{\mc{A}} be a point-to-set mapping algorithm from \me{\mc{F}} into the nonempty subsets of \me{\mc{F}}, \textit{i.e}, \eqn{\iterate{\ma}{i+1} \in \mc{A}(\iterate{\ma}{i})}. The set of accumulation points or the limiting points in the compact set \me{\mc{F}} be denoted by \me{\mc{F}^\ast} and the objective function be \me{f} is closed and continuous. Let \me{\{\iterate{\ma}{i}\}} be the sequence of iterates generated by the algorithm \me{\mc{A}} and the objective function \me{f}. If the mapping is strictly monotonic and uniformly compact, then the following conditions hold \cite{zangwill1969nonlinear} and \cite[Theorem 3.1]{meyer1976sufficient}.
\begin{itemize}
\item[(i)] all accumulation points will be a fixed point,
\item[(ii)] \eqn{f(\iterate{\ma}{i}) \rightarrow f(\ma^\ast)}, where \eqn{\ma^\ast} is a fixed point,
\item[(iii)] \eqn{\ma^\ast} is a regular point, \textit{i.e} \eqn{\|\iterate{\ma}{i} - \iterate{\ma}{i+1} \| \rightarrow 0}.
\end{itemize}

The mapping \eqn{\mc{A}} for the iterative algorithm is given by
\begin{equation}
\iterate{\ma}{i+1} = \mc{A}(\iterate{\ma}{i}) : \underset{\ma}{\mathrm{argmin}} \; f(\ma;\iterate{\ma}{i}) \; \forall \ma \in \mc{F}
\end{equation}
where \me{\iterate{\ma}{i}} belongs to the set of equally valid iterates \me{\mc{A}(\iterate{\ma}{i})} in the \me{\ith{i}} iteration\footnote{Index \me{i} denotes the \ac{AO} iteration to update the vector \me{\ma}, which involves two \ac{SCA} iterations to be performed until convergence, \textit{i.e}, one for the variable \me{\mx} by keeping \me{\my} fixed and another for the variable \me{\my} by fixing \me{\mx} as constant.}. The mapping is strictly monotonic on \me{\mc{F}} with respect to the objective function as \me{\ma^\prime \in \mc{A}(\ma)} implies \me{f(\ma^\prime) < f(\ma)} whenever \me{\ma} is not a fixed point, \textit{i.e}, \me{\ma \notin \mc{F}^\ast}, as discussed in Appendix \ref{mcity}. The sequence of iterates \me{\{\iterate{\ma}{i}\}} will have at least one accumulation point, which follows from the compactness of the set \cite{zangwill1969nonlinear}. The sequence \me{\{\iterate{\ma}{i}\}} converges to the set of fixed points \me{\mc{F}^\ast} for any feasible starting point \me{\iterate{\ma}{0} \in \mc{F}}, therefore the mapping is uniformly compact.

To show the set of fixed points \me{\mc{F}^\ast} are the minimizers for the objective function \me{f}, we rely on the strict monotonicity of the objective function for each iteration. Since \me{f(\mc{A}(\iterate{\ma}{i})) < f(\iterate{\ma}{i})}, the objective decreases monotonically for each iteration and the equality is achieved when \me{\ma^\ast \in \mc{A}(\ma^\ast)}, which is a generalized fixed point. Therefore, the fixed points in \me{\mc{F}^\ast} are the minimizers for the objective function \me{f} over the set \me{\mc{F}}.

\subsection{Uniqueness}
If the uniqueness of the iterates are guaranteed, then the algorithm will find a unique solution at each iteration as \me{\iterate{\ma}{i+1} = \mc{A}(\iterate{\ma}{i})} and the accumulation point is unique. The convergence of the iterates to a single fixed point is guaranteed by the discussions in \cite{zangwill1969nonlinear,meyer1976sufficient}. Even though the algorithm finds a single fixed point, all fixed points in \me{\mc{F}^\ast} are equally valid as a minimizer for the function \me{f} in \eqref{con}, since the \ac{SINR} in \eqref{eq:SINR} is invariant to the unitary rotations on the beamformers.

The uniqueness of the transmit precoders and the receive beamformers for each convex subproblem is achieved by the constraint \eqref{eqn-8} for the problem \eqref{eqn-9} and \eqref{eqn-mse-1.3} for the \ac{MSE} reformulation in \eqref{eqn-mse-2}. The receive beamformers \me{\mvec{w}{l,k,n}} in \eqref{eqn-10} are also unique for the given set of transmit precoders, since the convex subproblems in \eqref{eqn-9} and \eqref{eqn-mse-2} are susceptible to the transmit precoder phase rotations, \textit{i.e}, unitary transformations. 

When the objective function is zero, the uniqueness of the transmit precoders are not guaranteed by the constraints \eqref{eqn-8} and \eqref{eqn-mse-1.3}, since they are not active at the minimum of the subproblems. To obtain the unique set of transmit precoders and the receive beamformers in all cases, we can regularize the objective of the subproblems as 
\begin{equation}
\| \tilde{\mbf{v}} \|_q + c \, \| \mbf{x} - \mbf{x}^{(i)} \|^2
\end{equation}
where \me{\mbf{x}} is the vector containing all the optimization variables and \me{c} is a positive constant to make the objective strongly convex. The vector \me{ \mbf{x}^{(i)}} is the value of \me{\mbf{x}} in the \me{\ith{i}} iteration. %Note that the uniqueness of the transmit precoders and the receive beamformers for each convex subproblem can be shown by using the constraint \eqref{eqn-8} for the problem \eqref{eqn-9} and \eqref{eqn-mse-1.3} for the \ac{MSE} reformulation in \eqref{eqn-mse-2}. The receive beamformers \me{\mvec{w}{l,k,n}} in \eqref{eqn-10} are also unique for the given set of transmit precoders, since the convex subproblems in \eqref{eqn-9} and \eqref{eqn-mse-2} are susceptible to the transmit precoder phase rotations, \textit{i.e}, unitary transformations. When the objective function is zero, the uniqueness of the transmit precoders are not guaranteed by using \eqref{eqn-8} and \eqref{eqn-mse-1.3} constraints, since they are not active at the optimal solution of the subproblems. To obtain unique set of transmit precoders when the objective is zero even for a single \ac{BS}, we can regularize the objective with the transmit power as discussed in Appendix \ref{a-3} without affecting the optimal value.

\subsection{Stationary Points of the Nonconvex Problem}
To show the stationarity of the fixed points in \me{\mc{F}^\ast}, it must satisfy the \ac{KKT} conditions of \eqref{con}. As \me{i \rightarrow \infty}, it satisfies
\iftoggle{single_column}{
\begin{equation} \label{con-opt}\allowdisplaybreaks
f(\iter{\mx}{\ast}{i},\iter{\my}{\ast}{i},\iter{\mz}{\ast|x}{i}) = f(\iter{\mx}{\ast}{i+1},\iter{\my}{\ast}{i},\iter{\mz}{\ast|y}{i+1}) = 
f(\iter{\mx}{\ast}{i+1},\iter{\my}{\ast}{i+1},\iter{\mz}{\ast|x}{i+1})
\end{equation}}{
\begin{multline} \label{con-opt}\allowdisplaybreaks
f(\iter{\mx}{\ast}{i},\iter{\my}{\ast}{i},\iter{\mz}{\ast|x}{i}) = f(\iter{\mx}{\ast}{i+1},\iter{\my}{\ast}{i},\iter{\mz}{\ast|y}{i+1}) \\ = 
f(\iter{\mx}{\ast}{i+1},\iter{\my}{\ast}{i+1},\iter{\mz}{\ast|x}{i+1})
\end{multline}}
where \me{\{\iter{\mx}{\ast}{i+1},\iter{\my}{\ast}{i+1},\iter{\mz}{\ast|x}{i+1}\} \in \mc{F}^\ast} is a generalized fixed point and the minimizer of the objective function \me{f} in the feasible set \me{\iterate{\mc{X}}{i+1}_{\ast} \subset \mc{F}}. Using \cite[Theorem 10]{lanckriet2009convergence} and \cite[Theorems 2,11]{scutari_1}, we can show that the generalized fixed point is a stationary point of the nonconvex problem \eqref{con}. Since the feasible point \me{\{\iter{\mx}{\ast}{i+1},\iter{\my}{\ast}{i+1},\iter{\mz}{\ast|x}{i+1}\} \in \mc{F}^\ast \subset \mc{F}} is a limiting point of the sequence of iterates \me{\{\iterate{\ma}{i}\}} for \me{\mc{A}} under the mapping \me{f}, it is a stationary point of \eqref{con} over \me{\iterate{\mc{X}}{i+1}_{\ast} \subset \mc{F}}. The feasible point need not be a local optimum for the nonconvex problem unless all the constraints are inactive, \textit{i.e}, if the solution is inside the feasible set \me{\mc{F}}. In this scenario, the Lagrange multipliers are all zero to satisfy the complementary slackness criterion, and therefore, the solution point is a local optimum of the non convex problem in \eqref{con}.
