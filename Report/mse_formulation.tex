
In this section, we solve the \ac{JSFRA} problem by exploiting the equivalence between the \ac{MSE} and the achievable capacity for the receivers designed based on the \ac{MMSE} criterion \cite{viswanath1999optimal,mse_duality}. The \ac{MSE} for the data symbol, represented as \me{\epsilon_{l,k,n}}, is given by
\begin{subeqnarray} \label{mse-error}
\epsilon_{l,k,n} &=& \mathbb{E} \big [ ( d_{l,k,n} - \hat{d}_{l,k,n} )^2 \big ] \\
 &=&  \left | 1 - \mvec{w}{l,k,n}^\herm \mvec{H}{b_k,k,n} \mvec{m}{l,k,n} \right |^2 + \sum_{\mathclap{(j,i) \neq (l,k)}} \left | \mvec{w}{l,k,n}^\herm \mvec{H}{b_i,i,n} \mvec{m}{j,i,n} \right |^2 + N_0 \, \|\mvec{w}{l,k,n}\|^2
\end{subeqnarray}
where \me{\mvec{m}{l,k,n}, \mvec{w}{l,k,n}} denotes the transmit and the receive beamformer and \me{\hat{d}_{l,k,n}} is the received symbol as in \eqref{eqn-1}. Now, replacing the receive beamformer in \eqref{mse-error} with the \ac{MMSE} receiver shown in \eqref{eqn-10}, we obtain the following relation between the \ac{MSE} and the \ac{SINR} as
\begin{equation} \label{eqn-mse-eq}
\epsilon_{l,k,n} = \dfrac{1}{1 + \gamma_{l,k,n}}
\end{equation}
where \me{\gamma_{l,k,n}} is the received \ac{SINR} as in \eqref{eq:SINR}. Using the equivalence in \eqref{eqn-mse-eq}, the \ac{WSRM} objective can be reformulated as the \ac{WMMSE} equivalent to obtain the precoders for the \acs{MU}-\acs{MIMO} scenario as discussed in \cite{christensen2008weighted,wmmse_shi,hong2012decomposition}.

Let \me{{v}^{\prime}_k = Q_k - \sum_{n=1}^N \sum_{l=1}^L t_{l,k,n}} denote the queue deviation corresponding to the user \me{k} and \me{\tilde{{v}}^{\prime}_k \triangleq a_k^{1/{q}}v^{\prime}_k} represents the weighted equivalent. Now, using the \ac{MSE} relation, \eqref{eqn-6} is written as
\begin{IEEEeqnarray}{LL}\label{eqn-mse-1}
\underset{\substack{t_{l,k,n},\mvec{m}{l,k,n},\\ \epsilon_{l,k,n},\mvec{w}{l,k,n}}} {\text{minimize}} & \quad \|  \tilde{\mbf{v}}^{\prime}  \|_q \IEEEyessubnumber \label{eqn-mse-1.1} \\
\text{subject to} & \quad t_{l,k,n} \leq -\log_2(\epsilon_{l,k,n}) \IEEEyessubnumber \label{eqn-mse-1.2} \\
 & \quad \epsilon_{l,k,n} \geq  \left | 1 - \mvec{w}{l,k,n}^\herm \mvec{H}{b_k,k,n} \mvec{m}{l,k,n} \right |^2 + \sum_{\mathclap{(j,i) \neq (l,k)}} \left | \mvec{w}{l,k,n}^\herm \mvec{H}{b_i,i,n} \mvec{m}{j,i,n} \right |^2 + N_0 \, \|\mvec{w}{l,k,n}\|^2 \IEEEyessubnumber \label{eqn-mse-1.3} \\
& \quad \text{and} \; \eqref{eqn-4.3}  \IEEEyessubnumber \label{eqn-mse-1.4}
\end{IEEEeqnarray}
where \eqref{eqn-mse-1.3} bounds the \ac{MSE} by \me{\epsilon_{l,k,n}} and \eqref{eqn-mse-1.2} relaxes the transmitted rate \me{{t}_{l,k,n}} using the \ac{MSE} relation.

The \acl{QM} \ac{JSFRA} problem using alternative \ac{MSE} formulation given by \eqref{eqn-mse-1} is non-convex even for the fixed \me{\mvec{w}{l,k,n}} due to the constraint \eqref{eqn-mse-1.2}. In order to solve this efficiently, we use the \ac{SCA} method as discussed earlier in Section \ref{sec-3.2} by using the linear under estimator for the convex function on the r.h.s of \eqref{eqn-mse-1.2}. The first order Taylor approximation around a fixed \ac{MSE} value \me{\tilde{\epsilon}_{l,k,n}} for \eqref{eqn-mse-1.2} is given by
\begin{IEEEeqnarray}{rCl}
- \log_2(\tilde{\epsilon}_{l,k,n}) - \frac{\left ( {\epsilon}_{l,k,n} - \tilde{\epsilon}_{l,k,n} \right ) }{\log(2) \, \tilde{\epsilon}_{l,k,n}} &\geq& t_{l,k,n}
\label{mse-lin}
\end{IEEEeqnarray}

Now, using the above approximation for the rate constraint, the problem defined by \eqref{eqn-mse-1} can be solved optimal transmit precoders \me{\mvec{m}{l,k,n}}, \acp{MSE} \me{\epsilon_{l,k,n}} and the users rates over each sub-channel \me{t_{l,n,k}} for the fixed receive beamformers. Once the optimal precoders are obtained, the local \ac{MSE} variable  \me{\tilde{\epsilon}_{l,k,n}} is updated with the current update \me{\epsilon_{l,k,n}}. The optimization problem for a fixed receive beamformers \me{\mvec{w}{l,k,n}} is given as
\begin{IEEEeqnarray}{rCl}\label{eqn-mse-2}
\underset{\substack{t_{l,k,n},\mvec{m}{l,k,n},\\\epsilon_{l,k,n}}}{\text{minimize}} &\quad& \|  \tilde{\mbf{v}}^{\prime}  \|_q \IEEEyessubnumber \label{eqn-mse-2.1} \\
\text{subject to} & \quad & \eqref{eqn-4.3}, \: \eqref{mse-lin}  \IEEEyessubnumber \label{eqn-mse-2.2} \\
& \quad & \epsilon_{l,k,n} \geq  \left | 1 - \mvec{w}{l,k,n}^\herm \mvec{H}{b_k,k,n} \mvec{m}{l,k,n} \right |^2 + \sum_{\mathclap{(j,i) \neq (l,k)}} \left | \mvec{w}{l,k,n}^\herm \mvec{H}{b_i,i,n} \mvec{m}{j,i,n} \right |^2 + N_0 \, \|\mvec{w}{l,k,n}\|^2 \IEEEyessubnumber \label{eqn-mse-2.3}
\end{IEEEeqnarray}

\subsubsection*{Convergence}
Following the similar argument from Section \ref{sec-3.2.1}, we can see that the \ac{SCA} subproblems converges to a stationary point. The transmit precoders obtained by the \ac{MSE} relaxations are unique due to the \ac{MSE} expression in \eqref{eqn-mse-2.3}, since it removes the phase ambiguity due to the real valued expression for the transmit precoders as
\begin{equation}
\epsilon_{l,k,n} \geq  1 - 2 \Re{ \left \lbrace \mvec{w}{l,k,n}^\herm \mvec{H}{b_k,k,n} \mvec{m}{l,k,n} \right \rbrace} + \sum_{\mathclap{\forall (j,i)}} \left | \mvec{w}{l,k,n}^\herm \mvec{H}{b_i,i,n} \mvec{m}{j,i,n} \right |^2 + N_0 \, \|\mvec{w}{l,k,n}\|^2
\end{equation}
where \me{\Im{ \left \lbrace \mvec{w}{l,k,n}^\herm \mvec{H}{b_k,k,n} \mvec{m}{l,k,n} \right \rbrace}} is zero due to the real valued \ac{MSE}. Since the transmit precoders are obtained by minimizing the weighted \ac{MSE} criterion, it converges to the stationary \ac{KKT} point at each iteration. The receiver updates are made based on the \ac{MMSE} criterion, it provides the optimal receive beamformers for the given transmit precoders. Alternating between these two precoder designs guarantees the monotonicity of the objective function used in the problem \eqref{eqn-mse-2}. The \ac{MSE} relaxation method converges to the same stationary \ac{KKT} point as of the original \ac{WSRM} problem by making the gradients equal.
