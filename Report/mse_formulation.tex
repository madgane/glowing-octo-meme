
In this section, we solve the \ac{JSFRA} problem by exploiting the equivalence between the \ac{MSE} and the achievable sum rate for the receivers designed based on the \ac{MMSE} criterion \cite{mse_duality}. The \ac{MSE} \me{\epsilon_{l,k,n}}, for the data symbol is given by
\iftoggle{single_column}{
\begin{multline} \label{mse-error}
	\epsilon_{l,k,n} = \mathbb{E} \big [ ( d_{l,k,n} - \hat{d}_{l,k,n} )^2 \big ] = \left | 1 - \mvec{w}{l,k,n}^\herm \mvec{H}{b_k,k,n} \mvec{m}{l,k,n} \right |^2 + \sum_{\mathclap{(j,i) \neq (l,k)}} \left | \mvec{w}{l,k,n}^\herm \mvec{H}{b_i,k,n} \mvec{m}{j,i,n} \right |^2 + \enoise,
\end{multline}
}{
\begin{multline} \label{mse-error}
 \epsilon_{l,k,n} = \mathbb{E} \big [ ( d_{l,k,n} - \hat{d}_{l,k,n} )^2 \big ] = \left | 1 - \mvec{w}{l,k,n}^\herm \mvec{H}{b_k,k,n} \mvec{m}{l,k,n} \right |^2 \\
 + \sum_{\mathclap{(j,i) \neq (l,k)}} \left | \mvec{w}{l,k,n}^\herm \mvec{H}{b_i,k,n} \mvec{m}{j,i,n} \right |^2 + \enoise,
\end{multline}
}
where \me{\hat{d}_{l,k,n}} is the estimate of the transmitted symbol. \review{Using the \ac{MMSE} receive beamformer \eqref{eqn-10} in the \ac{MSE} expression \eqref{mse-error} and in the \ac{SINR} expression \eqref{eq:SINR}, we can arrive at the following relation between the \ac{MSE} and the \ac{SINR} as}
\begin{equation} \label{eqn-mse-eq}
\epsilon_{l,k,n} = \frac{1}{1 + \gamma_{l,k,n}}.
\end{equation}
The above equivalence is valid only if the receivers are based on the \ac{MMSE} criterion. Using the equivalence in \eqref{eqn-mse-eq}, the \ac{WSRM} objective can be reformulated as the \ac{WMMSE} equivalent to obtain the precoders for the \acs{MU}-\acs{MIMO} scenario as discussed in \cite{christensen2008weighted,wmmse_shi,hong2012decomposition}. \review{Note that the receiver is invariably based on the \ac{MMSE} criterion irrespective of the \me{\ell_q} norm used in the objective function to obtain the optimal transmit precoders \me{\mvec{m}{l,k,n}}.}

Let \me{{v}^{\prime}_k = Q_k - \sum_{n=1}^N \sum_{l=1}^L t_{l,k,n}} denote the queue deviation corresponding to user \me{k} and \me{\tilde{{v}}^{\prime}_k \triangleq a_k^{1/{q}}v^{\prime}_k} represents the weighted equivalent. By using the relaxed \ac{MSE} expression in \eqref{mse-error}, the problem in \eqref{eqn-3} can be expressed as
\iftoggle{single_column}{
\begin{IEEEeqnarray}{rCl}\label{eqn-mse-1}
\underset{\substack{t_{l,k,n},\mvec{m}{l,k,n},\\ \epsilon_{l,k,n},\mvec{w}{l,k,n}}} {\text{minimize}} & \quad & \|  \tilde{\mbf{v}}^{\prime}  \|_q \IEEEyessubnumber \label{eqn-mse-1.1} \\
\text{subject to} & \quad & t_{l,k,n} \leq -\log_2(\epsilon_{l,k,n}) \IEEEyessubnumber \label{eqn-mse-1.2} \\
 & \quad & \epsilon_{l,k,n} \geq  \left | 1 - \mvec{w}{l,k,n}^\herm \mvec{H}{b_k,k,n} \mvec{m}{l,k,n} \right |^2 + \sum_{\mathclap{(j,i) \neq (l,k)}} \left | \mvec{w}{l,k,n}^\herm \mvec{H}{b_i,k,n} \mvec{m}{j,i,n} \right |^2 + \enoise \eqspace \IEEEyessubnumber \label{eqn-mse-1.3} \\
& \quad &  \text{and} \; \eqref{eqn-4.3}.  \IEEEyessubnumber \label{eqn-mse-1.4}
\end{IEEEeqnarray}
}{{\allowdisplaybreaks
\begin{IEEEeqnarray}{CL}\label{eqn-mse-1}
	\underset{\substack{t_{l,k,n},\mvec{m}{l,k,n},\\ \epsilon_{l,k,n},\mvec{w}{l,k,n}}} {\text{minimize}} & \quad \|  \tilde{\mbf{v}}^{\prime}  \|_q \IEEEyessubnumber \label{eqn-mse-1.1} \\
	\text{subject to} & t_{l,k,n} \leq -\log_2(\epsilon_{l,k,n}) \IEEEyessubnumber \label{eqn-mse-1.2} \\
	& \sum_{{(j,i) \neq (l,k)}} \left | \mvec{w}{l,k,n}^\herm \mvec{H}{b_i,k,n} \mvec{m}{j,i,n} \right |^2 + \enoise \nonumber \\	
	& \qquad + \left | 1 - \mvec{w}{l,k,n}^\herm \mvec{H}{b_k,k,n} \mvec{m}{l,k,n} \right |^2 \leq \epsilon_{l,k,n}\IEEEyessubnumber \label{eqn-mse-1.3} \\
	& \quad \sum_{n = 1}^N \sum_{k \in \mathcal{U}_b} \sum_{l=1}^L \text{tr} \, (\mvec{m}{l,k,n} \mvec{m}{l,k,n}^\herm) \leq P_{{\max}}, \fall b, \IEEEyessubnumber  \eqspace \label{eqn-mse-1.4}
\end{IEEEeqnarray}}}

An alternative \ac{MSE} formulation given by \eqref{eqn-mse-1} is non-convex even for the fixed \me{\mvec{w}{l,k,n}} due to the constraint \eqref{eqn-mse-1.2}, which is in fact a \ac{DC} constraint. We resort to the \ac{SCA} approach \cite{marks1978technical} by relaxing the constraint by a sequence of convex subsets using first order Taylor series approximation around a fixed \ac{MSE} point \me{\tilde{\epsilon}_{l,k,n}} as
\begin{IEEEeqnarray}{rCl}
- \log_2(\tilde{\epsilon}_{l,k,n}) - \frac{\left ( {\epsilon}_{l,k,n} - \tilde{\epsilon}_{l,k,n} \right ) }{\log(2) \, \tilde{\epsilon}_{l,k,n}} &\geq& t_{l,k,n},
\label{mse-lin}
\end{IEEEeqnarray}

Using the above approximation for the rate constraint, the problem defined in \eqref{eqn-mse-1} is solved for optimal transmit precoders \me{\mvec{m}{l,k,n}}, \acp{MSE} \me{\epsilon_{l,k,n}}, and the user rates over each sub-channel \me{t_{l,n,k}} given the fixed receive beamformers. Once the optimal precoders are obtained, the local \ac{MSE} variable  \me{\tilde{\epsilon}_{l,k,n}} is updated with the current update \me{\epsilon_{l,k,n}}. The optimization problem for a fixed receive beamformers \me{\mvec{w}{l,k,n}} is given as
\begin{IEEEeqnarray}{CCl}\label{eqn-mse-2}
\underset{\substack{t_{l,k,n},\mvec{m}{l,k,n},\epsilon_{l,k,n}}}{\text{minimize}} &\quad& \|  \tilde{\mbf{v}}^{\prime}  \|_q \IEEEyessubnumber \label{eqn-mse-2.1} \\
\text{subject to} & \quad & \eqref{eqn-mse-1.3},\eqref{eqn-mse-1.4},\,\text{and} \:\eqref{mse-lin}. \IEEEyessubnumber \label{eqn-mse-2.2}
\end{IEEEeqnarray}

\begin{comment}
\subsubsection*{Convergence}
Following similar approach as in Section \ref{sec-3.2.1}, at each iteration, the \ac{SCA} subproblems converge to a stationary point of the original nonconvex problem. The uniqueness of the precoders are justified if there is no phase ambiguity in the stationary solution. By reorganizing \eqref{eqn-mse-1.3} as
\iftoggle{single_column}{
\begin{equation}
\epsilon_{l,k,n} \geq  1 - 2 \Re{ \left \lbrace \mvec{w}{l,k,n}^\herm \mvec{H}{b_k,k,n} \mvec{m}{l,k,n} \right \rbrace} + \sum_{\mathclap{\forall (j,i)}} \left | \mvec{w}{l,k,n}^\herm \mvec{H}{b_i,k,n} \mvec{m}{j,i,n} \right |^2 + N_0 \, \|\mvec{w}{l,k,n}\|^2,
\end{equation}
}{
\begin{multline}
	\epsilon_{l,k,n} \geq  1 - 2 \Re{ \left \lbrace \mvec{w}{l,k,n}^\herm \mvec{H}{b_k,k,n} \mvec{m}{l,k,n} \right \rbrace} \\ 
	+ \sum_{\mathclap{\forall (j,i)}} \left | \mvec{w}{l,k,n}^\herm \mvec{H}{b_i,k,n} \mvec{m}{j,i,n} \right |^2 + N_0 \, \|\mvec{w}{l,k,n}\|^2,
\end{multline}
}
we can see that the ambiguity in the phase rotations for the transmit and the receive beamformers are eliminated by the presence of real component in the \ac{MSE} expression. 

At each \ac{SCA} update, the transmit precoders are obtained uniquely by minimizing \eqref{eqn-mse-1} due to the convex nature of the relaxed problem. For a fixed transmit precoders, the \ac{MMSE} receiver improves the objective value \cite{christensen2008weighted,wmmse_shi}, leading to the monotonic convergence of the objective function. At each \ac{SCA} step, the optimal value of the previous iteration is also included in the domain of the problem, and the objective value can either decrease or stays the same after each iteration. Note that the objective function improves at each iteration, whereas the sum rate need not follow the same behavior.

\end{comment}