%The last mile wireless connectivity poses significant bottleneck in the overall data traffic for the interconnected networks. The main challenges in the wireless networks are due to the scarcity of the available resources either in terms of power or spectrum usage and the complexity of the receiver algorithms which impacts the mobile battery consumption. In order to overcome the receiver complexity, \ac{OFDM} is introduced for the wide band transmissions. To improve data rate, multiple antennas are installed at \acp{BS} and/or at user terminals to avail additional freedom in the form of spatial dimension. The inclusion of \ac{MIMO} technique in wireless networks provides higher data rate or lower outage for the same transmission power and bandwidth.
In a network with multiple \acp{BS} serving multiple users, the main driving factor for the transmission is the packets waiting at each \ac{BS} corresponding to the different users present in the network. We consider the problem of transmit precoder design over the space-frequency resources provided by the \ac{MIMO} \ac{OFDM} framework in the downlink \ac{IBC} to minimize the number of queued packets in the \acp{BS}. Since the resources are shared by multiple users associated with different \acp{BS}, the problem of interest can be viewed as a resource allocation one.

In general, the resource allocation problems such as admission control ones can be formulated by assigning a binary variable for each user to indicate the presence or absence of a particular resource \cite{admission_control}. Alternatively, linear transmit precoders, which are complex vectors, can be implicitly modeled as decision variables, thereby avoiding the use of binary decision variables. After the design stage, the non-zero precoders are used to determine the transmission rates of users on a space-frequency resource. A zero transmit precoder indicates the absence of the user on a given resource. In this way, the soft decisions are used in the optimization problem and the hard decisions are made after the algorithm convergences.
%The queue minimizing network optimization objective is used to design the beamformers across the coordinating \acp{BS}, since the transmissions are guided by the available backlogged packets. To achieve the best performance, we propose a joint resource allocation scheme over the space and frequency dimensions among the coordinating \acp{BS} to minimize the time that the packets stay in the queues prior to the transmission, and, hence, to avoid packet drops as an indirect objective.

The queue minimizing precoder designs are closely related to the \ac{WSRM} problem with additional rate constraints to limit the throughput beyond the number of backlogged packets associated with the users. The topics on \ac{MIMO} \ac{IBC} precoder design have been studied extensively with different performance criteria in the literature. Due to the nonconvex nature of the \ac{MIMO} \ac{IBC} precoder design problems, \ac{SCA} approach has become a powerful tool to deal with these problems. For example, in \cite{sin_algorithm}, the nonconvex part of the objective is linearized around an operating point to solve the \ac{WSRM} problem in a iterative manner. A similar approach using arithmetic-geometric inequality was proposed in \cite{tran2012fast}.

The relation between the achievable sum rate and the \ac{MSE} of the received symbol by using fixed \ac{MMSE} receivers can be used to solve the \ac{WSRM} problem \cite{mse_duality}. In \cite{christensen2008weighted,wmmse_shi}, the \ac{WSRM} problem is reformulated via \ac{MSE}, casting the problem as a convex one for fixed linearization coefficients. In this way, the original problem is expressed in terms of the \ac{MSE} weight, precoders, and decoders. Then the problem is solved using an alternating optimization method, i.e., finding a subset of variables while the other variables are fixed. The \ac{MSE} reformulation for the \ac{WSRM} problem was also studied in \cite{hong2012decomposition} by using \ac{SCA} to solve the problem in an iterative manner. Moreover, distributed precoder designs with \ac{QoS} requirements as additional rate constraints are studied for the \ac{MSE} reformulated \ac{WSRM} problem in \cite{kaleva2013primal,kaleva2013decentralized}.

The problem of precoder design for the \ac{MIMO} \ac{IBC} system can be solved either by using a centralized controller or by using decentralized algorithms, where each \ac{BS} handles the corresponding subproblem independently with the limited information exchange with other \acp{BS} via back-haul. The distributed approaches are usually based on the primal, or dual decompositions or the \ac{ADMM}, as discussed in \cite{palomar2006tutorial,boyd2011distributed}. In the  primal decomposition, the so-called coupling interference variables are fixed for the subproblem at each \ac{BS} to find the optimal precoders. The fixed interference is then updated using the subgradients as discussed in \cite{pennanen2011decentralized}. The dual and the \ac{ADMM} approaches control the distributed subproblems by fixing the \emph{`interference price'} for each \ac{BS} as detailed in \cite{tolli2011decentralized}.

By adjusting the weights in the \ac{WSRM} objective, we can find an arbitrary rate-tuple in the rate region that maximizes suitable objective measures. For example, if the weight of each user is set to be inversely proportional to its average data rate, the corresponding problem guarantees fairness on the average among the users. To reduce the number of backlogged packets, we can assign weights based on the current queue size of the users. Specifically, the queue states can be incorporated in the \ac{WSRM} objective $\sum_k w_k R_k$ by replacing the weight $w_k$ with the corresponding queue state $Q_k$ or its function, which is the outcome of minimizing the Lyapunov drift between the current and the future queue states \cite{neely2010stochastic}, where \me{R_k} denotes the achievable data rate of user \me{k}. In the \textit{backpressure algorithm}, the differential queues between the source and the receiver nodes are used to scale the transmission rate \cite{georgiadis2006resource}.

Earlier studies on the queue minimization problem are summarized in the survey papers \cite{berry2004cross,layering_as_opt}. In particular, the problem of power allocation to minimize the number of backlogged packets was considered in \cite{qps_cioffi} using geometric programming. Since the problem addressed in \cite{qps_cioffi} assumed single antenna transmitters and receivers, the queue minimizing problem reduces to the optimal power allocation problem. In the context of wireless networks, the \textit{backpressure algorithm} mentioned above was extended in \cite{weeraddana2011resource} by formulating the corresponding user queues as the weights in the \ac{WSRM} problem. Recently, the precoder design for the video transmission over \ac{MIMO} system was considered in \cite{video_queues}. In this design, the \ac{MU}-\ac{MIMO} precoders are designed by the \ac{MSE} reformulation as in \cite{christensen2008weighted} with the higher layer performance objectives such as playback interruptions and buffer overflow probabilities.

\subsubsection*{Main Contributions}
In this paper, we design precoders jointly over space-frequency resources to reduce the number of backlogged packets waiting at each \acp{BS}. The proposed formulation also limits the allocations beyond the number of backlogged packets without explicit rate constraints. Initially, we propose a centralized \ac{JSFRA} formulation, which is solved by two iterative algorithms based on the combination of \ac{SCA} and \ac{AO} due to the nonconvex nature of the problem. The proposed algorithms solve a sequence of convex problems obtained by fixing a subset of optimization variables or by approximating the nonconvex constraints by the convex ones. The first approach is performed by directly relaxing the \ac{SINR} expression, while in the second method, the equivalence between the \ac{MSE} and the \ac{SINR} is exploited. We then discuss the distributed implementation of the \ac{JSFRA} methods using primal decomposition and the \ac{ADMM}. Finally, we also propose a more practical iterative precoder design by directly solving the \ac{KKT} system of equations for the \ac{MSE} reformulation that is numerically shown to require minimal information exchange for each update. Note that the joint space-frequency channel matrix can be formed by stacking the channel of each sub-channel in a block-diagonal form for all users.

The rest of the paper is organized as follows. Section \ref{sec-2-3.2} introduces the system model and the problem formulation. The existing and the proposed centralized designs are presented in Section \ref{sec-3}. The distributed solutions are provided in Section \ref{sec-4} followed by the simulation results in Section \ref{sec-5}. Conclusions are drawn in Section \ref{sec-6}.