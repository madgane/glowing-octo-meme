%The last mile wireless connectivity poses significant bottleneck in the overall data traffic for the interconnected networks. The main challenges in the wireless networks are due to the scarcity of the available resources either in terms of power or spectrum usage and the complexity of the receiver algorithms which impacts the mobile battery consumption. In order to overcome the receiver complexity, \ac{OFDM} is introduced for the wide band transmissions. To improve data rate, multiple antennas are installed at \acp{BS} and/or at user terminals to avail additional freedom in the form of spatial dimension. The inclusion of \ac{MIMO} technique in wireless networks provides higher data rate or lower outage for the same transmission power and bandwidth.

In a network with multiple \acp{BS} serving \acp{MU}, the main driving factor for the transmission are the packets waiting at each \ac{BS} corresponding to the different users present in the network. These available packets are transmitted over the shared wireless resources subject to certain system limitations and constraints. We consider the problem of transmit precoder design over the space-frequency resources provided by the \ac{MIMO} \ac{OFDM} framework in the downlink \ac{IBC} to minimize the number of queued packets. Since the space-frequency resources are shared by multiple users associated with different \acp{BS}, it can be viewed as a resource allocation problem.

In general, the resource allocation problems are formulated by assigning a binary variable to each user indicating the presence or the absence of a particular resource \cite{admission_control}. In contrast to that, we use linear transmit precoders, which are the complex vectors, as a decision variable in determining the presence or the absence of a user on a particular resource. The purpose is two-fold. Firstly, the formulation determines the transmission rate on a certain resource, and, secondly, by making the transmit beamformer of a particular user to be a zero vector, the corresponding user will not be scheduled on a certain resource.

%The queue minimizing network optimization objective is used to design the beamformers across the coordinating \acp{BS}, since the transmissions are guided by the available backlogged packets. To achieve the best performance, we propose a joint resource allocation scheme over the space and frequency dimensions among the coordinating \acp{BS} to minimize the time that the packets stay in the queues prior to the transmission, and, hence, to avoid packet drops as an indirect objective.

The queue minimizing precoder designs are closely related to the \ac{WSRM} problem with additional rate constraints determined by the number of backlogged packets for each user in the system. The topics on \ac{MIMO} \ac{IBC} precoder design have been studied extensively with different performance criteria in the literature. Due to the nonconvex nature of the \ac{MIMO} \ac{IBC} precoder design problems, the \ac{SCA} method has become a powerful tool to deal with these problems \cite{bertsekas1999nonlinear}. For example, in \cite{sin_algorithm}, the nonconvex part of the objective has been linearized around an operating point in order to solve the \ac{WSRM} problem in an iterative manner. Similar approach of solving the \ac{WSRM} problem by using arithmetic-geometric inequality has been proposed in \cite{tran2012fast}.

The connection between the achievable capacity and the \ac{MSE} for the received symbol by using the fixed \ac{MMSE} receivers as shown in \cite{viswanath1999optimal,mse_duality} can also be used to solve the \ac{WSRM} problem. In \cite{christensen2008weighted,wmmse_shi}, the \ac{WSRM} problem is reformulated via \ac{MSE}, casting the problem as a convex one for fixed linearization coefficients. In this way, the original problem is expressed in terms of the \ac{MSE} weight, precoders, and decoders. Then the problem is solved using an alternating optimization method, i.e., finding a subset of variables while the remaining others are fixed. The \ac{MSE} reformulation for the \ac{WSRM} problem has also been studied in \cite{hong2012decomposition} by using the \ac{SCA} to solve the problem in an iterative manner. Additional rate constraints based on the \ac{QoS} requirements were included in the \ac{WSRM} problem and solved via \ac{MSE} reformulation in \cite{kaleva2013primal,kaleva2013decentralized}.

The problem of precoder design for the \ac{MIMO} \ac{IBC} system are solved either by using a centralized controller or by using decentralized algorithms where each \ac{BS} handles the corresponding subproblem independently with the limited information exchange with the other \acp{BS} via back-haul. The distributed approaches are based on primal, dual or \ac{ADMM} decomposition, which has been discussed in \cite{palomar2006tutorial,boyd2011distributed}. In the  primal decomposition, the so-called coupling interference variables are fixed for the subproblem at each \ac{BS} to find the optimal precoders. The fixed interference are then updated by using the subgradient method as discussed in \cite{pennanen2011decentralized}. The dual and \ac{ADMM} approaches control the distributed subproblems by fixing the \emph{`interference price'} for each \ac{BS} as detailed in \cite{tolli2011decentralized}.

By adjusting the weights in the \ac{WSRM} objective properly, we can find an arbitrary rate-tuple in the rate region that maximizes the suitable objective measures. For example, if the weight of each user is set to be inversely proportional to its average data rate, the corresponding problem guarantees fairness on an average among the users. As an approximation, we may assign weights based on the current queue size of the users. More specifically, the queue states can be incorporated to traditional weighted sum rate objective $\sum_k w_k R_k$ by replacing the weight $w_k$ with the corresponding queue state $Q_k$ or its function, which is the outcome of minimizing the Lyapunov drift between the current and the future queue states \cite{tassiulas,neely2010stochastic}. In backpressure algorithm, the differential queues between the source and the destination nodes are used as the weights scaling the transmission rate \cite{georgiadis2006resource}.

Earlier studies on the queue minimization problem were summarized in the survey paper \cite{berry2004cross,layering_as_opt}. In particular, the problem of power allocation to minimize the number of backlogged packets was considered in \cite{qps_cioffi} using geometric programming. Since the problem addressed in \cite{qps_cioffi} assumed single antenna transmitters and receivers, the queue minimizing problem reduces to the optimal power allocation problem. In the context of wireless networks, the backpressure algorithm mentioned above was extended in \cite{weeraddana2011resource} by formulating the corresponding user queues as the weights in the \ac{WSRM} problem. Recently, the precoder design for the video transmission over \ac{MIMO} system is considered in \cite{video_queues}. In this design, the \ac{MU}-\ac{MIMO} precoders are designed by the \ac{MSE} reformulation as in \cite{christensen2008weighted} with the higher layer performance objective such as playback interruptions and buffer overflow probabilities.

In this paper, we consider the problem of precoder design across the space-frequency resources to minimize the total number of queued packets waiting in all \acp{BS}. For this highly nonconvex problem, we first propose two centralized methods. In the first method, we relax the nonconvex constraint by the first order Taylor approximation around an operating point, which is updated in an iterative manner until convergence or to a certain accuracy. In the second method, we reformulate the \ac{JSFRA} problem using the \ac{MSE} equivalence with the rate expression to solve for the optimal precoders. For a distributed implementation, we further propose decentralized approaches based on primal and \ac{ADMM} schemes to identify the precoders independently across the \acp{BS} by exchanging limited information via back-haul. We also propose an iterative algorithm by solving the \ac{KKT} equations, which can be implemented efficiently in a distributed manner.

The paper is organized as follows. In Section \ref{sec-2-3.2}, we introduce the system model and the problem formulation for the queue minimizing precoder design. The existing and the proposed centralized precoder designs are presented in Section \ref{sec-3}. The distributed solutions are provided in Section \ref{sec-4} followed by the simulation results in Section \ref{sec-5}. Conclusions are drawn in Section \ref{sec-6}.