We consider the problem of resource allocation for a multi-cell \ac{MU-MIMO} transmission over an \ac{OFDM} system with the objective of minimizing the total number of backlogged packets of all users at a given instant. The available space and frequency resources in the considered system are shared among the users by \ac{BS} cooperation in order to minimize the number of packets waiting for transmission from each \ac{BS} to the respective users. In this paper, beamforming technique is utilized at \acp{BS} and linear single-user detection is employed at each receiver, i.e., the inter-user interference is treated as background noise. The queue minimizing network optimization objective is used to design beamformers across the coordinating \acp{BS}, since the transmissions are guided by the available backlogged packets. To achieve the best performance, we propose a joint resource allocation scheme over the space and frequency dimensions among the coordinating \acp{BS} to minimize the time that the packets stay in queues prior to transmission, and, hence, to avoid packet drops as an indirect objective.

Many existing beamformer designs for similar system models addresses the problem of \ac{WSRM} objective. By adjusting the weights properly, we can use the \ac{WSRM} schemes to achieve other performance measures. For example, if the weight of each user is set to be inversely proportional to his/her data rate, a \ac{WSRM} scheme can maintain fairness among users. Similarly, the \ac{WSRM} schemes can be used to solve the design problem considered in this paper. Specifically, to find a rate vector that minimizes the number of queued packets, we should assign weights based on the current queue size of users. More explicitly, the queue states can be incorporated to traditional weighted sum rate objective $\sum_k w_k R_k$ by replacing the weight $w_k$ with the corresponding queue state $Q_k$ or a function depending on it \cite{neely2010stochastic}, where the \ac{Q-WSRM} scheme is the outcome of minimizing the Lyapunov drift between the current and the future queue states.

The topic of \ac{MIMO} \ac{BC} precoder design has been studied extensively with different performance criteria in the literature. For the problem of \ac{WSRM} utility, due to the nonconvex nature of the linear \ac{MIMO} \ac{BC} precoder design, \cite{christensen2008weighted,wmmse_shi} addressed the problem using a reformulation via \ac{MMSE}, casting the problem as a convex one for  fixed receivers. In this way, the original problem is expressed in terms of the \ac{MSE} weight, precoders, and decoders. Then the problem is solved using an alternating optimization method, i.e., finding a set of variables while the remaining others are fixed. The \ac{WSRM} problem using an alternative \ac{MMSE} reformulation along with the additional rate constraints is considered in \cite{kaleva2013primal}. A fast converging algorithm for the \ac{WSRM} problem is proposed in \cite{tran2012fast} using a surrogate for a convex program at each iteration.

Earlier work on the queue minimization problem was addressed in the survey paper \cite{berry2004cross} and in particular \cite{qps_cioffi}, where the problem of optimal power allocation is considered to minimize the number of backlogged packets via \acl{GP} formulation. Since the problem considered in \cite{qps_cioffi} assumes multi-user \ac{OFDM} model without multi-antenna support, the queue minimizing scheduling reduces to the problem of optimal power allocation.

In this paper, we consider the problem of designing the precoders for a \ac{MU-MIMO} \ac{OFDM} scenario jointly across space-frequency dimension to minimize the number of queued packets associated with the users. We compare the performance of the proposed \ac{QM} \ac{JSFRA} scheme over the \ac{Q-WSRM} scheme performed over each sub-channel, where the weights are updated after each sub-channel with the unserved packets. We show that the \ac{JSFRA} scheme provides better performance over the sub-channel wise resource allocation technique like \ac{Q-WSRM} scheme.
