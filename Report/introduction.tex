
The last mile wireless connectivity poses significant bottleneck in the overall data traffic in the interconnected networks. The main challenges in the wireless networks are due to the scarcity of the available resources either in terms of power or spectrum usage and the complexity of the receiver algorithms which is proportional to the battery drain. In order to over come the receiver complexity, \ac{OFDM} based transmissions are introduced for the wideband transmissions. To improve the data rate, multiple antennas are installed at the \ac{BS} and at the user terminals to avail the additional freedom in the form of spatial dimension. The inclusion of \ac{MIMO} technique in the wireless network provides increased data rate or lower outage for the same transmission power and the channel use.

In a network with multiple \acp{BS} serving multiple users (\acs{MU}), the main driving factor for the transmission are the packets waiting at each \ac{BS} corresponding to the different users present in the network. These available packets are transmitted over the shared wireless resources in an efficient manner with the various system constraints. In this work, we consider the user allocation over the space-frequency resources provided by the \ac{MU-MIMO} \ac{OFDM} framework in the downlink broadcast transmission to reduce the number of queued packets corresponding to the users at the \acp{BS}. Since the space-frequency resources are shared by multiple users associated with different \acp{BS}, the problem can be considered as a resource allocation to minimize the number of backlogged packets at the \acp{BS}.

In general, the resource allocation problems are solved by assigning a binary variable to each user indicating the presence or the absence on a particular resource. In contrast to that, we use the transmit beamformers, which are the complex vectors, as a decision variable in determining the presence or the absence of a user on a particular resource. The purpose of using the transmit beamformers for the scheduling is two fold. Firstly, it determines the transmission rate on a certain resource and secondly, by making the transmit beamformer to be a zero vector, the corresponding user will not be scheduled on a certain resource. The queue minimizing network optimization objective is used to design beamformers across the coordinating \acp{BS}, since the transmissions are guided by the available backlogged packets. To achieve the best performance, we propose a joint resource allocation scheme over the space and frequency dimensions among the coordinating \acp{BS} to minimize the time that the packets stay in queues prior to transmission, and, hence, to avoid packet drops as an indirect objective.

The queue minimizing precoder designs are nothing but the \ac{WSRM} problem with the additional rate constraint determined by the number of backlogged packets for each user in the system. The topic of \ac{MIMO} \ac{BC} precoder design has been studied extensively with different performance criteria in the literature. Due to the nonconvex nature of the linear \ac{MIMO} \ac{BC} precoder design problem, it is solved by the sequence of approximated convex problems. In \cite{sin_algorithm}, the \ac{MIMO} rate expression is linearized around a fixed point in order to solve the \ac{DC} problem as a concave maximization problem in an iterative manner. Similar approach of solving the \ac{WSRM} problem via arithmetic-geometric inequality is proposed in \cite{tran2012fast}, providing better convergence rate compared to the former.

The relation between the achievable capacity and the \ac{MSE} for the received symbol by using the fixed \ac{MMSE} receivers as shown in \cite{viswanath1999optimal,mse_duality} are also used to solve the \ac{WSRM} objective. In \cite{christensen2008weighted,wmmse_shi}, the \ac{WSRM} problem is reformulated via \ac{MMSE}, casting the problem as a convex one for fixed \ac{MMSE} receivers. In this way, the original problem is expressed in terms of the \ac{MSE} weight, precoders, and decoders. Then the problem is solved using an alternating optimization method, i.e., finding a set of variables while the remaining others are fixed. Additional rate constraints based on the \ac{QoS} requirements are included in the \ac{WSRM} objective are solved via \ac{MSE} reformulation in \cite{kaleva2013primal}.

The problem of precoder design for the \ac{MIMO} \ac{BC} system are solved either by using the centralized controller or at each \ac{BS} independently by exchanging limited information via backhaul. The distributed approaches are performed using the primal or by the dual decomposition. In primal decomposition, the coupling interference variables are fixed for each \ac{BS} wise subproblems and solved for the optimal precoders. The fixed interference are then updated by using the subgradients as discussed in \cite{pennanen2011decentralized}. The dual approach controls the distributed subproblems by fixing the interference price for each \ac{BS} as detailed in \cite{tolli2011decentralized}.

By adjusting the weights properly, we can use the \ac{WSRM} schemes to achieve other performance measures. For example, if the weight of each user is set to be inversely proportional to his/her data rate, a \ac{WSRM} scheme can maintain fairness among users. Similarly, the \ac{WSRM} schemes can be used to solve the design problem considered in this paper. Specifically, to find a rate vector that minimizes the number of queued packets, we should assign weights based on the current queue size of users. The queue states can be incorporated to traditional weighted sum rate objective $\sum_k w_k R_k$ by replacing the weight $w_k$ with the corresponding queue state $Q_k$ or a function depending on it \cite{tassiulas,neely2010stochastic}, which is the outcome of minimizing the Lyapunov drift between the current and the future queue states. Similar algorithms are discussed in the networking literature as the backpreassure algorithm, where the differential queues between the source and the destination nodes are used as the weights scaling the transmission rate \cite{georgiadis2006resource}.

Earlier work on the queue minimization problem was addressed in the survey paper \cite{berry2004cross} and in particular \cite{qps_cioffi}, where the problem of optimal power allocation is considered to minimize the number of backlogged packets via \acl{GP} formulation. Since the problem considered in \cite{qps_cioffi} assumes multi-user \ac{OFDM} model without multi-antenna support, the queue minimizing scheduling reduces to the problem of optimal power allocation. In the context of the wireless networks, the backpreassure algorithm has been extended by formulating the wireless transmission rate as a \ac{MIMO} \ac{BC} optimization problem by using the corresponding user queues as the weight for the associated rate \cite{weeraddana2011resource}.

In the traditional \ac{WSRM} formulation, the precoders are designed for the sub-channels independently due to the interference-free \ac{OFDM} transmission. In this work, since the objective is to identify the transmission rates for all the users across the space-frequency resources, the precoders are to be designed jointly across the space-frequency dimension. In order minimize the number of queued packets at the \acp{BS}, the precoders are designed by the proposed \ac{JSFRA} formulation with two different approaches in a centralized manner. We also propose distributed approaches for the centralized problem for designing the precoders independently across the \acp{BS} with the limited information exchange.

The outline of this paper is as follows. In Section \ref{sec-2-3.2}, we propose the system model and the problem formulation for the current queue minimizing precoder design. Existing and the proposed precoder designs for the queue minimizing objective is discussed elaborately in Section \ref{sec-3}. The distributed solutions are outlined in Section \ref{sec-4} followed by the simulation results for both centralized and the distributed algorithms are provided in Section \ref{sec-5}. Conclusions are drawn in Section \ref{sec-6}.