The queue minimizing algorithms have been studied extensively in the networking literature to provide congestion-free routing between any two nodes in the network. One such algorithm is the \emph{backpressure algorithm} \cite{tassiulas,georgiadis2006resource,neely2010stochastic}. It finds an optimal control policy in the form of rate or resource allocation by considering the differential backlogged packets between the source and the destination nodes for the networking entities. %Even though the algorithm is primarily designed for the wired infrastructure, it can be extended to the wireless networks by designing the user rate variable \me{t_k} in accordance to the wireless network.

The \ac{Q-WSRM} formulation extends the \emph{backpressure algorithm} to the downlink \ac{MIMO}-\ac{OFDM} framework, in which the multiple \acp{BS} act as the source nodes and the user terminals as the receiver nodes. The control policy in the form of transmit precoders aims at minimizing the number of queued packets waiting in the \acp{BS}. In order to find the optimal strategy, we resort to the Lyapunov theory, which is predominantly used in the control theory to achieve system stability. Since at each time slot, the system is described by the channel conditions and the number of backlogged packets of each user, the Lyapunov function is used to provide a scalar measure, which grows large when the system moves toward the undesirable state. By following \cite{neely2010stochastic}, the scalar measure for the queue stability is given by
\begin{equation}
\mathrm{L}\sset{\mbf{Q}(i)} = \frac{1}{2} \, \sum_{k \in \mc{U}} Q_k^2(i)
\end{equation}
where \me{\mbf{Q}(i) = \sset{Q_1(i),Q_2(i),\dotsc,Q_K(i)}^\mathrm{T}} and \me{\tfrac{1}{2}} is used for the convenience. It provides a scalar measure of congestion present in the system \cite[Ch. 3]{neely2010stochastic}. 

To minimize the total number of backlogged packets for time instant \me{i}, the optimal transmission rate of all users is obtained by minimizing the Lyapunov drift expressed as  
\iftoggle{single_column}{
\begin{equation} \label{eqn-3.1}
\mathrm{L}\sset{\mbf{Q}(i+1)} - \mathrm{L}\sset{\mbf{Q}(i)} = \frac{1}{2} \Big [ \sum_{k \in \mc{U}} \, \Big ( \left [ Q_k(i) - t_k(i) \right ]^+ + \lambda_k(i) \Big )^2 - Q^2_k(i) \Big ].
\end{equation}}{
\begin{multline} \label{eqn-3.1}
\mathrm{L}\sset{\mbf{Q}(i+1)} - \mathrm{L}\sset{\mbf{Q}(i)} = \\ \frac{1}{2} \Big [ \sum_{k \in \mc{U}} \, \Big ( \left [ Q_k(i) - t_k(i) \right ]^+ + \lambda_k(i) \Big )^2 - Q^2_k(i) \Big ].
\end{multline}}
In order to eliminate the nonlinear operator \me{[x]^+}, we bound the expression in \eqref{eqn-3.1} as
\begin{equation}
\leq \sum_{k \in \mc{U}} \, \frac{\lambda^2_k(i) + t_k^2(i)}{2} + \sum_{k \in \mc{U}} Q_k(i) \set{ \lambda_k(i) - t_k(i) }
\label{drift-exp}
\end{equation}
by using the following inequality 
\begin{equation}
\sset{\max(Q-t,0) + \lambda)}^2 \leq Q^2 + t^2 + \lambda^2 + 2 Q (\lambda - t).
\end{equation}
The total number of backlogged packets at any given instant \me{i} is reduced by minimizing the conditional expectation of the Lyapunov drift expression \eqref{drift-exp} given the current number of queued packets \me{\mbfa{Q}(i)} waiting in the system. The expectation is taken over all possible arrival and transmission rates of the users to obtain the optimal rate allocation strategy. 

Now, the conditional Lyapunov drift, denoted by \me{\Delta{(\mbfa{Q}(i))}}, is given by the infimum over the transmission rate as
\iftoggle{single_column}{
\begin{IEEEeqnarray}{rCl} \label{backpressure-approx} \neqsub
	\underset{\mbf{t}}{\text{inf}} &\quad& \Delta(\mbf{Q}(i)) \triangleq \mathbb{E}_{\mbfa{\lambda},\mbf{t}} \set{\mathrm{L}\sset{\mbf{Q}(i+1)} - \mathrm{L}\sset{\mbf{Q}(i)} \vert \mbf{Q}(i)} \IEEEyessubnumber \\
	&\leq& \underbrace{\mathbb{E}_{\mbfa{\lambda},\mbf{t}} \set {\sum_{k \in \mc{U}} \, \frac{\lambda^2_k(i) + t_k^2(i)}{2} \vert \mbf{Q}(i)}}_{\le B} + \sum_{k \in \mc{U}} Q_k(i) A_k(i) - \mathbb{E}_{\mbfa{\lambda},\mbf{t}}\set{\sum_{k \in \mc{U}} Q_k(i) t_k(i)  \vert \mbf{Q}(i)}, \eqsub \label{backpressure-approx-1b}
\end{IEEEeqnarray}}{
\begin{IEEEeqnarray}{CL} \label{backpressure-approx} \neqsub
	\underset{\mbf{t}}{\text{inf}} \quad & \mathbb{E}_{\mbfa{\lambda},\mbf{t}} \set{\mathrm{L}\sset{\mbf{Q}(i+1)} - \mathrm{L}\sset{\mbf{Q}(i)} \vert \mbf{Q}(i)} \eqsub \\
	\leq & \underbrace{\mathbb{E}_{\mbfa{\lambda},\mbf{t}} \Big \lbrace \sum_{k \in \mc{U}} \frac{\lambda^2_k(i) + t_k^2(i)}{2} \vert \mbf{Q}(i) \Big \rbrace }_{\le B} + \sum_{k \in \mc{U}} Q_k(i) A_k(i) \nonumber \\
	& \qquad \qquad {} - \mathbb{E}_{\mbfa{\lambda},\mbf{t}}\Big \lbrace \sum_{k \in \mc{U}} Q_k(i) t_k(i)  \vert \mbf{Q}(i) \Big \rbrace, \eqsub \label{backpressure-approx-1b}
\end{IEEEeqnarray}}
where the subscripts \me{\mbf{t}} and \me{\mbfa{\lambda}} represents the vector formed by stacking the transmission and the arrival rate of all users in the system. Since the transmission and the arrival rates are bounded, the second order moments in the first term of \eqref{backpressure-approx-1b} can be bounded by a constant \me{B} without affecting the optimal solution of the problem \cite{neely2010stochastic}. The second term in \eqref{backpressure-approx-1b} follows from the Poisson arrival process.

The expression in \eqref{backpressure-approx} looks similar to the \ac{WSRM} formulation if the weights in the \ac{WSRM} problem are replaced by the numbers of backlogged packets of the corresponding users. The above approach was extended for the wireless networks in \cite{weeraddana2011resource}, in which the queues were used as weights in the \ac{WSRM} formulation to determine the transmit precoders. Since the expectation is minimized by minimizing the function inside, the \ac{Q-WSRM} formulation is given by 
\begin{IEEEeqnarray}{LCl} \label{q_gen_sum} \neqsub
\underset{\substack{\mvec{m}{l,k,n}, \mvec{w}{l,k,n}}}{\text{maximize}} &\quad& \sum_{k \in \mc{U}} \, Q_k \left ( \sum_{n = 1}^N \sum_{l=1}^L \log_2(1 + \gamma_{l,k,n}) \right ) \IEEEyessubnumber \label{eqn-3.1.1} \\
\text{subject to.} & \quad & \sum_{n = 1}^N \sum_{k \in \mathcal{U}_b} \sum_{l=1}^L \trace \, (\mvec{m}{l,k,n} \mvec{m}{l,k,n}^\herm) \leq P_{{\max}} \fall b. \IEEEyessubnumber \eqspace \label{eqn-3.1.3}
\end{IEEEeqnarray}

To avoid excessive allocation of the resources, we include an additional rate constraint \me{t_k \leq Q_k} to address \me{[x]^+} operation in \eqref{eqn-2a}. The rate constrained version of the \ac{Q-WSRM}, denoted by \ac{Q-WSRME} problem for a cellular system, is given by \label{q_gen_sum-1} with the additional constraints as
\begin{IEEEeqnarray}{rCl} \label{eqn-3.1.4}
\sum_{n=1}^N \sum_{l = 1}^L \log_2(1 + \gamma_{l,k,n}) \leq Q_k \: \fall k \in \mc{U}
\end{IEEEeqnarray}
where the precoders are associated with \me{\gamma_{l,k,n}} defined in \eqref{eq:SINR}. By using the number of queued packets as the weights, the resources can be allocated to the user with more backlogged packets; this essentially results in greedy allocation.

%As a special case of the problem defined in \eqref{q_gen_sum}, we can formulate the sum rate maximization problem by setting the weights in \eqref{eqn-3.1.1} as unity, leading to the problem as in \eqref{q_gen_sum} with \me{Q_k = 1, \forall k \in \mc{U}}. This approach provides a greedy queue minimizing allocation as compared to \ac{Q-WSRME}, since the resource allocation is driven by the channel conditions in comparison to the number of queued packets as in \ac{Q-WSRME}. Note that in both formulations, the resources allocated to the users are limited by the number of backlogged packets with an explicit maximum rate constraint defined by \eqref{eqn-3.1.4}.
