
The queue minimizing algorithms are discussed extensively in the networking literature for providing congestion-free routing between any two nodes\footnote{routers or user terminals} in the network. It is achieved by finding a control policy considering the differential backlogs of the packets present in the source and the destination node, where control policy refers to the resource or rate allocation strategy. The control policy satisfying the condition is known as \emph{backpreassure algorithm}, discussed extensively in \cite{tassiulas,georgiadis2006resource,neely2010stochastic}. The algorithm assumes the physical layer rate as a fixed entity due to the wired infrastructure. In order to use the algorithm in the wireless network, the user rate variable \me{t_{k}} should be modeled in accordance with the wireless network.

The \ac{Q-WSRM} scheme is formulated by extending the \emph{backpreassure algorithm}\footnote{even though, the algorithm's objective is to provide congestion free transmission, we use this scheme for the comparison purpose} for the \ac{MIMO} \ac{OFDM} network, where multiple \acp{BS} acts like a source nodes and the user terminals as a receiver nodes. The user rate variables \me{t_k} are obtained by the design of the transmit and the receive precoders to maximize the weighted sum rate. The weights used in the precoder design algorithm should be a function of the number of backlogged packets corresponding to each user. In order to find the weights, we adopt the Lyapunov drift minimization as discussed in \cite{neely2010stochastic}, where the Lyapunov function is given by the squared sum of the number backlogged packets as
\begin{equation}
\mathrm{L}\sset{\mbf{Q}(i)} = \frac{1}{2} \, \sum_{k \in \mc{U}} Q_k(i),
\end{equation}
where \me{\mbf{Q}(i)} denotes the stacked user queues at the \me{\ith{i}} slot. The Lyapunov function provides a measure of congestion in the system, as discussed in \cite[Ch. 3]{neely2010stochastic}. Now the Lyapunov function drift is given by
\begin{IEEEeqnarray}{rCl}\label{eqn-3.1}
\mathrm{L}\sset{\mbf{Q}(i+1)} - \mathrm{L}\sset{\mbf{Q}(i)} &=& \frac{1}{2} \left [ \sum_{k \in \mc{U}} \, \Big ( \left [ Q_k(i) - t_k(i) \right ]^+ + \lambda_k(i) \Big )^2 - Q^2_k(i) \right ] \IEEEyessubnumber \label{eqn-3.1.0} \\
&\leq& \sum_{k \in \mc{U}} \, \frac{\lambda^2_k(i) + t_k^2(i)}{2} + \sum_{k \in \mc{U}} Q_k(i) \set{ \lambda_k(i) - t_k(i) }, \IEEEyessubnumber \label{drift-exp}
\end{IEEEeqnarray}
where the inequality is due to the upper bound
\begin{equation}
\sset{\max(Q-t,0) + \lambda)}^2 \leq Q^2 + t^2 + \lambda^2 + 2 Q (\lambda - t).
\end{equation}
In order to minimize the number of backlogged packets at each instant, minimization of the Lyapunov drift \eqref{eqn-3.1} is carried over all possible control decisions in the form of transmission rates \me{t_k} to users in the system. The Lyapunov drift conditioned on the current backlogged packets \me{\mbfa{Q}(i)} is given by
\begin{IEEEeqnarray}{rCl}
\underset{\mbf{t}}{\text{minimize}} && \Delta(\mbf{Q}(i)) \triangleq \mathbb{E} \set{\mathrm{L}\sset{\mbf{Q}(i+1)} - \mathrm{L}\sset{\mbf{Q}(i)} \vert \mbf{Q}(i)} \IEEEyessubnumber \\
 &\leq& \mathbb{E} \set {\sum_{k \in \mc{U}} \, \frac{\lambda^2_k(i) + t_k^2(i)}{2} \vert \mbf{Q}(i)} + \sum_{k \in \mc{U}} Q_k(i) A_k(i) + \mathbb{E}\set{\sum_{k \in \mc{U}} Q_k(i) t_k(i)  \vert \mbf{Q}(i)}, \IEEEyessubnumber
\end{IEEEeqnarray}
where the second term on the r.h.s is due to i.i.d assumption on the arrivals. We can bound the first term on the r.h.s by a constant \me{B} for all possible control actions taken for a given channel condition \cite{neely2010stochastic}, the resulting problem can be given as
\begin{equation}
\underset{t_k}{\text{maximize}} \quad \sum_{k \in \mc{U}} \, Q_k(i) \, t_k(i),
\label{backpreassure}
\end{equation}
where the expectation operator is dropped since it is maximized by maximizing the function inside in an opportunistic manner.

Now the problem in \eqref{backpreassure} looks similar to the \ac{WSRM} problem if the weights are replaced by the number of backlogged packets corresponding to the users. The above discussed approach is extended for the wireless networks in \cite{weeraddana2011resource}, where the queue weighted sum rate maximization is considered as the objective function to determine the transmit precoders. In order to avoid the excessive  allocation of the resources, we include an additional rate constraint \me{t_k \leq Q_k} to address \me{[x]^+} operation in \eqref{eqn-2a}. With this, the \ac{Q-WSRM} problem for a wireless system is given by
\begin{IEEEeqnarray}{rCl} \label{q_gen_sum}
\underset{\substack{\mvec{M}{k,n},\gamma_{l,k,n} \\ t_{l,k,n},\mvec{W}{k,n}}}{\text{maximize}} &\quad& \sum_{k \in \mc{U}} \, Q_k \left ( \sum_{n = 1}^N \sum_{l=1}^L t_{l,k,n} \right ) \IEEEyessubnumber \label{eqn-3.1.1} \\
\text{subject to.} & \quad & t_{l,k,n} \leq \log_2(1+\gamma_{l,k,n}) \IEEEyessubnumber \label{eqn-3.1.2} \\
& \quad &\sum_{n = 1}^N \sum_{k \in \mathcal{U}_b} \text{tr} \, (\mvec{M}{k,n} \mvec{M}{k,n}^\herm) \leq P_{{\max}}, \fall b, \IEEEyessubnumber \label{eqn-3.1.3} \\
& \quad & \sum_{n=1}^N \sum_{l = 1}^L t_{l,k,n} \leq Q_k, \fall k \in \mc{U} \IEEEyessubnumber \label{eqn-3.1.4}
\end{IEEEeqnarray}
where the precoders are associated with the \me{\gamma_{l,k,n}} defined in \eqref{eq:SINR}. By using the number of queued packets as the weights, the resources can be allocated to the user with the more number of backlogged packets, which essentially does the allocation in a greedy manner. As a special case of the problem defined in \eqref{q_gen_sum}, we can formulate the sum rate maximization problem by setting the weights in \eqref{eqn-3.1.1} as unity, leading to the problem as
\begin{IEEEeqnarray}{rCl}\label{gen_sum}
\underset{\substack{\mvec{M}{k,n},\gamma_{l,k,n} \\ t_{l,k,n},\mvec{W}{k,n}}}{\text{maximize}} &\quad& \sum_{k \in \mc{U}} \, \sum_{n = 1}^N \sum_{l=1}^L t_{l,k,n} \IEEEyessubnumber \\
\text{subject to.} & \quad & \eqref{eqn-3.1.1}, \eqref{eqn-3.1.2}, \eqref{eqn-3.1.3} \; \text{and} \; \eqref{eqn-3.1.4}. \IEEEyessubnumber
\end{IEEEeqnarray}

The problem defined in \eqref{gen_sum} provides an efficient queue minimizing approach as compared to \eqref{q_gen_sum}, since the resource allocation is driven by the channel conditions as compared to the number of queued packets as in \eqref{q_gen_sum}. It is to be noted that in both formulations, the resources allocated to the users are limited by the backlogged packets \eqref{eqn-3.1.4}, thereby handling the problem of over allocation explicitly with an additional rate constraint.
