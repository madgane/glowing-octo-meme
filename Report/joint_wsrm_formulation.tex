
The \ac{WSRM} technique is predominantly used to design the precoders for the \ac{MIMO} \ac{BC} channels to maximize the weighted sum throughput in the downlink direction. The weights are used to provide different priorities to the users based on their requirements and \ac{QoS} criterion. Since the objective of this work is mainly concerned with the reduction of the number of queued packets, we use the length of the queued packets for each user as the respective weights. Using the length of the queued packets as the corresponding weights is obtained from the well known algorithm, called the \textit{backpreassure algorithm}, which is used to decide the routing decisions for each node in the network to efficiently send the packets to the desired destination with minimal packet drops and delay. The backpreassure algorithm is the outcome of minimizing the Lyapunov drift conditioned over the current queue states \me{\mbf{Q}(i - 1)}, where \me{\mbf{Q}} is the vector formed by stacking the number of queued packets of each user in the system. Assuming the queue length grows according to \eqref{eqn-2a} for all the users, the Lyapunov drift is given by \cite{georgiadis2006resource,neely2010stochastic}
\begin{equation}
L(i) = \frac{1}{2} \, \| \mbf{Q}(i) \|^2_2 = \frac{1}{2} \, \sum_{k \in \mc{U}} Q_k(i)^2.
\end{equation}
The conditional Lyapunov drift is given by taking expectation over all channel realizations as
\begin{equation}
\Delta(i) = \mathrm{E}_H \left [ L(i) - L(i - 1) | \mbf{Q}(i - 1) \right ],
\label{eqn-3.1.0}
\end{equation}
since the transmission depends on the channel condition. Now, by substituting \eqref{eqn-2a} in \eqref{eqn-3.1.0}, the resulting expression can be upper bounded by
\begin{equation}
\Delta(i)\leq \sum_{k \in \mc{U}} \, \mathrm{E}_H \left [ \mbf{Q}(i-1) \left ( \mbf{\lambda}(i) - \mbf{t}(i) \right ) | \mbf{Q}(i-1) \right ] + \mathrm{E}_H \left [ \mbf{\lambda}^2(i-1) \right ] + \mathrm{E} \left [ \mbf{t}^2(i-1) \right ],
\end{equation}
where \me{\mbf{\lambda}} and \me{\mbf{t}} are the stacked arrivals and transmissions of all users. Since the expectation is conditioned over \me{\mbf{Q}(i-1)} and the second order statistics of the arrivals and the transmissions are bounded \cite{neely2010stochastic}, the drift minimization solution is given as
\begin{equation}
\underset{\mbf{t}(i)}{\text{minimize}} \sum_{k \in \mc{U}} \, Q_k(i-1) \: t_k(i).
\label{backpreassure}
\end{equation}

Now from \eqref{backpreassure}, the optimization variables \me{\mbf{t}(i)}, which is the vectorized transmission rates, are to be identified for each user for the current slot \me{i}. Since the current system can exploit both space and frequency resources through \ac{MIMO} \ac{OFDM} model, the transmission rates \me{\mbf{t}(i)} for the current instant can be controlled using the transmit precoders \me{\mbf{M}_{k,n}} designed for each user over each sub-channel resource. Since the decision variables \me{\mbf{t}(i)} depends on the queue state at \me{\mbf{Q}(i - 1)}, we drop the time variable \me{i} for the easy representation. The above problem defined in \eqref{backpreassure} can be rewritten to include the space-frequency allocation variable, namely \me{\mbf{M}_{k,n}}, as
\begin{IEEEeqnarray}{rCl} \label{q_gen_sum}
\underset{\substack{\mvec{M}{k,n},\gamma_{l,k,n} \\ t_{l,k,n}}}{\text{maximize}} &\quad& \sum_{k \in \mc{U}} \, Q_k \left ( \sum_{n = 1}^N \sum_{l=1}^L t_{l,k,n} \right ) \IEEEyessubnumber \label{eqn-3.1.1} \\
\text{subject to.} & \quad & t_{l,k,n} \leq \log_2(1+\gamma_{l,k,n}) \IEEEyessubnumber \label{eqn-3.1.2} \\
& \quad &\sum_{n = 1}^N \sum_{k \in \mathcal{U}_b} \text{tr} \, (\mvec{M}{k,n} \mvec{M}{k,n}^\herm) \leq P_{{\max}}, \fall b, \IEEEyessubnumber \label{eqn-3.1.3} \\
& \quad & \sum_{n=1}^N \sum_{l = 1}^L t_{l,k,n} \leq Q_k, \fall k \in \mc{U} \IEEEyessubnumber \label{eqn-3.1.4}
\end{IEEEeqnarray}
where \eqref{eqn-3.1.4} is due to the \me{[x]^+} operation in \eqref{eqn-2a} and \me{\gamma_{l,k,n}} is defined in \eqref{eq:SINR}. By using the number of queued packets as the weights, the resources can be allocated to the user with the more number of backlogged packets, which essentially does the allocation in a greedy manner. As a special case of the problem defined in \eqref{q_gen_sum}, we can formulate the sum rate maximization problem by setting the weights in \eqref{eqn-3.1.1} as unity, leading to the problem as
\begin{IEEEeqnarray}{rCl}\label{gen_sum}
\underset{\substack{\mvec{M}{k,n},\gamma_{l,k,n} \\ t_{l,k,n}}}{\text{maximize}} &\quad& \sum_{k \in \mc{U}} \, \sum_{n = 1}^N \sum_{l=1}^L t_{l,k,n} \IEEEyessubnumber \\
\text{subject to.} & \quad & \eqref{eqn-3.1.1}, \eqref{eqn-3.1.2}, \eqref{eqn-3.1.3} \; \text{and} \; \eqref{eqn-3.1.4}. \IEEEyessubnumber
\end{IEEEeqnarray}

The problem defined in \eqref{gen_sum} provides an efficient queue minimizing approach as compared to \eqref{q_gen_sum}, since the resource allocation is driven by the channel conditions as compared to the number of queued packets in the problem defined in \eqref{q_gen_sum}. In both formulations, the resources allocated to the users are limited by the backlogged packets \eqref{eqn-3.1.4}, thereby handling the problem of over allocation. The resource allocation constraint in \eqref{eqn-3.1.4} renders the problem difficult to be solved by the iterative algorithm to design the precoders, which will be discussed later.
