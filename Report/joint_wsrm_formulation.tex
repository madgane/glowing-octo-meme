
The queue minimizing algorithms are discussed extensively in the networking literature to provide congestion-free routing between any two nodes in the network.\footnote{routers or user terminals} One such algorithm  is the \emph{backpressure algorithm}, discussed in detail in \cite{tassiulas,georgiadis2006resource,neely2010stochastic}. The algorithm determines an optimal control policy in the form of rate or resource allocation for the nodes in the network by considering the differential backlogged packets between the source and the destination nodes. Even though the algorithm is primarily designed for the wired infrastructure, it can be extended to the wireless networks by designing the user rate variable \me{t_k} in accordance to the wireless network.

The \ac{Q-WSRM} formulation extends the \emph{backpressure algorithm} to the \ac{MIMO}-\ac{OFDM} framework, in which the multiple \acp{BS} acts as the source nodes and the user terminals as the receiver nodes. The control policy in the form of transmit precoders are designed to minimize the number of queued packets waiting at the \acp{BS}. In order to find the optimal algorithm, we use the Lyapunov function which is predominantly used in the control theory for the system stability. Since at each time slot, the system can be described by the channel conditions and the number of backlogged packets of each user, Lyapunov function is used to provide a scalar measure, which grows large when the system moves towards the undesirable state. Following the approach in \cite{neely2010stochastic}, the scalar measure for the queue stability is given by
\begin{equation}
\mathrm{L}\sset{\mbf{Q}(i)} = \frac{1}{2} \, \sum_{k \in \mc{U}} Q_k(i)
\end{equation}
where \me{\mbf{Q}(i)} denotes the stacked user queues at the \me{\ith{i}} slot and \me{\frac{1}{2}} is used for the convenience. The Lyapunov function provides a measure of congestion in the system, as discussed in \cite[Ch. 3]{neely2010stochastic}. Now the Lyapunov function drift is given by
\begin{IEEEeqnarray}{rCl}\label{eqn-3.1}
\mathrm{L}\sset{\mbf{Q}(i+1)} - \mathrm{L}\sset{\mbf{Q}(i)} &=& \frac{1}{2} \left [ \sum_{k \in \mc{U}} \, \Big ( \left [ Q_k(i) - t_k(i) \right ]^+ + \lambda_k(i) \Big )^2 - Q^2_k(i) \right ] \IEEEyessubnumber \label{eqn-3.1.0} \\
&\leq& \sum_{k \in \mc{U}} \, \frac{\lambda^2_k(i) + t_k^2(i)}{2} + \sum_{k \in \mc{U}} Q_k(i) \set{ \lambda_k(i) - t_k(i) } \IEEEyessubnumber \label{drift-exp}
\end{IEEEeqnarray}
where the inequality is due to the upper bound
\begin{equation}
\sset{\max(Q-t,0) + \lambda)}^2 \leq Q^2 + t^2 + \lambda^2 + 2 Q (\lambda - t)
\end{equation}
In order to minimize the squared sum at each instant, minimization of the Lyapunov drift \eqref{eqn-3.1} is carried over all possible rate allocations in the form of transmission rates \me{t_k} to users in the system. The Lyapunov drift conditioned on the current backlogged packets \me{\mbfa{Q}(i)} is given by
\begin{IEEEeqnarray}{rCl} \label{backpressure-approx}
\underset{\mbf{t}}{\text{minimize}} && \Delta(\mbf{Q}(i)) \triangleq \mathbb{E}_{\mbfa{\lambda},\mbf{t}} \set{\mathrm{L}\sset{\mbf{Q}(i+1)} - \mathrm{L}\sset{\mbf{Q}(i)} \vert \mbf{Q}(i)} \IEEEyessubnumber \\
 &\leq& \mathbb{E}_{\mbfa{\lambda},\mbf{t}} \set {\sum_{k \in \mc{U}} \, \frac{\lambda^2_k(i) + t_k^2(i)}{2} \vert \mbf{Q}(i)} + \sum_{k \in \mc{U}} Q_k(i) A_k(i)  \mathbb{E}_{\mbfa{\lambda},\mbf{t}}\set{\sum_{k \in \mc{U}} Q_k(i) t_k(i)  \vert \mbf{Q}(i)} \IEEEyessubnumber
\end{IEEEeqnarray}
where the second term on the r.h.s follows from the Poisson arrival process. Assuming the second order moments of the transmission rates and the arrival rates are bounded, it can be replaced by a suitable bound \me{B} in order to eliminate from the optimization problem \cite{neely2010stochastic}.

Now the expression in \eqref{backpressure-approx} looks similar to the \ac{WSRM} formulation if the weights in the \ac{WSRM} problem are replaced by the number of backlogged packets corresponding to the users. The above discussed approach is extended for the wireless networks in \cite{weeraddana2011resource}, where the queue weighted sum rate maximization is considered as the objective function to determine the transmit precoders. The \ac{Q-WSRM} formulation is given by
\begin{IEEEeqnarray}{rCl} \label{q_gen_sum}
	\underset{\substack{\mvec{M}{k,n}, \mvec{W}{k,n}}}{\text{maximize}} &\quad& \sum_{k \in \mc{U}} \, Q_k \left ( \sum_{n = 1}^N \sum_{l=1}^L \log_2(1 + \gamma_{l,k,n}) \right ) \IEEEyessubnumber \label{eqn-3.1.1} \\
	\text{subject to.} & \quad & \sum_{n = 1}^N \sum_{k \in \mathcal{U}_b} \text{tr} \, (\mvec{M}{k,n} \mvec{M}{k,n}^\herm) \leq P_{{\max}}, \fall b \IEEEyessubnumber \label{eqn-3.1.3}
\end{IEEEeqnarray}
In order to avoid the excessive  allocation of the resources, we include an additional rate constraint \me{t_k \leq Q_k} to address \me{[x]^+} operation in \eqref{eqn-2a}. The extended version of the \ac{Q-WSRM}, denoted by \ac{Q-WSRME} problem for a cellular system is given by \label{q_gen_sum-1} with the additional constraint
\begin{IEEEeqnarray}{rCl} \label{eqn-3.1.4}
\sum_{n=1}^N \sum_{l = 1}^L \log_2(1 + \gamma_{l,k,n}) \leq Q_k, \fall k \in \mc{U}
\end{IEEEeqnarray}
where the precoders are associated with the \me{\gamma_{l,k,n}} defined in \eqref{eq:SINR}. By using the number of queued packets as the weights, the resources can be allocated to the user with the more number of backlogged packets, which essentially does the allocation in a greedy manner.

As a special case of the problem defined in \eqref{q_gen_sum}, we can formulate the sum rate maximization problem by setting the weights in \eqref{eqn-3.1.1} as unity, leading to the problem as
\begin{IEEEeqnarray}{rCl}\label{gen_sum}
\underset{\substack{\mvec{M}{k,n},\mvec{W}{k,n}}}{\text{maximize}} &\quad& \sum_{k \in \mc{U}} \, \sum_{n = 1}^N \sum_{l=1}^L \log_2(1 + \gamma_{l,k,n}) \IEEEyessubnumber \\
\text{subject to.} & \quad & \eqref{eqn-3.1.3} \; \text{and} \; \eqref{eqn-3.1.4} \IEEEyessubnumber
\end{IEEEeqnarray}
The problem defined in \eqref{gen_sum} provides a greedy queue minimizing approach as compared to \ac{Q-WSRME}, since the resource allocation is driven by the channel conditions in comparison with the number of queued packets as in \ac{Q-WSRME}. Note that in both formulations, the resources allocated to the users are limited by the backlogged packets with an explicit maximum rate constraint defined by \eqref{eqn-3.1.4}.
