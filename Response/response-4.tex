This paper proposes a practical method for minimizing the number of currently backlogged packets in a wireless multi-cell MIMO-OFDM network. Resource allocation is performed over space (beamformers) and frequency (sub-carriers), and a norm of the queue deviations is minimized. The problem studied is very relevant, but still most work in the literature has so far focused on the infinite queue model which clearly does not reflect reality particularly well. The proposed methods seem practical, due to their possibility to be implemented as distributed methods coupled with distributed CSI acquisition. The paper has a multitude of approaches to the problem, but there are several areas which must be improved before a possible publication.

\vspace{0.1in}
\review{ First, we thank the reviewer for providing valuable and insightful comments. The comments are really helpful in improving the manuscript.}
\begin{enumerate}
\cmnt{1} {First, this reviewer is not convinced by the arguments for showing the convergence of the JSFRA method. "The SCA method" is often referred to, but never really defined or referenced. The three required conditions (as stated on p. 6, col. 1, rows 38-40) do not, as far as I can tell, appear in [27]. Indeed, [27] is concerned with optimization problems where the objective function is non-convex, but the constraint set is convex and separable over the blocks of variables. Perhaps you meant to cite [A], wherein non-convex constraints are handled in a similar way? Numerically, the algorithms do converge, and the argument put forward makes sense, but the treatment must be improved to be more rigorous.}

\resp \review{We agree that the original manuscript lacks the convergence proof for the JSFRA approach. We have presented a rigorous convergence proof for the proposed algorithm in Appendix B on the supplementary document. We have modified to include the reference suggested by the reviewer in [32] instead of [27], which is for the nonconvex objective function with the convex constraints.}

\cmnt{2} {Second, the optimization problems formulated only depend on $Q_k$, the current levels of backlogged packets, and not on the arrival rates. This is due to how the conditional Lyapunov drift is minimized. This approach completely removes the queue dynamics from the optimization problem, essentially leading to greedy one-shot optimization in every time instant. The framework would be more interesting if some sort of optimization (or tracking) is performed over several time instants, rather than the one-shot approach that is currently used for the JSFRA algorithms. Possibly, some expectation over the queues would be optimized then. Even if no analytical treatment of the tracking over several time-steps is added, I would at least highly recommend adding some simulation results where the proposed one-shot algorithms are performed sequentially over several time instants.}

\resp \review{
	\begin{enumerate}
	\item We thank the reviewer for the suggestion. In this paper, we proposed the precoder design to minimize the number of backlogged packets in the system at a given time. Since 
	the objective is minimized in each time instant, it minimizes the objective on an average as well.
	\item We focused on the design of the precoders based on the current knowledge of the channel state information in the system model. 
	\item As suggested by the reviewer, in order to justify the gains over multiple time instants, we have now included Section V-C with Fig. 4 to compare the performance of the JSFRA scheme for different \me{\ell_q} norm over multiple time slots. Plots comparing the average number of backlogged packets for different schemes at each instant and the number of queued packets remained in the system after each transmission instant are provided in Fig. 4.
	\end{enumerate}
}

\cmnt{3} {Third, the distributed methods (at least the primal decomposition and ADMM) seem to be fairly straight-forward applications of existing results. This reviewer recommends spending more space on the convergence, than on the description of the distributed techniques. Still, it would be nice with a direct description of what local CSI is required, and how it is acquired, to perform the local computations for the primal decomposition and ADMM methods. For the description of the signaling of the CSI in Sec. IV-B, are you envisioning a TDD system?}

\resp \review{
	\begin{enumerate}
		\item In accordance with the reviewer, we have shortened the discussions on the distributed approaches and discussed the convergence proof of the decentralized algorithm in Appendix C. 
		\item For the distributed precoder design, we assume that each \ac{BS} \me{b} knows the equivalent downlink channel \me{\mvec{w}{l,k,n}^\herm \mbf{H}_{b,k,n}} of all users in the system by using precoded uplink pilots, where the precoders are the MMSE receiver of all the users. Note that it includes the cross equivalent downlink channels of the neighbor BS users as well. To update the MMSE receiver, the equivalent channel for the \me{\ith{k}} user \me{\mbf{H}_{b,k,n} \mvec{m}{l,k^\prime,n}, \forall k^\prime \in \mc{U}_b, \forall b \in \mc{B}} is obtained from the BSs through user specific downlink precoded pilots. We have included the information on what each network entity knows in the paragraph after (36).
		\item We have included the type of duplexing scheme adopted in the model, \textit{i.e}, TDD system, in the system description in Section II, last paragraph.
		\end{enumerate}
	}

\cmnt{4} {Finally, some readers might be confused by the "joint space-frequency" terminology, believing that the beamforming is performed over a joint space-frequency channel space, where the space-frequency channels are formed by block-diagonal matrices, each block belonging to one sub-carrier. This could easily be clarified.} 

\resp \review{As per the reviewer's suggestion, we have clarified the space-frequency terminology in the lines before the last paragraph in Section I.}

\cmnt{5} {Please see the itemized questions}

\begin{enumerate}

\cmnts{a} - {p. 1, col. 1, row 42: "userss"}

\resp \review{Corrected in the revised manuscript in Section I first paragraph.}

\cmnts{b} - {p. 1, col. 2, row 18: the precoders are used \_implicitly\_ as decision variables. This is the whole point, to avoid explicitly modeling the hard decisions in the optimization, and instead do soft decisions during the iterations, and then finally hard decisions after convergence.}

\resp \review{We thank the reviewer for highlighting the insightful point. We have included the statement in the revised manuscript in Section I, second paragraph.}

\cmnts{c} - {p. 1, col. 2, row 33: Which chapter in [2] is referred to? With a quick look-through of the table of contents, I can't find and chapter or section treating the SCA method?}

\resp \review{We apologize for the inappropriate reference. We have updated the manuscript to include the proper reference on SCA with [22,32,34].}

\cmnts{d} - {p. 2, col. 2, row 36: Write $\text{rank}(.)$ and $\min$ instead}

\resp \review{We have updated the revised manuscript with the reviewer suggestions on the second paragraph in Section II.}

\cmnts{e} - {p. 3, col. 1, row 26: It would be more clear to explicitly write out the dependence of $\mathbf{M}$ and $\mathbf{W}$ in $\tilde{v}$ here}

\resp \review{We removed the matrix representation of the transmit precoders and the receive beamformers from the manuscript to avoid the confusion as suggested by the reviewer.}

\cmnts{f} - {p. 3, col. 2, row 26: Which general MIMO-OFDM problem are you talking about here, and what is combinatorial about it? Is it the problem of selecting users to be served on orthogonal subcarriers? There is nothing inherently combinatorial over the problem in (6) as far as I can tell, as the beamformers are used as soft decision variables.}

\resp \review{We thank the reviewer for pointing out the confusion. We have removed the word "combinatorial" from the text on the initial paragraph in Section III.}

\cmnts{g} - {p. 4, col. 2, row 40: "In fact, (5) provides similar expression of ..." This sentence is very hard to understand.}

\resp \review{We have rephrased the sentence to avoid difficulty in the understanding. The manuscript is updated with the revised text on the first paragraph in Section III.B.}

\cmnts{h} - {(16d): suggest your write out the power constraints here, in order to be faster be able to interpret the optimization problem. There is hardly any spaced saved by referring back to (6b).}

\resp \review{We have updated the manuscript with the explicit power constraint in (16d) as
\begin{equation}
\sum_{n = 1}^N \sum_{k \in \mathcal{U}_b} \sum_{l=1}^L \trace \, (\mvec{m}{l,k,n} \mvec{m}{l,k,n}^\herm) \leq P_{{\max}}, \fall b
\end{equation}}

\cmnts{i} - {p. 5, col. 1, rows 27-30: Here you might want to quickly mentioned how one could show the NP-hardness of (16).}

\resp \review{We have included the NP-hardness of the proposed solution by modeling the current problem to solve the \ac{WSRM} formulation, which is known to be NP-hard on the last sentence in Section III-B third paragraph.} 

\cmnts{j} - {p. 5, col. 1, row 50: "According to the SCA method...". I am not sure exactly how you define "\_the\_ SCA method"? Clarify or cite the definition.}

\resp \review{We have removed the sentence mentioning the SCA method. We have updated the manuscript to discuss this as SCA approach in the sentences after (17).}

\cmnts{k} - {p. 5, col. 2, row 31: Here is a case where it makes sense to reference earlier optimization constraints. However, are (19d) and (18) not the same??}

\resp \review{We thank the reviewer for pointing out the mistake. We have included all the constraints except the linearization constraint in (20), which is (19) in the original manuscript. Expression (19d) and (18) are the same. We have removed this double reference from (20).}

\cmnts{l} - {p. 5, col. 2, row 51: Slightly confusing with the notation between the iterates in (21b) and the MMSE filter in (22b).}

\resp \review{We have updated the manuscript to avoid the ambiguity in the representation of the optimal and the MMSE receiver. The optimal receiver is denoted by \me{\mvec{w}{l,k,n}^{o}} in (22) and the MMSE receiver by \me{\mvec{w}{l,k,n}} in (23).}

\cmnts{m} - {p. 6, col. 1, row 9: You might want to add somewhere that (22b) can be used instead of the fixed-point of (21b), since the scaling of the receive filters do not matter in the SINRs. However, does it affect the convergence of the algorithm?}

\resp \review{We thank the reviewer for the comment. We have updated the manuscript to include this discussion in the sentences after (23). The performance remains the same and the convergence rate is not affected by this scaling, which can be seen from Fig. 1 comparing the precoder convergence using the optimal and the MMSE receiver for the JSFRA scheme with \me{\ell_1} norm.}

\cmnts{n} - {p. 6, col. 2, rows 8-10: I don't fully understand the reasoning on the relation between the constraint sets in the different iterations. Why is this the case?}

\resp \review{To solve the nonconvex problem (16), we linearize the DC constraint (16b) around a fixed operating point. Since the operating point happens to be the optimal solution from the earlier iteration, the optimal point of the previous iterations is also present in the current feasible set. Therefore, at each iteration, the algorithm finds a better solution or the same compared to the previous solution, which leads to the monotonically decreasing objective. Please refer to Appendix B in the supplementary document for more details on the convergence analysis of the centralized algorithm.}

\cmnts{o} - {p. 7, col. 1, row 35: Just because a problem is convex does not mean that it has a unique solution. (Although it seems to me that (26) should have a unique solution.) Is the problem in (26) strictly convex?}

\resp \review{
	\begin{enumerate}
		\item We thank the reviewer for the critical comment and the suggestion. It is true that the convex problem need not have a unique solution in general. Using Minkowski inequality \me{\|x + y\|_q \leq \|x\|_q + \|y\|_q }, we can see that the JSFRA problem is not strictly convex for all exponents \me{\ell_q} used in the objective function. 
		\item The uniqueness of the beamformers for the MSE reformulation in (28) is guaranteed due to the MSE constraint (26c), which is susceptible to the transmit beamformer phase rotations.
		\item On the other hand, the uniqueness of the beamformers for the SCA approach in (20) cannot be guaranteed unless the imaginary part of \me{q_{l,k,n} = \Im \{ \mvec{w}{l,k,n} \mvec{H}{b_k,k,n} \mvec{m}{l,k,n} = 0\}. It can be achieved by using the MMSE receiver for a randomly chosen initial transmit beamformer. 
		\item The solution is not unique when the objective is zero even for a single BS. In this case, we can regularize the objective with a strictly convex term as
		\begin{equation}
		\|\tilde{\mbf{v}} \|_q + c \, \|\mbf{x} - \mbf{x}^{(i)}\|^2
		\end{equation}
		where \me{\mbf{x}} denotes the stacked vector of optimization variables and \me{\mbf{x}^{(i)}} denotes the value of \me{\mbf{x}} in the \me{\ith{i}} iteration. It is discussed briefly in the Appendix B-D. It makes the objective strictly convex, therefore, it has a unique minimum.
	\end{enumerate}
		}

\cmnts{p} - {Table 1: "backpreassure"}

\resp \review{It is updated in the manuscript in Table I.}

\cmnts{q} - {p. 11, col. 1, row 56: "performances". I'm not sure this is a countable noun.}

\resp \review{We thank the reviewer for pointing out the gramatical errors. We checked the gramatical validity and kept the plural in some cases.}

\end{enumerate}

\end{enumerate}