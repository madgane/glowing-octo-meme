In this paper, the authors proposed a traffic aware resource allocation scheme for multi-cell MIMO-OFDM systems, where the precoders at all BSs are chosen to minimize the total user queue deviations. The problem is nonconvex and the authors proposed two centralized algorithms based on the successive approximation (SCA) technique to find a stationary point. Moreover, several distributed algorithms are also proposed using primal decomposition, alternating directions method of multipliers (ADMM), and decomposition via KKT conditions, respectively.

Most sections of this paper are well written. The results and algorithms also seem valid. However, the motivation of minimizing the total user queue deviations is not well justified. The convergence results of some algorithms are not clearly presented. The presentation of the distributed solutions needs significant improvement. Analysis and comparison of the signaling overhead and computational complexity between the centralized and distributed algorithms are also necessary to justify the advantages of distributed algorithms.

\review{We thank the reviewer for providing valuable and insightful comments.}

\begin{itemize}
	
\cmnt{1} In Section II.B, please provides more justifications for the problem formulation in (6). For example, the Queue weighted sum rate maximization (Q-WSRM) is throughput optimal, i.e., if there exists a scheme which can make all queues stable, then the Q-WSRM can also do this. How about the proposed formulation in (6)? Is it also throughput optimal?

\resp \review{We thank the reviewer for the comment. The Q-WSRM scheme and the proposed schemes are all throughput optimal. It can be seen that the proposed extension Q-WSRME and the JSFRA formulations aim at minimizing the number of backlogged packets in addition to avoiding the over allocation of available resources. The JSFRA formulation using \me{\ell_1} norm as the objective minimizes the number of backlogged packets in a greedy manner at each instant. By increasing the exponent \me{q \rightarrow \infty}, we obtain fair allocation at every transmission instant. We have included the discussions on the average number of backlogged packets at each instant for different arrival rates in Section V-C on page 13, col. 1. We have provided justifications for the problem formulation in (6) and (16) in page 5, col. 1, first paragraph. The equivalence between the Q-WSRM scheme and the \me{\ell_2} norm objective in the JSFRA formulation can be seen easily when the number queue size increases. We have clarified this point in the revised manuscript. Please refer to the first paragraph in Section III-B.}

\cmnt{2} Do the proposed solutions based on (6) achieve better average delay performance than the existing solutions? By the way, in the simulations, you should also add a figure comparing the average delay performance, instead of just comparing the performance metric defined by (6). This will better justify the advantage of the proposed solutions.

\resp \review{The proposed algorithm is a function of the number of backlogged packets and the channel conditions of all users in the network. Since the delay is proportional to the average number of backlogged packets in system, the proposed algorithm performs equally good in comparison with the existing Q-WSRM approach. It can be easily verified by the equivalence between the \me{\ell_2} norm JSFRA and the Q-WSRM scheme on page 4, col. 2, Section III-B starting paragraph. We have also provided the performance of different JSFRA algorithm and Q-WSRM scheme based on the average number of backlogged packets in the system over multiple transmission slots by varying the average number arrivals for all users in the system in Fig. 4 on page 13. We can still control the delay behavior by controlling the priority factor \me{a_k} in (6a), which affects the resource allocation. For example, if we consider a VOIP transmission, the delay requirements are stringent and it can be achieved by increasing the priority factor \me{a_k} in the current formulation. In addition, we can also control the delay by changing the exponent used in the objective \me{\ell_q} of the JSFRA problem. We considered the residual number of backlogged packets as a performance measure, since we assume only the instantaneous channel state informations and together with the number of backlogged packets, resources can be allocated only for a given instant.}

\cmnt{3} In Section III.B, the convergence conditions under Algorithm 1 are not clear. First, you should be more specific about what is the SCA subproblem. Do you mean problem (19)? Second, does the uniqueness of the transmit and receive beamformers mean that the solution of the original problem in (16) is unique, or the solutions of the subproblems in (19) and (20) are unique, respectively?

\resp \review{We understand the reviewer's point. We have updated the revised manuscript to discuss the convergence of the centralized algorithm in detail in Appendix B on page 14, col. 1. The discussions are provided for the convergence of the iterative algorithm. The uniqueness of the convex subproblem (20) can be guaranteed when the objective is non-zero for all coordinating BSs in the precoder design. It is due to the linear constraint on the transmit precoders in (19), which is susceptible to the transmit precoder phase rotations. On the other hand, the constraint in (16b) is not, since the precoder variable is inside the absolute value operator. Once the algorithm finds a unique set of transmit precoders, all unitary rotations are also valid for the original problem in (16). The uniqueness of the transmit and the receive beamformers are discussed in detail in Appendix B-D on page 15, col. 2.}

\cmnt{4} It is better to clearly summarize the convergence conditions and results (i.e., does it converge to a stationary point or the optimal solution) for all algorithms in a theorem/proposition.

\resp \review{The discussions on the convergence of the centralized algorithms are provided in Appendix B on page 14, col. 1 and for the distributed algorithms in Section IV-C on page 8, col. 2.}

\cmnt{5} At the end of Section III, you mentioned that the proposed reduced complexity resource allocation scheme is sensitive to the order in which the subchannels are selected for the optimization problem. Please provide a discussion how to choose this order.

\resp \review{We understand the reviewer's concern. Since the algorithm designs the precoders for each sub-channel at a time by using the total number of unserviced packets. For designing the precoders for the sub-channel \me{j+1}, we assume the transmit precoders and the rates of all users are obtained for the sub-channels \me{\{1,2,\dotsc,j\}}. Now, the number of backlogged packets for the \me{\ith{j+1}} sub-channel is given by 
\[Q_k - \sum_{i=1}^j \sum_{l=1}^L t_{l,k,i}\].
Since it depends on the rates of the already completed sub-channels \me{\{1,2,\dotsc,j\}}, it is susceptible to the ordering used to determine the precoders in each sub-channels. It is discussed in detail in Section III-D on page 7, col. 1.}

\cmnt{6} In the distributed algorithms, it is not clear what exact information is exchanged between the BSs or between the BSs and users. Moreover, the signaling overhead should be analyzed and compared with the centralized solution. The proposed distributed algorithms require exchanging over-the-air signaling or backhaul signaling for many times within each channel coherent time (e.g., from Fig. 2, the distributed algorithm requires 20-30 iterations to converge even when there are only 3 subchannels). I don’t think this is acceptable in practice. Is the signaling overhead of the distributed algorithm really smaller than the centralized algorithm which only requires exchange the CSI between the BSs for once within each channel coherent time?

\resp \review{The distributed algorithms are derived for the convex subproblem, which leads to the same stationary point asymptotically as that of the centralized solution, but indeed we would require large number of iterations for the convergence. In reality, we have to limit the number of iterations required for each distributed algorithm, thereby leading to a point which may not be the same stationary point as when the algorithm is allowed to converge. The number of ADMM or primal updates are controlled by the parameter \me{J_{\max}} in the Algorithm 2. In practice, we can combine the ADMM or primal update, receiver update and the SCA update all at once, \textit{i.e}, \me{J_{\max} = 1} to minimize the overhead in the backhaul signaling and to speed up the convergence. To this end, we have provided a practical scheme in Section IV-D on page 9, col. 1. Note that in the time-correlated case, it is often also enough to update the precoders once per radio frame. The decentralized schemes are never allowed to converge until the end, it is only important to follow the fading process when \me{J_{\max} = 1}. The performance of the distributed algorithm based on dual decomposition scheme is discussed for the time-correlated fading in Section C of [13] in the revised manuscript, which shows that it is enough for the distributed precoder design to follow the fading process to provide desired performance.}

\cmnt{7} The convergence analysis of the distributed algorithms is not clear. For example, what is the exact condition to ensure the convergence of the distributed algorithms. Does the distributed algorithms also converge to a stationary point?

\resp \review{We have updated the manuscript to include the discussions on the convergence of the distributed algorithm on page 8, col. 2 last paragraph in Section IV-C. The condition for the convergence of the distributed algorithm is that when the dual variable updates are not significant or when the decrease in the objective value is below some threshold. The distributed algorithm reaches the same stationary point as of the centralized algorithm if we let the inner loop in Algorithm 2 on page 8, col. 2 to converge.}

\cmnt{8} I’m totally confused with the ADMM approach in Section IV.A. Many notations, such as the local interference vector and consensus interference vector are used without formal definition. What is the difference between the local interference vector and consensus interference vector? What are their relationships with the actual interference vector. It seems that you are using the same notation for all of these interference vectors and I can’t tell when a notation refers to a local interference vector, a consensus interference vector, or the actual interference vector. These questions should be clarified and perhaps you should choose the notation system more carefully. For example, in (36), there are 3 similar notations and I don’t know which one is local interference vector and which one is the actual interference vector.

\resp \review{We understand the reviewer's concern. We have shortened the section on ADMM by citing the relevant earlier work on the distributed implementation of the centralized algorithm for the min power problem [29]. We have rewritten the distributed algorithms in Section IV-A and Section IV-B. The variables are discussed in Section IV on page 8, col. 1. It follows the existing literature on the ADMM scheme, which treats the interference as a variable at each BS and the consensus on the interference is achieved upon convergence of the distributed algorithm as discussed in [11].}

\cmnt{9} In the distributed algorithms, it is not clear what information is available at each node. For example, what are your assumption on CSIT (CSI knowledge at each BS) and CSIR (CSI knowledge at each user)? How to 2 obtain the information used to perform the required calculation at each node (such as calculating the actual interference, MMSE receiver and the dual variables)?

\resp \review{We understand the reviewer's concern. For the distributed precoder design, we assume that each \ac{BS} \me{b} knows the equivalent downlink channel \me{\mvec{w}{l,k,n}^\herm \mbf{H}_{b,k,n}} of all users in the system by using precoded uplink pilots, where the precoders are the MMSE receiver of all the users. Note that it includes the cross equivalent downlink channels of the neighbor BS users as well. To update the MMSE receiver, the equivalent channel for the \me{\ith{k}} user \me{\mbf{H}_{b,k,n} \mvec{m}{l,k^\prime,n}, \forall k^\prime \in \mc{U}_b, \forall b \in \mc{B}} is obtained from the BSs through user specific downlink precoded pilots. We have included the information on what each network entity knows in page 7, col. 2 last paragraph. We have included the type of duplexing scheme adopted in the model, \textit{i.e}, TDD system, in the system description in Section II, last paragraph. We have included the type of duplexing scheme adopted in the model, \textit{i.e}, TDD system, in the system description in Section II, last paragraph.}

\cmnt{10} Do you have any convergence result for the proposed distributed solution based on the KKT conditions in Section IV.B? It seems that the iterative method to solve the KKT conditions is totally heuristic.

\resp \review{It is true that the KKT conditions are based on the centralized algorithm in (20). It follows the same points as that of the centralized approach if the algorithm is iterated in the same order, \textit{i.e} the dual variables are allowed to converge before the update of the fixed operating point and the receiver. Here instead we perform the update of the transmit precoders, receive beamformers and the dual variables all at once in each iteration. Thus, we sacrifice the formal convergence for the improved speed of convergence. It may not be a stationary point of the original nonconvex problem but it is guaranteed to provide better performance in the sum rate compared to the distributed approaches presented in Section IV-A and Section IV-B for the same number of iterations. We have provided additional details on the convergence of the KKT based approach in page 10, col 1, last paragraph.}

\cmnt{11} Since queue is a dynamic system evolving according to (3), it doesn’t make sense to compare the queue deviations at a given time. You should compare average queue deviations in the simulations. Moreover, you should also compare the average delay performance instead of just comparing the performance metric (queue deviations) defined in this paper. Using the queue deviations as the performance metric also needs more justification.

\resp \review{We thank the reviewer for raising the critical comment. Since the manuscript is about the precoder design to minimize the total number of backlogged packets in the system, we have provided the convergence behavior of the proposed precoder design formulations for a given time instant. In accordance with the reviewer comment, we have included the Section V-C on page 13, col. 1 to discuss the queue deviation over multiple transmission slots. We have presented Fig. 4a by comparing the average number of backlogged packets for different algorithms with various arrival rates and Fig. 4b for the number of backlogged packets at each instant. Note that the question on delay arises when we are considering the resource allocation over certain duration. Since the paper is about the precoder design to minimize the number of backlogged packets at each instant, it may not be a valid performance metric for our objective. We still agree that delay can be controlled by reducing the number of packets on an average, as we can see from Fig. 4, on page 13, the proposed algorithms are equally good in minimizing the average number of backlogged packets. Note that we can also prioritize the users by controlling the variable \me{a_k} in (6a) to address the QoS constraints for a particular user, in addition to that, we can also change the objective to \me{\ell_2} and \me{\ell_{\infty}} norm to address the delay and the fairness implicitly. }

\cmnt{12} What is “SRA” in the simulation figures?

\resp \review{The \acf{SRA} is updated in the revised manuscript on page 7, col. 1, Section III-D.}

\cmnt{13} In the discussion for Fig. 1, you mentioned that JSFRA converge to the optimal point, and all algorithms are Pareto-optimal. Since the problem is non-convex, why these algorithm can find optimal solution or Paretooptimal point?

\resp \review{We thank the reviewer for pointing out the mistake in the text. Since the problem is nonconvex, the JSFRA formulation can find a local optimal point upon convergence. The converged point of the JSFRA problem is in fact the stationary point of the original nonconvex problem, which is discussed in Appendix B-E on page 15, col. 2. We have removed the statement mentioning the pareto-optimal solutions in the discussions on Fig. 1 in the revised manuscript.}

\end{itemize}
