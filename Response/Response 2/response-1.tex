Comments:
The response to the reviewer's concerns are generally satisfying, except the convergence proof.

\vspace{0.1in}
\review{We thank the reviewer for providing valuable and insightful comments.}

For the resubmitted manuscript, the reviewer still has the following concerns

\begin{enumerate}
\cmnt{1} Considering the length of the manuscript, it would be better to shorten some parts that are not new in this manuscript, e.g. III.A. More space can be left for convergence proof, which is very important. 

\resp \review{We thank the reviewer for the comment. 
		}

\cmnt{2} In convergence proof (48), why does the 2nd inequality hold? In fact, to prove the feasibility of  \eqn{{m_{k+1}^{(i)}, w_{*}^{(i-1)};m_k^{(i)} }}, the part between 2nd and 3rd inequality is not necessary, ''<=0'' directly follows the 2nd inequality since the solution \eqn{m_{k+1}^{(i)}, \gamma_{k+1}^{(i)} } is the optimal solution, and therefore  feasible.  

\resp \review{
	}

\cmnt{3} The solutions SCA iterations \eqn{m^{(i)}_k} does not necessarily converge. In fact \eqn{m} has compact feasible region, and thus \eqn{m^{(i)}_k} has limit points for any specific \eqn{i}.  However \eqn{m_{*}^{(i)}} does not necessarily exist( the whole sequence \eqn{m^{(i)}_k} may be not convergent). Similar problem happens to \eqn{w^{(i)}_k}. 

\resp \review {	
}

\cmnt{4} Strict monotonicity with respect to the objective function \eqn{f} should be rigorously proved. Note that to guarantee the uniqueness of the beamformer iterates, (52) instead of the objective function is used. 

\resp \review{}

\cmnt{5} Note that the conclusions [32, Thm 2] and [26, Thm 10] have lots of assumptions. To invoke these reference, explicit exposition should be provided to show that these conclusions can be applied to our problem. The same questions occur to the proof in Appendix B, where conclusions in [11] [36] and [37] are used. Too many details are omitted to make the proof convincing and clear. 

\resp \review{}

\end{enumerate}
