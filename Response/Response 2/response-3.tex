\review{The authors have introduced changes in the manuscript that improved the paper's quality. Additionally, the authors have taken into account the reviewers' comments giving clarifications and modifying the content when required. More specifically, the following aspects have been treated:}

\vspace{1eM}
\underline{\textit{Reply:}} We thank the reviewer for recognizing the changes made on our earlier manuscript. We also thank the reviewer for providing constructive comments to help us to improve our manuscript.

\begin{enumerate}
\cmnt{1} \review{Convexity of problem (16). The paragraphs surrounding (16) allow a better understanding of the usage of the additional variables, i.e. gamma and beta, to remove the equality constraint in (2). For the reviewer remains however unclear, the procedure/criterion to determine the operating point for the parameter \eqn{\tilde{\beta}} required in (19) and used in the convex subproblems (20) and (21).}

\resp We thank the reviewer for the comment. We have included additional information on how to choose the operating point while initializing the iterative algorithm. It is included in the fourth paragraph following (23). We have also provided a detailed discussion on how to update the operating point after each iteration in the paragraph following (23). Note that the objective function is convex, and therefore we can update operating point for the current step with the solution obtained in the previous iteration as
\begin{equation}
\tilde{\mbf{m}}_{j}^{(i)} = \mvec{m}{j}^{(i)}
\end{equation}
where \eqn{\tilde{\mbf{m}}_{j}^{(i)}} denotes the operating point for the \eqn{\ith{j+1}} \ac{SCA} iteration in the \eqn{\ith{i}} \ac{AO} step and \eqn{\mbf{m}_{j}^{(i)}} is the solution obtained by solving (20) in the \eqn{\ith{j}} \ac{SCA} iteration.

\cmnt{2} \review{Proof of convergence. The proof of convergence introduced by the authors in the Appendix seems correct and enhances the content of the manuscript.}

\resp We thank the reviewer for the comment. We have included several materials to make the proof of convergence more rigorous.

Additional Comments - 

\begin{enumerate}
	
	\cmnts{a} - \review{The reviewer considers that closing statements regarding the applicability of the proposed schemes are missing. Since the results are quite similar (when not identical), which formulation is preferable between the centralized schemes? Which one for the distributed solutions?}
	
	\resp We have updated the manuscript to include a discussion on the possible conditions for selecting an algorithm for a specific implementation. The choice of considering a centralized approach is equally good when the number of receive antenna is greater than one, \textit{i.e.}, \eqn{N_R > 1}. However, in case of a single antenna receiver, the \ac{JSFRA} formulation outlined in Section III-B is more efficient than the \ac{MSE} reformulation, as there is no need for the receiver update, thereby reducing the complexity significantly. 
	
	As far as the distributed approaches are concerned, the primal and the \ac{ADMM} schemes are equally favorable when \eqn{N_R = 1} since the signaling required to be exchanged between the coordinating \acp{BS} involves only the scalar interference values. However, when \eqn{N_R > 1}, then the \ac{KKT} based distributed approach is more efficient than the primal or the \ac{ADMM} schemes, since it has less signaling overhead for a throughput improvement. The choice of selecting \eqn{\ell_q} norm is discussed in Section II-B last paragraph. The above information is included in the paragraph before the simulations section in Section IV.
	
	However, if the reviewer is interested to understand the performance of the \ac{KKT} based scheme in a fading environment, please refer to the response provided for the reviewer 2, question 2 in this document for more details.
	
	\cmnts{b} - \review{The last discussion in section IV-C could benefit from restructuring. The information on how to obtain a practical distributed precoder design and to avoid backhaul exchange is too condensed and difficult to understand.}
	
	\resp It has been rewritten to improve the readability as suggested by the reviewer. We have included the reference [28] for more illustrative discussions on the backhaul signaling.
	
	\cmnts{c} - \review{For the simulation results, why not to unify configurations when possible? Having to read a different configuration for each graph is cumbersome and no additional comparisons are possible between figures. E.g. PL uniformly distributed between [0,-6] dB in Fig. 2 and [0,-3] dB in Fig. 3.}
	
	\resp We thank the reviewer for the comment. The main purpose of considering several path loss models is simply to show that the proposed algorithms works reasonably well on various scenarios. For benchmarking, we have also included the centralized algorithm to draw the difference between the other schemes. We are conducting more numerical experiments to have a more complete comparison of the schemes. Unfortunately those results cannot be included in the current manuscript due to space limitation. We aim to make them available online as a companion technical report whey they are ready.
	
	\cmnts{d} - \review{In Fig. 1, the description of the system model does not agree with the statement of N = 3 subchannels.}
	
	\resp We apologize for the error which has been fixed in the revised manuscript.
	
	\cmnts{e} - \review{In Fig. 4(b), the performance of Q-WSRME seems to be (in average) worse than Q-WSRM. However, that should not be the case, since Q-WSRME is taking into account the over allocation. Any reason for this?}
	
	\resp Indeed the performance of the Q-WSRME is better than Q-WSRM in Fig. 4(b). The misleading observation was probably due to the visualization effect where the colors representing the performance of sum arrivals and Q-WSRME are quite similar. We have updated Fig. 4 (and all other figures) in the revised manuscript to avoid such visualization problems.
	
	\cmnts{f} - \review{p 6, col 2, row 49: it should be \eqn{t_{l,k,n}} instead of \eqn{t_{l,n,k}}}
	
	\resp This issue is corrected in the revised manuscript.
	
	\cmnts{g} - \review{p 10, col 1, row 28: is it \eqn{\lambda} a dual variable?}
	
	\resp We thank the reviewer for pointing out the mistake. In fact, it should be the dual variable \eqn{\sigma_{l,k,n}} and not \eqn{\lambda_{l,k,n}}. We have fixed this problem in the revised manuscript.
	
	\cmnts{h} - \review{p 10, col 2, row 59: typo wHith}	
	
	\resp We have fixed this error.
	
	\cmnts{i} - {In general, a grammar check is recommended, several mistakes with respect to singular and plural nouns have been observed, e.g. -p 6, col 2, row 56 "... for A fixed receiverS" is not correct.}
	
	\resp We thank the reviewer for the comment and we have proofread the manuscript carefully to eliminate all grammar errors.
	
\end{enumerate}

\end{enumerate}

