This manuscript focuses on the beamforming and scheduling optimization for IBC MIMO-OFDM system, including the centralized and decentralized optimization methods. This is an interesting and important topic.

\vspace{0.25in}
\review{We thank the reviewer for reading the manuscript and providing valuable comments. The comments are really helpful in improving the manuscript.}

\begin{enumerate}

\cmnt{1} The number of transmitted packets \me{t_k}'s are optimization variables, which should be explicitly stated in the problem formulation of (6), (16), (19), (20) and (26) to avoid confusing.

\resp \review{Please note that the objective function uses \me{v_k =  Q_k - t_k = Q_k - \sum_{n = 1}^N \sum_{l = 1}^{L} \log_2(1+\gamma_{l,k,n})} expression instead of including an additional constraint for the transmitted packets using the rate expression and thus \me{t_k} is not a variable. However, in the MSE formulation, we have explicitly stated the optimization variable \me{t_k}, since it is present in the DC constraint (27).}

\cmnt{2} The manuscript states that the inequalities (16b) and (16c) achieve equality at optimality(line 23, page 5). This is not obvious. An easy case to check this statement is that assuming the system has two BS and each BS serves one user. When \me{Q_1=0} and 2nd BS has sufficiently large power, (16b) and (16c) do not hold equality. Rigorous proof is needed if authors stick to this statement.

\resp \review{
	\begin{enumerate}
		\item We thank the reviewer for the insightful comment. We have updated the manuscript to include the statement that the proposed approximation in (16b) and (16c) are the under-estimator for the SINR expression in (2) on the third paragraph in Section III-B.
		\item The under-estimator is tight when there is at least one user in each BS that cannot be served by the current transmission as discussed in Appendix A. It has multiple solutions, when the objective is zero even for a single BS. In this case, we can use a regularization term with the total power in the objective to obtain an unique solution. We have discussed the uniqueness of the proposed algorithm in detail in Appendix B-D.
	\end{enumerate}
	 }

\cmnt{3} The solution in (21) is obtianed for MMSE, i.e. for 2-norm(q=2). If \me{q=1} or \me{q=\infty}, it is actually an equivalent linear programming problem. Details for this solution should be provided.

\resp \review{We thank the reviewer for the critical comment. Note that the receiver has no explicit relation with the choice of \me{\ell_q} norm used in the objective function. The dependency is implicitly implied by the transmit precoders \me{\mvec{m}{l,k,n}}, which indeed depend on the value of \me{q}. We have modified the text to include this information on the sentence following after (22).}

\cmnt{4} The convergence proof need to be rigorous. The inequality of (23a) is opposite to the reference [28]. Also the statement on uniqueness of the transmit and the receive beamformers are not correct. Although we can choose one antenna to be real value, this does not mean the problem has unique solution!

\resp \review{We agree with the reviewer. Appendix B in the revised manuscript discuss the convergence proof rigorously. The uniqueness is also justified implicitly by the linear approximation performed on the DC constraint (16b), which makes the transmit precoders to be susceptible to the phase rotation. The uniqueness discussions are presented in Appendices B-D.}

\cmnt{5} (25) is generally wrong. (25) only holds when the MSE is minimized(by MMSE receiver) and the snr is the optimized(which is obtained by general eignvalue decompostion). This is clearly stated in the reference [5] and [6]. This can also be easily checked by comparing (25) and (2). Consequently the alternative formulation (26) based on this conclusion is questionable.

\resp \review{
	\begin{enumerate}
		\item We agree that the MSE equivalence with the SINR expression is valid only when the receiver is based on the MMSE objective. In our solution based on MSE reformulation, we have used the MMSE receiver irrespective of the \me{\ell_q} norm used in the objective. 
		\item Since the receivers are based on the MMSE objective, the equivalence is valid between the MSE and the SINR expression. Since the transmit precoders are updated in accordance with the objective, the MSE reformulation yields the same solution as the SCA approach based JSFRA scheme presented in Section III-B. It can also be verified from Fig. 1 for \me{\ell_1} objective. We have updated the manuscript to include this comments in the lines following (25) in Section III-C.
	\end{enumerate}
	}

\cmnt{6} For ADMM aproach, the determination of the value of \me{\rho} in equation (35a) should be discussed. 1. The numbers of transmitted packets for users  \me{t_k}'s are optimization variables. So they should be explicitly stated in the problem formulation (6), (16), (20) and (26) to avoid confusing.

\resp \review{We agree with the reviewer's concern. We have included reference [11] that discusses on the valid choices of the step size parameter for ADMM. We have included the statement that it depends on the system model under consideration. This information is updated on the manuscript in the lines following after (40).}

\end{enumerate}

