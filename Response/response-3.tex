This manuscript focuses on the beamforming and scheduling optimization for IBC MIMO-OFDM system, including the centralized and decentralized optimization methods. This is an interesting and important topic.

\vspace{0.1in}
\review{We thank the reviewer for reading the manuscript and providing valuable comments. The comments are really helps to improve the manuscript.}

\begin{enumerate}
\cmnt{1} The number of transmitted packets \me{t_k}'s are optimization variables, which should be explicitly stated in the problem formulation of (6), (16), (19), (20) and (26) to avoid confusing.

\resp \review{Please note that the objective function uses \me{v_k =  Q_k - t_k = Q_k - \sum_{n = 1}^N \sum_{l = 1}^{L} \log_2(1+\gamma_{l,k,n})} expression instead of including an additional constraint for the transmitted packets using the rate expression and thus \me{t_k} is not required as a variable. However, in the MSE formulation, we have explicitly stated the optimization variable \me{t_k}, since it is present in the DC constraint (27).
	\begin{enumerate}
		\item (6), (16), (20) and (21) does not depend upon the variable \me{t_k}
		\item (26) and (28) depend on the variable \me{t_k} and it is defined as an optimization variable.
	\end{enumerate}	
	}

\cmnt{2} The manuscript states that the inequalities (16b) and (16c) achieve equality at optimality(line 23, page 5). This is not obvious. An easy case to check this statement is that assuming the system has two BS and each BS serves one user. When \me{Q_1=0} and 2nd BS has sufficiently large power, (16b) and (16c) do not hold equality. Rigorous proof is needed if authors stick to this statement.

\resp \review{We thank the reviewer for the insightful comment. We have updated the manuscript to include the statement that the proposed approximation in (16b) and (16c) together provides an under-estimator for the SINR expression in (2). This information is highlighted in the third paragraph in Section III-B.}

\review{For the constraints (16b) and (16c) to be tight, there should be at least one user in each \ac{BS} with enough backlogged packets that cannot be served with the given power budget. On the other hand, to make the constraints active in all cases, the objective of the \ac{JSFRA} formulation should be regularized with the transmit power without affecting the solution as
\begin{equation}
\| \tilde{\mbf{v}} \|_q + \varphi \sum_{k \in \mc{U}} \sum_{n = 1}^{N} \sum_{l=1}^{L} \mathrm{tr} \left ( \mvec{m}{l,k,n} \mbf{m}^\herm_{l,k,n} \right )
\end{equation}
where \me{\varphi \approx 0}. Note that the modified objective will relax the power constraint by making the constraints (16b) and (16c) tight in the final solution. Note that the tightness discussion is valid only for the active spatial streams, \textit{i.e}, \me{\mvec{m}{l,k,n} \neq \mbf{0}^\tran}.
}
\begin{comment}
	\begin{enumerate}
		\item We thank the reviewer for the insightful comment. We have updated the manuscript to include the statement that the proposed approximation in (16b) and (16c) are the under-estimator for the SINR expression in (2) on the third paragraph in Section III-B.
		\item The under-estimator is tight when there is at least one user in each BS that cannot be served by the current transmission as discussed in Appendix A. It is not tight when the objective is zero even for a single BS. For this, we can regularize the objective with the total transmitted power as
		\begin{equation}
		\|\tilde{\mbf{v}}\|_q + \varphi \, \|\mbf{m}\|^2
		\end{equation} 
		where \me{\mbf{m}} is the stacked vector of all transmit precoders and \me{\varphi} is a constant chosen closer to zero such that it has no effect on the optimal solution. When the original objective is zero, \textit{i.e}, \me{\|\tilde{\mbf{v}}\| = 0}, the the problem becomes the min power design. Therefore, when the backlogged packets can be transmitted with the given power constraint, the regularized objective will find a minimum power precoders to empty the backlogged packets. It makes the objective tight at the optimal point.
		\item Note that the regularized objective makes the inequalities (16b) and (16c) tight at the optimal solution by making the power constraint in (16d) loose. It is discussed in Appendix A provided on the supplementary document.
	\end{enumerate}
	 }
\end{comment}

\cmnt{3} The solution in (21) is obtained for MMSE, i.e. for 2-norm(q=2). If \me{q=1} or \me{q=\infty}, it is actually an equivalent linear programming problem. Details for this solution should be provided.

\resp \review{We thank the reviewer for the critical comment. Note that the receiver has no explicit relation with the choice of \me{\ell_q} norm used in the objective function. The dependency is implicitly implied by the transmit precoders \me{\mvec{m}{l,k,n}}, which indeed depend on the value of the exponent \me{q} used in \me{\ell_q} norm. We have modified the text to include this information on the sentence following after (22).}

\cmnt{4} The convergence proof need to be rigorous. The inequality of (23a) is opposite to the reference [28]. Also the statement on uniqueness of the transmit and the receive beamformers are not correct. Although we can choose one antenna to be real value, this does not mean the problem has unique solution!

\resp \review{\begin{enumerate}
	\item We thank the reviewer for the comment. Rigorous convergence proof for the proposed algorithm is provided in Appendix A on the supplementary document. 	
	\item The uniqueness is justified implicitly by the MSE relation in (26c) for the MSE reformulation in (28), which makes the transmit precoders to be susceptible to the phase rotation. In general, the uniqueness of the beamformer can be guaranteed by adding a strongly convex term in the objective as discussed in Appendix A-C.
\end{enumerate}}

\cmnt{5} (25) is generally wrong. (25) only holds when the MSE is minimized(by MMSE receiver) and the SNR is the optimized(which is obtained by general Eignvalue decomposition). This is clearly stated in the reference [5] and [6]. This can also be easily checked by comparing (25) and (2). Consequently the alternative formulation (26) based on this conclusion is questionable.

\resp \review{
	\begin{enumerate}
		\item We agree that the MSE equivalence with the SINR expression is valid only when the receiver is based on the MMSE objective. In our solution based on MSE reformulation, we have used the MMSE receiver irrespective of the \me{\ell_q} norm used in the objective. The formulation assumes the MMSE receiver for all exponents used in the objective, and therefore, the precoders can be designed by using the MSE relation.
		\item The receivers are based on the MMSE objective, and therefore the equivalence is valid between the MSE and the SINR expression. Since the transmit precoders are updated in accordance with the objective, the MSE reformulation yields the same solution as the SCA approach based JSFRA scheme presented in Section III-B. It can also be verified from Fig. 1 with the \me{\ell_1} objective. We have updated the manuscript to include these comments in the lines following (25) in Section III-C.
	\end{enumerate}
	}

\cmnt{6} For ADMM aproach, the determination of the value of \me{\rho} in equation (35a) should be discussed. 1. The numbers of transmitted packets for users \me{t_k}'s are optimization variables. So they should be explicitly stated in the problem formulation (6), (16), (20) and (26) to avoid confusing.

\resp \review{
	\begin{enumerate}
		\item We agree with the reviewer's comment. We have included reference [11] that discusses on the valid choices of the step size parameter for ADMM. We have included the statement that the parameter \me{\rho} depends on the system model under consideration. This information has been updated on the manuscript in the lines following after (40).
		\item We considered the JSFRA scheme without the MSE reformulation as a representative example to discuss the distributed schemes. Since the rate variables are not included in the optimization problem, we do not include the rate variables \me{t_k} as an optimization variables in (37). On the contrary, for the MSE reformulation approach, as suggested by the reviewer, we need to have the rate \me{t_k}'s as the optimization variables.
	\end{enumerate}
	}

\end{enumerate}

