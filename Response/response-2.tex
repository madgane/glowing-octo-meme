\vspace{0.25in}
\review{We would like to thank the reviewer for providing valuable comments.}
\vspace{0.25in}
\begin{enumerate}

\cmnt{1} The logic from (6) to (16) is not clear. The only difference is the two newly introduced NON-CONVEX constraints (16b) and (16c), while the objective function (16a) and the constraint (16d) is the same as (6). The equivalence between (6) and (16) is not straightforward and it is confusing why the reformulation in (16) is beneficial.

\resp \review{The reformulation is required, since the SINR expression in (2) cannot be handled directly in the problem defined in (6). Note that the equality constraint imposed by the SINR expression in (2) is handled by the two explicit inequality constraints (16b) and (16c), which are the under estimator to the original problem in (6). Since the inequalities are used to replace the equality constraint, the objective could be improved by the iterative algorithm and the equality will be achieved after the convergence, thereby having one-to-one mapping at the final solution. We have updated the manuscript to include the details for better clarity on third paragraph in Section III-B.}

\cmnt{2} The authors use the successive convex approximation framework, but the approximate problem proposed by the authors is actually not convex. Inspecting (19), its objective function is the same as in (6), and the non-convexity of (6) comes exactly from the objective function, so (19) is not a convex problem. The same flaw is repeated several times in the approximate problems proposed by the authors.

\resp \review{
	\begin{enumerate}
		\item Please note that the objective function in the problem formulation is convex, which can be verified by writing the difference function inside the norm operator in epigraph form. Since the problem in (16) is not jointly convex on the variables \me{\mvec{m}{l,k,n}} and \me{\mvec{w}{l,k,n}}, we use alternating optimization (AO) approach by fixing \me{\mvec{w}{l,k,n}} and optimize for \me{\mvec{m}{l,k,n}} as a variable.
		\item Even after fixing \me{\mvec{w}{l,k,n}} as constant, the problem in (16) is nonconvex due to the DC constraint (16b), which is handled by the first order relaxation around some fixed operating point. Once the linear relaxation is performed, the problem in (20) is a convex optimization problem with the variables being \me{\mvec{m}{l,k,n},\gamma_{l,k,n},\beta_{l,k,n}}. The manuscript is updated to illustrate this clearly on the lines following after (19).
	\end{enumerate}
}

\cmnt{3} The authors proposed to use block coordinate descent method to solve (16). But as the authors have already pointed out, to apply block coordinate descent method, the constraint sets for different variables should be disjoint (uncoupled), which is however not the case in (16), because receive and transmit precoders (i.e w and m) are coupled in the constraints. It is confusing on its own why the authors made a statement that contradicts the proposed methodology, and the convergence followed is in question.

\resp \review{We thank the reviewer for the pointing out the flaw in the text. We have removed the incorrect statement on the block coordinate descent method with the AO approach on the fourth paragraph in Section III-B after (16). Indeed, due to the coupling of the transmit and the receive precoder variables, we cannot use the standard block coordinate descent method proof for the convergence of the proposed algorithm as pointed out by the reviewer. We have provided a completely rewritten convergence proof in Appendix B. }

\cmnt{4} Regarding the convergence of the SCA, the authors cited [27] for the convergence conditions, but the reference is wrong, because the conditions after the three bullets on page 6 are not mentioned in [27]. In case the authors disagree, please make the citation more specific, for example, specify the theorem/statement/proposition in [27] where those conditions are specified.

\resp \review{We apologize for the inappropriate reference cited in the original manuscript. We have provided a completely rewritten proof on the convergence of the centralized algorithm in Appendix B.}

\cmnt{5} The authors also cited [28] to establish the convergence of SCA. But the techniques of [27] and [28] are different, and the convergence conditions are different too. It is not clear why the authors need two set of convergence conditions for a single problem, and the resulting convergence analysis itself is not solid enough.

\resp \review{We agree with the reviewer's comment. We have provided the updated proof for the convergence of the centralized algorithm on Appendix B in the revised manuscript.}

\cmnt{6} Another comment on reference: to the reviewer's knowledge, the term SCA is never explicitly used in [2]. So please either correct the reference or be more specific (section, theorem, etc.).

\resp \review{We have removed the inappropriate citation of the references.}

\cmnt{7} The authors propose primal decomposition method, ADMM approach to the non-convex problem (19), while their convergence analysis is based on literature that proved convergence for convex problems only, e.g., [13]. So the convergence analysis is not trustworthy.

\resp \review{Please note that the distributed algorithm is performed for the convex subproblem presented in (20) and (28). Since the problem is convex, the distributed approach convergence can be guaranteed under certain conditions. Those are discussed in Appendix C on the convergence analysis for the distributed algorithms.}

\cmnt{8} The length of the paper is too extensive. Some of the reformulations as mentioned in the previous comment can be skipped. Also, Section III.D. is not deeply explained and does not bring additional value to the paper. The implications of ordering the sub-channels for the iterative approach should be carefully studied and extensively explained in a different publication.

\resp \review{We have included the sub-channel wise resource allocation or (SRA) scheme in Section III-D for the completeness. It is presented in the manuscript as an alternative suboptimal approach to perform sub-channel wise distributed precoder design by the centralized controller. We have updated the manuscript with more details for better understanding.  We have also stream-lined the manuscript so that some redundant discussion has been removed here and there. We also improved the grammar and fluency.}

\cmnt{9} Information regarding the value of q used to obtain the simulation results is missing (with exception of Fig. 3).

\resp \review{We have included the information regarding the norm used for the simulation in the figure captions. It is updated for Fig. 1 and Fig. 2.}

\cmnt{10} In Fig. 1 and Fig. 2, the labels for the system model do not fit with the written description. Additionally, the reference scheme Q-WSRM is not optimal, since it over allocates resources if there are few queued packets. Therefore, it is not interesting for comparison purposes.

\resp \review{We understand the reviewer's concern. We have updated the manuscript to include the descriptions in the text to refer the legends used in the figures. Table 1 is provided for the purpose of insight, since it deals with the SISO scenario with 3 sub-channels and 3 users. }

\cmnt{11} Assuming that Fig. 2 and Fig. 3 where obtained based on the same simulation setup, i.e. user queues, number of transmit and receive antennas and number of base stations, it is not clear why results in Fig. 3 are worse than Fig. 2 when comparing JSFRA. Even more, since the number of sub-channels is larger in Fig. 3, the result seems contradictory.

\resp \review{
	\begin{enumerate}
		\item We thank the reviewer for pointing out the issue in the system description for the figures. Please note that the simulation scenario is different for Fig. 2 and Fig. 3 in terms of path loss distribution and the sub-channel numbers.
		\item We have provided different scenario with the centralized algorithm as reference for studying the performance under different system model. The user path loss is distributed uniformly between $[0,-6]$ dB for Fig. 2 and between $[0,-3]$ dB for Fig. 3.
		\item We have updated the manuscript to include these details. In addition, the number of backlogged packets in the system for Fig. 3 is \me{[9,16,14,16,9,13,11,12]} bits. These facts are updated in the revised manuscript on Section V-B third paragraph.
	\end{enumerate}
  }

\end{enumerate}
