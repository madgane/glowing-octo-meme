\documentclass[10pt,letterpaper,onecolumn]{article}

\usepackage[left=1.0in,right=1.0in,top=1in,bottom=1in]{geometry}
\usepackage{amsmath,amssymb}
\usepackage[dvipdfmx]{graphicx}
\usepackage{acronym,etoolbox,color,multicol,xcolor}
\usepackage{subeqnarray,multirow,cite,array,setspace,fixltx2e,lineno,verbatim,mathtools,pgfplots,subfig}

\newcommand{\mbf}[1]{\mathbf{#1}}
\newcommand{\me}[1]{\( #1 \)}
\newcommand{\mc}[1]{\mathcal{#1}}
\newcommand{\fall}{\forall}
\newcommand{\set}[1]{\left \lbrace #1 \right \rbrace }
\newcommand{\mvec}[2]{\mathbf{#1}_{#2}}
\newcommand{\ith}[1]{{#1}^\mathrm{th}}
\newcommand{\pr}[1]{{#1}^\prime}
\newcommand{\mbfa}[1]{{\boldsymbol{#1}}}
\newcommand{\herm}{\mathrm{H}}
\newcommand{\sset}[1]{\left [ #1 \right ]}
\newcommand{\rfrac}[2]{{}^{#1}/{}_{#2}}
\newcommand{\eqspace}{\IEEEeqnarraynumspace}
\newcommand{\enoise}{\widetilde{N}_0}
\newcommand{\eqsub}{\IEEEyessubnumber}
\newcommand{\review}[1]{{\textcolor[rgb]{0 0 0.6}{#1}}}
\newcommand{\trace}{\mathrm{tr}}
\newcommand{\tran}{\mathrm{T}}
\newcommand{\R}[1]{\label{#1}\linelabel{#1}}
\newcommand{\lr}[1]{page~\pageref{#1}, line~\lineref{#1}}
\newcommand{\eqn}[1]{\(#1\)}
\newcommand{\mx}{\mbf{m}}
\newcommand{\my}{\mbf{w}}
\newcommand{\mz}{\mbfa{\gamma}}
\newcommand{\mxb}{{{\mbf{m}}}}
\newcommand{\myb}{{{\mbf{w}}}}
\newcommand{\iterate}[2]{{#1}^{(#2)}}
\newcommand{\iter}[3]{{#1}_{#2}^{(#3)}}
\newcommand{\ma}{\mbf{x}}
\acrodef{MSE}{mean squared error}
\acrodef{IBC}{interference broadcast channel}
\acrodef{MC}{multi-cell}
\acrodef{BS}{base station}
\acrodef{MIMO}{multiple-input multiple-output}
\acrodef{SISO}{single-input single-output}
\acrodef{MU}{multiple users}
\acrodef{OFDM}{orthogonal frequency division multiplexing}
\acrodef{WSRM}{weighted sum rate maximization}
\acrodef{QoS}{quality of service}
\acrodef{SCA}{successive convex approximation}
\acrodef{SNR}{signal-to-noise ratio}
\acrodef{MMSE}{minimum \acl{MSE}}
\acrodef{SIR}{signal-to-interference ratio}
\acrodef{SINR}{signal-to-interference-plus-noise ratio}
\acrodef{Q-WSRM}{queue \acl{WSRM}}
\acrodef{QM}{queue minimizing}
\acrodef{SRA}{spatial resource allocation}
\acrodef{JSFRA}{joint space-frequency resource allocation}
\acrodef{WMMSE}{weighted \acl{MMSE}}
\acrodef{KKT}{Karush-Kuhn-Tucker}
\acrodef{GP}{geometric programming}
\acrodef{SOC}{second-order cone}
%\acrodef{BCDM}{block coordinate descent method}
\acrodef{ADMM}{alternating directions method of multipliers}
\acrodef{PD}{primal decomposition}
\acrodef{DD}{dual decomposition}
\acrodef{FFR}{fractional frequency reuse}
\acrodef{DC}{difference of convex}
\acrodef{Q-WSRME}{\ac{Q-WSRM} extended}
\acrodef{TDD}{time division duplexing}
\acrodef{CSI}{channel state information}
\acrodef{AO}{alternating optimization}
\acrodef{OTA}{over-the-air}
\acrodef{PL}{path loss}
\acrodef{TDM}{time division multiplexing}
\acrodef{UC}{uncoordinated}

\DeclareGraphicsExtensions{{eps}}
\graphicspath{{Figures/}}
\renewcommand \thesection{\Roman{section}}
\renewcommand \thesubsection{\arabic{section}.\arabic{subsection}}

\renewcommand{\thefigure}{\Roman{figure}}
\renewcommand{\thesubfigure}{\alph{subfigure}}
\renewcommand{\theequation}{\roman{equation}}

\begin{document}
	
	\title{Reviewers Comments \& Authors Replies}
	
	\date{}
	\maketitle
	
	\begin{tabular}{p{1.25in}p{4.25in}}
		\textbf{Manuscript No.} & Paper T-SP-18051-2014, submitted to \emph{``IEEE Transactions on Signal Processing"} \\ \\
		\textbf{Title} & ``Traffic Aware Resource Allocation Schemes for Multi-Cell MIMO-OFDM Systems" \\ \\
		\textbf{Authors} & Ganesh Venkatraman, Antti T\"{o}lli, Markku Juntti, and Le-Nam Tran
	\end{tabular}
	
	\vspace{0.35in}
	The authors are grateful to the associate editor for giving us an additional opportunity to revise and correct our convergence analysis. In this regard, we would like to give a special thank to Review 3 who pointed out the issues with our convergence analysis of the proposed solutions. In this revision, we have mainly focused on providing a more mathematically rigorous  proof of the convergence analysis of the proposed iterative algorithms.
	%

\begin{enumerate}
\item We have provided additional information to improve the continuity in the algorithm formulation from the SINR expression as suggested by the reviewers.
\item We have included the discussion for the MSE based reformulation for different norms in the objective function.
\item We have rewritten the reduced complexity spatial resource allocation (SRA) in Section III-D for better readability.
\item We have shortened the ADMM approach as suggested by the reviewers in Section IV-B.
\item Rigorous convergence analysis is provided for the centralized algorithm in Appendix B.
\item We have included Section V-C on the queue behavior over multiple time instants. We have added Fig. 4 to show the performance of the proposed scheme over the existing precoder design algorithms.
\item We have addressed the issues in the citations and we have provided additional references to prove the convergence of the centralized algorithm as suggested by the reviewers.
\end{enumerate}
	Towards this end the following changes have been made to the revised manuscript.
	\begin{enumerate}
		\item We have restated the convergence results of the centralized algorithms in this revision, claiming that any accumulated point (i.e., a limit point) is a stationary point. The proof is essentially based on the convergence of a subsequence of iterates. We have rewritten Appendix A-D according to these changes. In particular, to discuss the convergence of the sequence of iterates generated by the centralized method, we use a unified superscript index \eqn{t} to refer to the step index of both \ac{AO} and \ac{SCA} procedures (\textit{i.e.}, index \eqn{t} denotes \ac{SCA} and \ac{AO} update step (\eqn{k,i})). Using this index \eqn{t}, we denote iterate \eqn{\ma^t} as a stacked vector of the minimizer obtained by solving convex subproblem (48) at each step \eqn{t} as \eqn{\mat{t} = \mbf{z}^{(i)}_k} for some \ac{SCA} and \ac{AO} iteration. Using this notation, we have revised our convergence discussions in Appendices A-C and A-D.		
		\item We have updated the convergence of distributed algorithms in a similar way.
	\end{enumerate}
	
	
	We have  also revised some parts of text to support the changes mentioned above and to further improve the readability. Specifically, the following parts of the revision have been rewritten.
	\begin{itemize}
		\item Page 6, paragraph after (28): We have elaborated the discussions on showing that the feasible set is bounded.
		\item Page 8, last paragraph in Section IV-B: We discussed the conditions under which the overall algorithm converges upon using the distributed schemes in each \ac{SCA} step.
		\item Last sentence after (51) on page 14: We emphasized the convergence of objective sequence only in each \ac{AO} step. 
		\item Last sentence after (60) on page 15: We have rewritten the convergence claim on the sequence of iterates generated by the Algorithm 2 similar to that of centralized method.
	\end{itemize}
	
	\vspace{1eM}
	In what follows, the comments are listed, each followed immediately by the corresponding reply from the authors. The reviewers questions in the revised manuscript are highlighted in blue color and the authors responses are presented in black. Unless otherwise stated, all the numbered items (figures, equations, references, citations, etc) in this response letter refer to the revised manuscript. All revisions in the manuscript are highlighted in blue color.
	\newpage
	\section*{Response to first reviewer comments}
	Comments:
\review{The response to the reviewer's concerns are generally satisfying, except the convergence proof.}

\vspace{0.1in}
We thank the reviewer for providing valuable and insightful comments.

For the resubmitted manuscript, the reviewer still has the following concerns

\begin{enumerate}
\cmnt{1} \review{Considering the length of the manuscript, it would be better to shorten some parts that are not new in this manuscript, e.g. III.A. More space can be left for convergence proof, which is very important. }

\resp We understand the reviewer concern. Based on the suggestion, we have removed couple of paragraphs from Section III.A and shortened the discussions on the simulation section to provide additional details for the convergence proof. 

\cmnt{2} \review{In convergence proof (48), why does the 2nd inequality hold? In fact, to prove the feasibility of  \eqn{{m_{k+1}^{(i)}, w_{*}^{(i-1)};m_k^{(i)} }}, the part between 2nd and 3rd inequality is not necessary, \eqn{\leq 0} directly follows the 2nd inequality since the solution \eqn{m_{k+1}^{(i)}, \gamma_{k+1}^{(i)} } is the optimal solution, and therefore feasible.  }

\resp We thank the reviewer for the critical comment. It is not possible to define the inequality with the previous optimal point. Since it is not possible to comment on the inclusion of the previous constraint set in the current update (except the earlier optimal point), the inequality is not guaranteed. Based on the comment from the reviewer, we have removed the inequality specifying the previous operating point from (48), since the newly found optimal point is inside the feasible set. 

\cmnt{3} \review{The solutions SCA iterations \eqn{\mbf{m}^{(i)}_k} does not necessarily converge. In fact \eqn{\mbf{m}} has compact feasible region, and thus \eqn{\mbf{m}^{(i)}_k} has limit points for any specific \eqn{i}.  However \eqn{m_{*}^{(i)}} does not necessarily exist( the whole sequence \eqn{\mbf{m}^{(i)}_k} may be not convergent). Similar problem happens to \eqn{\mbf{w}^{(i)}_k}. }

\resp We understand the reviewer concern. Even though the \ac{SCA} algorithm converges, it is not guaranteed that the iterates involved in the iterative algorithm, namely, \eqn{\mbf{m}^{(i)}_k} and \eqn{\mbf{w}^{(i)}_k} to converge. It is true that the iterates need not converge when the objective function is convex. We have modified the discussion on the strong convexity of the objective function to impose the uniqueness of the iterates in each \ac{SCA} update as well. By regularizing the objective function with a strongly convex term like \eqn{\|\mbf{m} - \mbf{m}^{(i)}_k\|^2}, we can guarantee the uniqueness of the iterates upon the \ac{SCA} convergence. We thank the reviewer for citing this issue on the sequence convergence and the uniqueness of the minimizer. We have included this information in the centralized convergence proof on Appendix A-B after (44), after (49) and on Appendix A-C to describe the strong convexity.

\cmnt{4} \review{Strict monotonicity with respect to the objective function \eqn{f} should be rigorously proved. Note that to guarantee the uniqueness of the beamformer iterates, (52) instead of the objective function is used.} 

\resp We thank the reviewer for the pointing out the issue involving the strict monotonicity. As suggested by the reviewer, we have included Appendix A-D to discuss the strict monotonicity of the objective function in each update point of the algorithm. Due to the strong convexity of the objective function, strict monotonicity is ensured after each update and also upon convergence. In order to ensure the strict monotonicity in the objective until convergence, \textit{i.e}, \eqn{f(\mbf{z}^{(i)}_{k+1}) \leq f(\mbf{z}^{(i)}_k), \forall k} and equal only when \eqn{\mbf{z}^{(i)}_k = \mbf{z}^{(i)}_{k+1}}, which is a limit point. Since each of the earlier update in the \ac{SCA} are strictly monotonic, we consider the point where the original convex function, say, \eqn{\hat{f}} has multiple limit points with the same objective value \eqn{\hat{f}(\mbf{z})}, where \eqn{f(\mbf{z}) = \hat{f}(z) + \|\mbf{z} - \mbf{z}_k^{(i)}\|^2} is the modified objective function from the \eqn{\ith{k+1}} iteration. Since \eqn{\hat{f}(\mbf{z})} is convex, it can have multiple limit points, therefore, it is monotonic. However, due to the strong convexity of \eqn{f(\mbf{z})} in (52), the limit point of the \ac{SCA} updates is also unique, and therefore has strict monotonicity in \eqn{f(\mbf{z}^{(i)}_k)} in each update. It can be justified, if \eqn{f(\mbf{z}^{(i)}_k)} is the unique minimizer in two consecutive steps even by including other limit points in the feasible set \eqn{\iter{\mc{X}}{k+1}{i}} with the same original objective value as \eqn{\hat{f}(\mbf{z}^{(i)}_{k})}. This information and the proof is also included in the revised manuscript in Appendix A-D.

\cmnt{5} \review{Note that the conclusions [32, Thm 2] and [26, Thm 10] have lots of assumptions. To invoke these reference, explicit exposition should be provided to show that these conclusions can be applied to our problem. The same questions occur to the proof in Appendix B, where conclusions in [11] [36] and [37] are used. Too many details are omitted to make the proof convincing and clear. }

\resp We thank the reviewer for raising the concern. We have updated the manuscript to include the details regarding the stationary point discussion in Appendix A-F and the convergence proof analysis for the primal and the \ac{ADMM} algorithms in Appendix B. Additional detail includes the conditions required for showing the stationarity of the limit point and we have included more details to utilize the conclusions from [37] to show the convergence of the ADMM scheme.

\end{enumerate}

	
	\newpage
	\section*{Response to second reviewer comments}
	\vspace{0.1in}
\review{We would like to thank the reviewer for providing valuable comments.}
\begin{enumerate}
\cmnt{1} The logic from (6) to (16) is not clear. The only difference is the two newly introduced NON-CONVEX constraints (16b) and (16c), while the objective function (16a) and the constraint (16d) is the same as (6). The equivalence between (6) and (16) is not straightforward and it is confusing why the reformulation in (16) is beneficial.

\resp \review{The reformulation is required, since the SINR expression in (2) cannot be handled directly in the problem defined in (6). Note that the equality constraint imposed by the SINR expression in (2) is handled by the two explicit inequality constraints (16b) and (16c), therefore leading to an approximation for the original problem in (6). We have updated the manuscript to include these details for better clarity in the third paragraph in Section III-B.}

\cmnt{2} The authors use the successive convex approximation framework, but the approximate problem proposed by the authors is actually not convex. Inspecting (19), its objective function is the same as in (6), and the non-convexity of (6) comes exactly from the objective function, so (19) is not a convex problem. The same flaw is repeated several times in the approximate problems proposed by the authors.

\resp \review{
	\begin{enumerate}
		\item Please note that the objective function in the problem formulation (16) is convex, since the \me{\gamma_{l,k,n}} is now treated as an optimization variable and not as an expression. Since the problem in (16) is not jointly convex on the variables \me{\mvec{m}{l,k,n}} and \me{\mvec{w}{l,k,n}}, we use alternating optimization (AO) approach by fixing \me{\mvec{w}{l,k,n}} and optimize for \me{\mvec{m}{l,k,n}} as a variable.
		\item Even after fixing \me{\mvec{w}{l,k,n}} as a constant, the problem in (16) is nonconvex due to the DC constraint (16b), which is handled by the first order relaxation around some fixed operating point. Once the linear relaxation is performed, the problem in (20) is a convex optimization problem with the variables being \me{\mvec{m}{l,k,n},\gamma_{l,k,n},\beta_{l,k,n}}. The manuscript is updated to illustrate this clearly in the lines following (19).
	\end{enumerate}
}

\cmnt{3} The authors proposed to use block coordinate descent method to solve (16). But as the authors have already pointed out, to apply block coordinate descent method, the constraint sets for different variables should be disjoint (uncoupled), which is however not the case in (16), because receive and transmit precoders (i.e w and m) are coupled in the constraints. It is confusing on its own why the authors made a statement that contradicts the proposed methodology, and the convergence followed is in question.

\resp \review{We thank the reviewer for pointing out the inappropriate text. We have removed the incorrect statement on the block coordinate descent method with the AO approach in the paragraph following (16) in Section III-B. Indeed, due to the coupling of the transmit and the receive precoder variables, we cannot use the proof of the standard block coordinate descent method for the convergence of the proposed algorithm as pointed out by the reviewer. We have provided a completely rewritten convergence proof in Appendix A. }

\cmnt{4} Regarding the convergence of the SCA, the authors cited [27] for the convergence conditions, but the reference is wrong, because the conditions after the three bullets on page 6 are not mentioned in [27]. In case the authors disagree, please make the citation more specific, for example, specify the theorem/statement/proposition in [27] where those conditions are specified.

\resp \review{We apologize for the inappropriate reference cited in the original manuscript. We have provided a completely rewritten proof on the convergence of the centralized algorithm in Appendix A on the supplementary document.}

\cmnt{5} The authors also cited [28] to establish the convergence of SCA. But the techniques of [27] and [28] are different, and the convergence conditions are different too. It is not clear why the authors need two set of convergence conditions for a single problem, and the resulting convergence analysis itself is not solid enough.

\resp \review{We agree with the reviewer's comment. We have provided the updated proof for the convergence of the centralized algorithm on Appendix A on the supplementary document.}

\cmnt{6} Another comment on reference: to the reviewer's knowledge, the term SCA is never explicitly used in [2]. So please either correct the reference or be more specific (section, theorem, etc.).

\resp \review{We have removed the inappropriate citation of the references, and provided appropriate references for SCA in [22,32,34].}

\cmnt{7} The authors propose primal decomposition method, ADMM approach to the non-convex problem (19), while their convergence analysis is based on literature that proved convergence for convex problems only, e.g., [13]. So the convergence analysis is not trustworthy.

\resp \review{Please note that the distributed algorithm is performed for the \textbf{convex subproblem} presented in (20) and (28). Since the problem is convex, the distributed approach convergence can be guaranteed under certain conditions. Those are discussed in Appendix B on the convergence analysis for the distributed algorithms on the supplementary document.}

\cmnt{8} The length of the paper is too extensive. Some of the reformulations as mentioned in the previous comment can be skipped. Also, Section III.D. is not deeply explained and does not bring additional value to the paper. The implications of ordering the sub-channels for the iterative approach should be carefully studied and extensively explained in a different publication.

\resp \review{We have included the sub-channel wise resource allocation or (SRA) scheme in Section III-D for the completeness. It is presented in the manuscript as an alternative suboptimal approach to perform sub-channel wise distributed precoder design by the centralized controller. We have updated the manuscript with more details.  We have also stream-lined the manuscript so that some redundant discussion has been removed here and there. We also improved the grammar and continuity of the paper.}

\cmnt{9} Information regarding the value of q used to obtain the simulation results is missing (with exception of Fig. 3).

\resp \review{We have included the information regarding the norm used for the simulations in the figure captions of Fig. 1 and Fig. 2.}

\cmnt{10} In Fig. 1 and Fig. 2, the labels for the system model do not fit with the written description. Additionally, the reference scheme Q-WSRM is not optimal, since it over allocates resources if there are few queued packets. Therefore, it is not interesting for comparison purposes.

\resp \review{We thank the reviewer for pointing out the issue. We have updated the manuscript to include the descriptions in the text to refer the legends used in the figures. Table 1 is provided for the purpose of insight, since it deals with the SISO scenario with 3 sub-channels and 3 users. }

\cmnt{11} Assuming that Fig. 2 and Fig. 3 where obtained based on the same simulation setup, i.e. user queues, number of transmit and receive antennas and number of base stations, it is not clear why results in Fig. 3 are worse than Fig. 2 when comparing JSFRA. Even more, since the number of sub-channels is larger in Fig. 3, the result seems contradictory.

\resp \review{
	\begin{enumerate}
		\item We thank the reviewer for pointing out the issue in the system model description for the figures. Please note that the simulation scenarios are different for Fig. 2 and Fig. 3 in terms of path loss distribution and the sub-channel numbers.
		\item We have provided different scenario with the centralized algorithm as reference for studying the performance under different system models. The user path loss is distributed uniformly between $[0,-6]$ dB for Fig. 2 and between $[0,-3]$ dB for Fig. 3.
		\item We have updated the manuscript to include these details. In addition, the number of backlogged packets in the system for Fig. 3 is \me{[9,16,14,16,9,13,11,12]} bits. These numbers have been updated in the revised manuscript on the third paragraph of Section V-B.
	\end{enumerate}
  }

\end{enumerate}

	
	\newpage
	\section*{Response to third reviewer comments}
	\reviewF{The authors have modified the manuscript taken into account the reviewers' comments. However, the proof of convergence still lacks of convincing argumentation to make this paper suitable for publication. Two main aspects are arguable as follows: }

\vspace{1eM}
\underline{\textit{Reply:}} We thank the reviewer for pointing the flaw in our earlier convergence proof. Based on the reviewer's suggestion, we have rewritten Appendix A-D with similar arguments used to discuss the convergence of projected gradient algorithms in [R3]. We hope that our current changes will be convincing enough to make the convergence proof complete.

\begin{itemize}
	
	\item \reviewF{When discussing the convergence of the \ac{SCA} algorithm (Appendix A-C), the authors claim that the \ac{SCA} algorithm converges because the objective function is strictly decreasing and the minimizer in each iteration is unique. However, this is in general not enough to guarantee the convergence of the whole sequence. Instead, one can only claim that every limit point of the whole sequence is a stationary point, or globally optimal solution if the problem is convex. As an example, simply consider the gradient projection algorithm, where each subproblem is strongly convex and the objective function  is strictly decreasing but the whole sequence may not converge. The authors are referred to Bertsekas'  book: Nonlinear Programming.}

	\vspace{1eM}	
	\underline{\textit{Reply:}} We agree with the reviewer's comment. The convergence of iterates cannot be guaranteed based on the strict monotonicity of the objective sequence and the uniqueness of minimizer in each step. We apologize for our emphasis on the convergence of sequence of iterates. In view of reviewer's suggestion, we have revised our claim based on subsequence convergence theory. 
		
	Since the feasible set is bounded, the iterates generated by the iterative algorithm are bounded. Therefore, there exists at least one subsequence that converges to a limit point. Now, by using a convergent subsequence of original sequence of iterates, we argued that every limit point is a stationary point of the original nonconvex problem. In order to do so, we used a unified superscript index \eqn{t} to represent both \ac{AO} index \eqn{i} and \ac{SCA} indexing \eqn{k}. Now, by using this unified index \eqn{t}, we represent iterate \eqn{\mbf{x}^{t}} as the solution variable produced by the iterative algorithm in every \ac{SCA} \eqn{k} and \ac{AO} step {i}. By using this notation and the subsequence argument, we have revised our convergence discussions in Appendices A-C and A-D. 
		
	Even though the feasible sets of the original nonconvex problem in (6) and the relaxed \ac{MSE} reformulated problem in (26) are bounded due to the total power constraint, we noted that the solution space of the relaxed nonconvex problem (16) need not be bounded. It is because of the newly introduced variables in (16b) and (16c) to relax the \ac{SINR} expression, since \eqn{\beta_{l,k,n}} can assume any value satisfying (16c) without violating any other constraints in (16) when \eqn{\gamma_{l,k,n} = 0}. In order to ensure boundedness of the feasible set, we can include an additional bounding constraint on \eqn{\beta_{l,k,n}} as \eqn{\beta_{l,k,n} \leq B_{\max}}, where \eqn{B_{\max}} can be a maximum interference threshold seen by any user in the network with the maximum transmit power constraint of \eqn{P_{\max}}. Note that the additional bounding constraint on \eqn{\beta_{l,k,n}} will not alter the solution space of other optimization variables. However, as a consequence of considering the regularized objective in (46b), the additional constraint on \eqn{\beta_{l,k,n}} is not required, since \eqn{\beta_{l,k,n}} is bounded due to the proximal term in (46b). We have highlighted this information on page 6, after (28). 
	
	\vspace{1eM}	
	\item \reviewF{Even under the assumption that the inner loop SCA converges, there are doubts with respect to the convergence of the alternating optimization algorithm described by (54). The authors cite [27] to justify convergence, but it is not straightforward to claim that the algorithmic model considered in this manuscript is the same as in [27]. More specifically, let us consider the variable \eqn{x} (so it is consistent with the authors' notation in Appendix A-D). In the algorithmic model of [27], at iteration \eqn{t+1}, all elements of \eqn{\mbf{x}} are updated simultaneously, based on \eqn{\sol{t}}. But in the authors' model, the update of \eqn{\mbf{x}} is performed in two phases: in the first phase, only the elements of \eqn{\mbf{m}} and \eqn{\mbf{\gamma}} in \eqn{\mbf{x}} are updated based on \eqn{\sol{t}}. In the second phase, the remaining elements of \eqn{\mbf{x}} are updated, i.e. \eqn{\mbf{w}} and \eqn{\mbf{\gamma}}, based on both \eqn{\sol{t}} and the elements that have been updated in the first phase. This is not a trivial difference and thus the direct application of the conclusion from [27] in the current context cannot be taken for granted. Due to the open concerns after several revision rounds, it is recommended to reject the paper.}
	
	\vspace{1eM}
	\underline{\textit{Reply:}} We understand the reviewer's concern on using [R1] to our problem involving \ac{SCA} and \ac{AO} updates, which involves partial update of optimization variables. Even though our revised convergence proof is not based on [R1], we clarify our usage of [R1] for \ac{AO} update. Therefore, we refer to [R2], where the problem considered was to design tight frames based on solving matrix nearness problem using \ac{AO} approach. In order to prove the convergence of the sequence of iterates generated by the \ac{AO} iterations, [R1] is used with the composition of two sub-algorithms as discussed briefly in [Appendix I and II, R2].
		
	We note that in our previous manuscript, the convergence of the sequence of beamformer iterates claim based on [R1] was incomplete. We apologize for the incomplete usage of the theorem presented in [R1]. In [Theorem. 3.1, R1], it has been shown that if the strict monotonicity of the objective sequence is ensured and the iterates are bounded, then \textit{either the sequence of iterates converges or the accumulation points of \eqn{\{\sol{t}\}} is a continuum}. Therefore, our earlier claim on the guaranteed convergence of the sequence of iterates is not correct as pointed out by the reviewer. Ensuring strict monotonic decrease in the objective sequence by restricting the solution set is discussed briefly in [Section 4, R1]. By using [R1], we can only claim the convergence of subsequence of \eqn{\{\sol{t}\}} and not the sequence itself.
	
	We revised Appendix A-D titled \textit{"Stationarity of Limit Points"} to discuss briefly on the convergence of iterates. We show that every limit point of the sequence of iterates generated by the centralized algorithm is a stationary point.
	
	\vspace{1eM}
	\begin{itemize}
		
		\item[R1.] R.R. Meyer, ``Sufficient Conditions for the Convergence of Monotonic Mathematical Programming Algorithms," \emph{Journal of Computer and System Sciences}, vol. 12, no. 1, pp. 108-121, 1976
		\item[R2.] J. Tropp, I. Dhillon, R. Heath, and T. Strohmer, ``Designing structured tight frames via an alternating projection method," \emph{IEEE Trans. Inform. Theory}, vol. 51, no. 1, pp. 188–209, Jan. 2005.
		\item[R3.] D. P. Bertsekas, ``Nonlinear Programming," 2nd ed., \emph{Athena Scientific}, 1999.
	\end{itemize}
	
\end{itemize}
	
	\newpage
	\section*{Response to fourth reviewer comments}
	\review{This reviewer's concerns have been addressed, and this manuscript is now deemed fit for publication.}

\vspace{1eM}
\underline{\textit{Reply:}} We thank the reviewer for the constructive comments and recommending the revised manuscript for the publication. 

	
\end{document}

