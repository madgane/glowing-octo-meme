\begin{enumerate}
	\item We have rewritten Appendix A-D to discuss the stationarity of limit points of the sequence of beamformer iterates generated by the centralized algorithms.
	\item We have updated the convergence of distributed algorithm to reflect the same.
\end{enumerate}

In what follows, the comments are listed, each followed immediately by the corresponding reply from the authors. The reviewers questions in the revised manuscript are highlighted in blue color and the authors responses are presented in black. Unless otherwise stated, all the numbered items (figures, equations, references, citations, etc) in this response letter refer to the revised manuscript. All revisions in the manuscript are highlighted in blue color.

\subsection*{List of Changes:}
\vspace*{1eM}
\begin{itemize}
	\item Page 6, paragraph after (28)
	\item Page 8, last paragraph in Section IV-B
	\item Page 12, paragraph following itemized requirements in Appendix A
	\item Last sentence after (51) on page 14
	\item Last paragraph in Appendix A-C
	\item Appendix A-D is rewritten using subsequence convergence
	\item Last sentence after (60) on page 15
\end{itemize}

Finally, to discuss the convergence of the sequence of iterates generated by the centralized scheme in Algorithm 1, we use an unified superscript index \eqn{t} to represent both \ac{AO} index \eqn{i} and \ac{SCA} indexing \eqn{k}. By doing so, we represent iterate \eqn{\mbf{x}^{t}} as the solution variable produced by the iterative algorithm in every \ac{SCA} \eqn{k} and \ac{AO} step {i}. Using this notation, we have revised our convergence discussions in Appendices A-C and A-D. 
