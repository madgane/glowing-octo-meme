The \ac{ADMM} approach can also be used to decouple the precoder design across multiple \acp{BS} to solve the convex subproblem in \eqref{eqn-decent-1}. Generally, the \ac{ADMM} is preferred over the \acl{DD} in \cite{tolli2011decentralized} for its robustness and improved convergence behavior \cite{boyd2011distributed}. In contrast to the \acl{PD}, the \ac{ADMM} relaxes the interference constraints by including in the objective function of each subproblem with a penalty pricing \cite{palomar2006tutorial,boyd2011distributed}. Similar approach for the precoder design in the minimum power context was considered in \cite{admm_robust}.

Using the formulation presented in \cite{boyd2011distributed,admm_robust}, we can write the \ac{BS} \me{b} specific \ac{ADMM} subproblem for the \me{\ith{j}} iteration as
\iftoggle{single_column}{\allowdisplaybreaks
\begin{IEEEeqnarray}{rCl} \label{eqn-dual-3}\allowdisplaybreaks \neqsub
\underset{\substack{\gamma_{l,k,n}, \mvec{m}{l,k,n} \\ \beta_{l,k,n}, \mbfa{\zeta}_b}}{\text{minimize}} &\quad& \| \tilde{\mbf{v}}_{b} \|_q + \mbfa{\nu}^{(j)\,\tran}_b  \left ( \mbfa{\zeta}_b - \mbfa{\zeta}^{(j)}_b \right ) +  \frac{\rho}{2} \left \| \mbfa{\zeta}_b - \mbfa{\zeta}^{(j)}_b \right \|^2 \eqsub \eqspace \\
\text{subject to} &\quad& \sum_{n = 1}^N \sum_{k \in \mathcal{U}_b} \sum_{l=1}^L \trace \, (\mvec{m}{l,k,n} \mvec{m}{l,k,n}^\herm) \leq P_{{\max}} \eqsub \label{power_update} \\
&\quad&\sum_{\bar{b} \in \bar{\mc{B}}_b} \zeta_{l,k,n,\bar{b}} + \sum_{\mathclap{\{\bar{l},\bar{k}\}\neq\{l,k\}}} |\mvec{w}{l,k,n}^\herm \mvec{H}{b,k,n} \mvec{m}{\bar{l},\bar{k},n} |^2 + \enoise \leq \beta_{l,k,n} \eqspace \eqsub \\
&\quad& \sum_{k \in \mc{U}_b} \sum_{l=1}^L |\mvec{w}{\bar{l},\bar{k},n}^\herm \mvec{H}{b,\bar{k},n} \mvec{m}{l,k,n} |^2 \leq \zeta_{\bar{l},\bar{k},n,b} \; \forall \bar{k} \in \bar{\mc{U}}_b \; \forall n \eqsub \\
&\quad& \text{and } \eqsub \eqref{eqn-8}
\end{IEEEeqnarray}
}{\begin{IEEEeqnarray}{rl} \label{eqn-dual-3}\allowdisplaybreaks \neqsub
	&\underset{\substack{\gamma_{l,k,n}, \mvec{m}{l,k,n} \\ \beta_{l,k,n}, \mbfa{\zeta}_b}}{\text{minimize}} \: \; \| \tilde{\mbf{v}}_{b} \|_q + \mbfa{\nu}^{(j)\,\tran}_b   ( \mbfa{\zeta}_b - \mbfa{\zeta}^{(j)}_b  ) +  \tfrac{\rho}{2}  \| \mbfa{\zeta}_b - \mbfa{\zeta}^{(j)}_b  \|^2 \eqsub \eqspace \\
	&\text{subject to} \quad \sum_{n = 1}^N \sum_{k \in \mathcal{U}_b} \sum_{l=1}^L \trace \, (\mvec{m}{l,k,n} \mvec{m}{l,k,n}^\herm) \leq P_{{\max}} \eqsub \label{power_update} \\
	&\sum_{\bar{b} \in \bar{\mc{B}}_b} \zeta_{l,k,n,\bar{b}} + \sum_{\mathclap{\{\bar{l},\bar{k}\}\neq\{l,k\}}} |\mvec{w}{l,k,n}^\herm \mvec{H}{b,k,n} \mvec{m}{\bar{l},\bar{k},n} |^2 + \enoise \leq \beta_{l,k,n} \eqspace \eqsub \\
	&\sum_{k \in \mc{U}_b} \sum_{l=1}^L |\mvec{w}{\bar{l},\bar{k},n}^\herm \mvec{H}{b,\bar{k},n} \mvec{m}{l,k,n} |^2 \leq \zeta_{\bar{l},\bar{k},n,b} \; \forall \bar{k} \in \bar{\mc{U}}_b \; \forall n \eqsub \\
	& \text{and } \eqsub \eqref{eqn-8}
\end{IEEEeqnarray}
}
where \me{\mbfa{\zeta}^{(j)}_b} denotes the interference vector updated from the earlier iteration and \me{\mbfa{\nu}^{(j)}_b} represents the dual vector corresponding to the equality constraint at the \me{\ith{j}} iteration as
\begin{equation} \label{const-admm1}
\mbfa{\zeta}_b = \mbfa{\zeta}^{(j)}_b.
\end{equation}

Upon solving \eqref{eqn-dual-3} for \me{\mbfa{\zeta}_b \forall b} in the \me{\ith{j}} iteration, the next iterate is updated by exchanging the corresponding interference terms between two \acp{BS} \me{b} and \me{b_k} as
\begin{equation}
\mbfa{\zeta}_{b_k}(b)^{(j+1)} = \mbfa{\zeta}_{b}(b_k)^{(j+1)} = \frac{\mbfa{\zeta}_{b}(b_k) + \mbfa{\zeta}_{b_k}(b)}{2}.
\label{if-sg-update}
\end{equation}
The dual vector for the next iteration is updated by using the subgradient search to maximize the dual objective as
\begin{equation}
\mbfa{\nu}_b^{(j+1)} = \mbfa{\nu}_b^{(j)} + \rho \: (\mbfa{\zeta}_b - \mbfa{\zeta}_b^{(j+1)}  )
\label{dual-sg-update}
\end{equation}
where step size parameter \me{\rho} is chosen in accordance with \cite{boyd2011distributed} to depend on the system model. The convergence rate is susceptible to the choice of step size parameter \me{\rho}. In numerical simulations, we consider step size \me{\rho=2}. The above iterative procedure is performed until convergence or terminated when exceeding a predetermined number of steps, say, \eqn{J_{\max}}. Algorithm \ref{algo-3} outlines a practical way of implementing the \ac{ADMM} based precoder design with minimal signaling overhead.
\begin{algorithm}
	\SetAlgoLined
	\DontPrintSemicolon
	\BlankLine
	\SetKwInput{KwInit}{Initialize}
	\KwIn{\me{a_k, \, Q_k, \, \mvec{H}{b,k,n},\; \fall b \in \mathcal{B}, \, \fall k \in \mathcal{U}, \fall n \in \mathcal{N}}}
	\KwOut{\me{\mvec{m}{l,k,n}} and \me{\mvec{w}{l,k,n} \forall l,k,n}}
	\KwInit{\me{i=0,j=0} and \me{{\mbf{m}}_{l,k,n}} satisfying \eqref{power_update}}
	update \me{\mvec{w}{l,k,n}} with \eqref{eqn-10} and \me{\tilde{\mbf{u}}_{l,k,n}} using \eqref{eqn-6.3} and \eqref{eqn-wsrm-expr} \;
	initialize the vectors \me{\mbfa{\nu}^{(0)}_b = \mbfa{0}^{\mathrm{T}}, \mbfa{\zeta}^{(0)}_b = \mbfa{0}^{\mathrm{T}}, \forall b \in \mc{B}} \;
	\ForEach{\ac{BS} \me{b \in \mc{B}}}{
		\Repeat{convergence or \me{i \geq I_{\max}}}{
			\Repeat{convergence or \me{j \geq J_{\max}}, \me{\forall b \in \mc{B}}}{
				solve for \me{\mvec{m}{l,k,n}} and \me{\mbfa{\zeta}_{b}} with \eqref{eqn-dual-3} using \me{\mbfa{\zeta}^{(j)}_{b}} \;
				exchange \me{\mbfa{\zeta}_{b}} among \acp{BS} in \me{\mc{B}} \;
				update \me{\mbfa{\zeta}^{(j+1)}_b} and \me{\mbfa{\nu}^{(j+1)}_b} using \eqref{if-sg-update} and \eqref{dual-sg-update}\;
				update \me{j = j + 1}\;
			}
			evaluate \me{\mvec{w}{l,k,n}} using \eqref{eqn-10} at each user \;
			update \me{\tilde{\mbf{u}}_{l,k,n}} using \eqref{eqn-6.3} and \eqref{eqn-wsrm-expr} for \ac{SCA} point or \me{\tilde{\epsilon}_{l,k,n}} using \eqref{eqn-mse-1.3} for \ac{MSE} operating point \;
			$i = i + 1$, \me{j=0} \;
		}
	}
	\caption{Distributed \ac{JSFRA} scheme using \ac{ADMM}}
	\label{algo-3}
\end{algorithm}
%\begin{algorithm}
%	\SetAlgoLined
%	\DontPrintSemicolon
%	\BlankLine
%	\SetKwInput{KwInit}{Initialize}
%	\KwIn{\me{a_k, \, Q_k, \, \mvec{H}{b,k,n},\; \fall b \in \mathcal{B}, \, \fall k \in \mathcal{U}, \fall n \in \mathcal{N}}}
%	\KwOut{\me{\mvec{m}{l,k,n}} and \me{\mvec{w}{l,k,n} \fall users}}
%	\KwInit{\me{i=0,j=0} and \me{{\mbf{m}}_{l,k,n}} satisfying \eqref{power_update}}
%	update \me{\mvec{w}{l,k,n}} with \eqref{eqn-10} and \me{\tilde{\mbf{u}}_{l,k,n}} using \eqref{eqn-6.3} and \eqref{eqn-wsrm-expr} \;
%	initialize the vectors \me{\mbfa{\nu}^{(0)}_b = \mbfa{0}^{\mathrm{T}}, \mbfa{\zeta}^{(0)}_b = \mbfa{0}^{\mathrm{T}}, \forall b \in \mc{B}} \;
%	\Repeat{convergence or \me{i \geq I_{\max}}}{ 
%		\Repeat{convergence or \me{k \geq K_{\max}}}{
%			\Repeat{convergence or \me{j \geq J_{\max}}, \me{\forall b \in \mc{B}}}{
%				solve for \me{\{\mbf{m}\}} and \me{\mbfa{\zeta}_{b}} with \eqref{eqn-dual-3} using \me{\mbfa{\zeta}^{(j)}_{b}} \;
%				exchange \me{\mbfa{\zeta}_{b}} among \acp{BS} in \me{\mc{B}} \;
%				update \me{\mbfa{\zeta}^{(j+1)}_b} and \me{\mbfa{\nu}^{(j+1)}_b} using \eqref{if-sg-update} and \eqref{dual-sg-update}\;
%				update \me{j = j + 1}\;
%			}
%			update \me{\{\tilde{\mbf{u}}\}} using \eqref{eqn-6.3} and \eqref{eqn-wsrm-expr} for \ac{SCA} point or \me{\{\tilde{\epsilon}\}} using \eqref{eqn-mse-1.3} for \ac{MSE} operating point \;
%			$k = k + 1$, \me{j=0} \;
%		}
%		using \me{\{\mbf{m}\}}, update the receivers \me{\{\mbf{w}\}} of all users \;		
%		update \me{\{\tilde{\mbf{u}}\}} or \me{\{\tilde{\epsilon}\}} respectively with \me{\{\mbf{m}\}} and \me{\{\mbf{w}\}} \;
%		$i = i + i$, \me{k=0} \;
%	}
%	\caption{Distributed \ac{JSFRA} scheme using \ac{ADMM}}
%	\label{algo-3}
%\end{algorithm}

Assuming that the following conditions are satisfied by the distributed schemes in each \ac{SCA} step, \textit{i.e.}, (i) by ensuring the uniqueness of the minimizer, and (ii) the distributed methods are carried out until convergence or for \eqn{j \to \infty}, then the convergence analysis follows the discussions presented in Appendices \ref{sec-3.5} and \ref{sec-dist-conv}. Moreover, it is still valid even if the receivers are updated together with the transmit precoders as in Algorithm \ref{algo-3}, since the strict monotonic decrease in the objective sequence can still be ensured. The reason is that the \ac{MMSE} receiver is optimal for the fixed transmit precoders obtained after each \ac{SCA} update. However, in practice, to reduce the amount of information exchange between the coupling \acp{BS}, the distributed schemes are often iterated for a limited number of iterations, say, for \eqn{J_{\max}} only. In such a case, convergence of the distributed algorithms cannot be ensured, since it is impossible to show that the objective value is monotonically decreasing in each \ac{SCA} update step. We recall that in each primal or \ac{ADMM} iteration, the global objective cannot be guaranteed to decrease monotonically.
	
%	Additionally, if the receivers are updated together with the transmit precoders using \eqref{eqn-10} in each \ac{SCA} step, strict monotonicity is still ensured, since the \ac{MMSE} receivers are optimal for the fixed transmit precoders.}
%However, if the distributed methods are performed for an adequate number of iterations to ensure the strict monotonicity in each \ac{SCA} update, then the limit point of the sequence of iterates is a stationary point of the nonconvex problem \eqref{eqn-6}. Note that the receivers can also be updated together with the transmit precoders in each \ac{SCA} step, \textit{i.e.}, \eqn{K_{\max} = 1}, since the \ac{MMSE} receivers are optimal for fixed transmit precoders obtained in each \ac{SCA} step.}

\begin{comment}
\begin{algorithm}
\SetAlgoLined
\DontPrintSemicolon
\BlankLine
\SetKwInput{KwInit}{Initialize}
\KwIn{\me{a_k, \, Q_k, \, \mvec{H}{b,k,n},\; \fall b \in \mathcal{B}, \, \fall k \in \mathcal{U}, \fall n \in \mathcal{N}}}
\KwOut{\me{\mvec{m}{l,k,n}} and \me{\mvec{w}{l,k,n} \fall l}}
\KwInit{\me{i=0,j=0} and \me{{\mbf{m}}_{l,k,n}} satisfying \eqref{power_update}}
update \me{\mvec{w}{l,k,n}} with \eqref{eqn-10} and \me{\tilde{\mbf{u}}_{l,k,n}} using \eqref{eqn-6.3} and \eqref{eqn-wsrm-expr} \;
initialize the interference vectors \me{\mbfa{\zeta}^{(0)}_b = \mbfa{0}^{\mathrm{T}}, \forall b \in\mc{B}} \;
initialize the dual vectors \me{\mbfa{\nu}^{(0)}_b = \mbfa{0}^{\mathrm{T}}, \forall b \in \mc{B}} \;
\ForEach{\ac{BS} \me{b \in \mc{B}}}{
\Repeat{convergence or \me{i \geq I_{\max}}}{
\Repeat{convergence or \me{j \geq J_{\max}}}{
solve for \me{\mvec{m}{l,k,n}} and \me{\mbfa{\zeta}_{b}} with \eqref{eqn-dual-3} using \me{\mbfa{\zeta}^{(j)}_{b}} \;
exchange \me{\mbfa{\zeta}_{b}} among \acp{BS} in \me{\mc{B}} \;
update \me{\mbfa{\zeta}^{(j+1)}_b} and \me{\mbfa{\nu}^{(j+1)}_b} using \eqref{if-sg-update} and \eqref{dual-sg-update}\;
\me{j = j + 1}\;
}
precoded downlink pilot transmission with \me{\mvec{m}{l,k,n}}\;
update \me{\mvec{w}{l,k,n}} to all \acp{BS} in \me{\mc{B}} using precoded uplink pilots \cite{komulainen2013effective}\;
update \me{\tilde{\mbf{u}}_{l,k,n}} using \eqref{eqn-6.3} and \eqref{eqn-wsrm-expr} for \ac{SCA} point or \me{\tilde{\epsilon}_{l,k,n}} using \eqref{eqn-mse-1.3} for \ac{MSE} operating point \;
$i = i + 1$, \me{j=0} \;
}
}
\caption{Distributed \ac{JSFRA} scheme using \ac{ADMM}}
\label{algo-3}
\end{algorithm}
The \ac{ADMM} decomposition method is based on the \acl{DD}, however it shows better convergence properties. In contrast to the \acl{PD} problem, the \ac{ADMM} method relaxes the interference constraints by including it in the objective function of each subproblem with a penalty pricing \cite{palomar2006tutorial,boyd2011distributed}. In order to decouple the problem \eqref{eqn-decent-1}, the coupling variables \me{\zeta_{l,k,n,b}} in \eqref{eqn-decent-3} are replaced by the respective local copies \me{\mbfa{\zeta}^{\set{b}}, \forall b \in \mc{B}}, which are then solved for an optimal solution. Now the sub problems are coupled by the global consensus vector \me{\mbfa{\zeta}} maintaining the complete stacked  interference profile of all users in the system as
\begin{IEEEeqnarray}{ll}
	\mbfa{\zeta} &= \left [ \zeta_{1,\bar{\mc{U}}_{1}(1),1,1}, \dotsc, \zeta_{L,\bar{\mc{U}}_{1}(1),1,1}, \dotsc, \zeta_{L,\bar{\mc{U}}_{1}(|\bar{\mc{U}}_{1}|),1,1}, \right. \nonumber \\
	& \left . \dotsc, \zeta_{L,\bar{\mc{U}}_{N_B}(|\bar{\mc{U}}_{N_B}|),1,N_B}, \dotsc, \zeta_{L,\bar{\mc{U}}_{N_B}(|\bar{\mc{U}}_{N_B}|),N,N_B} \right ] \eqspace \IEEEyessubnumber \\
	n_{b_k} &= |\mbfa{\zeta}^{\set{b_k}}| = N L \sum_{b \in \mc{B}} \left | \bar{\mc{U}}_b\right |. \IEEEyessubnumber
\end{IEEEeqnarray}

Let \me{\mbfa{\zeta}(b_k)} denote the consensus entries corresponding to \ac{BS} \me{b_k}. Let \me{\mbfa{\nu}^{\set{b_k}}} represent the stacked dual variables corresponding to the equality condition \me{\mbfa{\zeta}^{\set{b_k}} = \mbfa{\zeta}(b_k)} used in the subproblems. In order to limit the local interference assumptions \me{\zeta^{\set{b_k}}_{l,k,n,b}} in \ac{BS} \me{b_k}, the \ac{ADMM} method augments a scaled quadratic penalty of the interference deviation between the local and consensus value for the interference from the \ac{BS} \me{b} as \me{\zeta_{l,k,n,b}} in the objective function. At optimality, the locally assumed and the consensus interference values will be equal,  providing no contribution to the objective function. The optimal step size used to update the dual variables is the scaling factor \me{\rho} used to scale the penalty term in the objective function \cite{boyd2011distributed,bertsekas1999nonlinear}. The equality constraint for the local and the consensus interference vector \me{\mbfa{\zeta}^{\set{b_k}} = \mbfa{\zeta}(b_k)} present in each subproblem is relaxed by the taking the partial Lagrangian. Now, the subproblem at \ac{BS} \me{b} for the \me{\ith{i}} iteration is given by
\iftoggle{single_column}{\allowdisplaybreaks
\begin{IEEEeqnarray}{rCl} \label{eqn-dual-3}
\underset{\substack{\gamma_{l,k,n}, \mvec{m}{l,k,n} \\ \beta_{l,k,n}, \mbfa{\zeta}^{\set{b}(j)}}}{\text{minimize}} &\quad & \| \tilde{\mbf{v}}_{b} \|_q + \mbfa{\nu}^{{\set{b}(j-1)}T} \left ( \mbfa{\zeta}^{\set{b}(j)} - \mbfa{\zeta}^{(j-1)}(b) \right ) + \frac{\rho}{2} \Big \| \, \underbrace{\mbfa{\zeta}^{\set{b}(j)}}_{\text{local}} - \underbrace{\mbfa{\zeta}^{(j-1)}(b)}_{\text{consensus}} \, \Big \|^2_2 \IEEEyessubnumber \label{eqn-dual-3a} \\
\text{subject to} && \beta_{l,k,n} \geq \sum_{\substack{j = 1\\j \neq l}}^L |\mvec{w}{l,k,n}^\herm \mvec{H}{{b},k,n} \mvec{m}{j,k,n} |^2 + \sum_{{\hat{b}} \in \bar{\mc{B}}_{b}} \zeta^{\set{b}(j-1)}_{l,k,n,{\hat{b}}} \nonumber \\
&& \qquad {} \qquad {} + \sum_{i \in \mc{U}_{b} \backslash \{k\}} \sum_{j = 1}^L |\mvec{w}{l,k,n}^\herm \mvec{H}{{b},k,n} \mvec{m}{j,i,n} |^2 + \enoise \IEEEyessubnumber \label{eqn-dual-1d} \\
&& \zeta^{\set{b}(j)}_{\pr{l},\pr{k},n,{b}} \geq \sum_{k \in \mc{U}_{\hat{b}}} \sum_{l = 1}^L |\mvec{w}{\pr{l},\pr{k},n}^\herm \mvec{H}{b,\pr{k},n} \mvec{m}{l,k,n} |^2 \; \forall \pr{k} \in \bar{\mc{U}}_{b}, \; \forall n \in \mc{N} \IEEEyessubnumber \label{eqn-dual-1e} \\
&\quad& \eqref{eqn-8} \; \text{and} \; \eqref{eqn-4.3}, \IEEEyessubnumber \label{eqn-dual-1f}
\end{IEEEeqnarray}
}{{\allowdisplaybreaks
\begin{IEEEeqnarray}{rCl} \label{eqn-dual-3}
\underset{\substack{\gamma_{l,k,n}, \mvec{m}{l,k,n} \\ \beta_{l,k,n}, \mbfa{\zeta}^{\set{b}(j)}}}{\text{minimize}} && \| \tilde{\mbf{v}}_{b} \|_q + \mbfa{\nu}^{{\set{b}(j-1)}T} \left ( \mbfa{\zeta}^{\set{b}(j)} - \mbfa{\zeta}^{(j-1)}(b) \right )  \nonumber \\
&& + \frac{\rho}{2} \Big \| \, \underbrace{\mbfa{\zeta}^{\set{b}(j)}}_{\text{local}} - \underbrace{\mbfa{\zeta}^{(j-1)}(b)}_{\text{consensus}} \, \Big \|^2_2 \IEEEyessubnumber \label{eqn-dual-3a}  \\
\text{subject to} &\quad& \nonumber \\
\beta_{l,k,n} \geq \sum_{\mathclap{j = 1,j \neq l}}^L && |\mvec{w}{l,k,n}^\herm \mvec{H}{{b},k,n} \mvec{m}{j,k,n} |^2 + \sum_{{\hat{b}} \in \bar{\mc{B}}_{b}} \zeta^{\set{b}(j-1)}_{l,k,n,{\hat{b}}} \nonumber \\
+ \sum_{i \in \mc{U}_{b} \backslash \{k\}} \sum_{j = 1}^L && |\mvec{w}{l,k,n}^\herm \mvec{H}{{b},k,n} \mvec{m}{j,i,n} |^2 + \enoise \IEEEyessubnumber \eqspace \label{eqn-dual-1d} \\
\zeta^{\set{b}(j)}_{\pr{l},\pr{k},n,{b}} \geq \sum_{k \in \mc{U}_{\hat{b}}} && \sum_{l = 1}^L |\mvec{w}{\pr{l},\pr{k},n}^\herm \mvec{H}{b,\pr{k},n} \mvec{m}{l,k,n} |^2, \forall \pr{k} \in \bar{\mc{U}}_{b} \IEEEyessubnumber \eqspace \label{eqn-dual-1e} \\
&& \eqref{eqn-8} \; \text{and} \; \eqref{eqn-4.3}, \IEEEyessubnumber \label{eqn-dual-1f}
\end{IEEEeqnarray}}}
where the superscript \me{i} represents the current iteration or the information exchange index and \me{\mbfa{\zeta}^{(j-1)}} denotes the updated global interference level from the \me{\ith{(j-1)}} information exchange of the local interference vector \me{\mbfa{\zeta}^{\set{b}(j-1)}, \forall b \in \mc{B}}.

Now, the local problem \eqref{eqn-dual-3} at each \ac{BS} \me{b} is solved either by the \ac{SCA} approach discussed in Section \ref{sec-3.2.1} or by using the \ac{MSE} reformulation approach outlined in Section \ref{sec-3.3}. Once the local problems are solved at each \ac{BS}, the new update for the global interference vector \me{\mbfa{\zeta}^{(j)}} and the dual variables \me{\mbfa{\nu}^{\set{b}(j)}} are performed at each \ac{BS} independently by exchanging the corresponding local copies of the interference vector \me{\mbfa{\zeta}^{\set{b}(j)}, \forall b \in \mc{B}}. Since the entries in \me{\mbfa{\zeta}^{(j)}} relate exactly to two \acp{BS} only, each entry in \me{\mbfa{\zeta}^{(j)}} can be updated by exchanging the local copies from the corresponding two \acp{BS}. For instance, the entry \me{\zeta^{(j)}_{l,\mc{U}_{b_k}(1),n,b}} depends on the local interference value \me{\zeta^{\set{b_k}(j)}_{l,\mc{U}_{b_k}(1),n,b}} assumed by the \ac{BS} \me{b_k} and the actual interference caused by \ac{BS} \me{b} as in \me{\zeta^{\set{b}(j)}_{l,\mc{U}_{b_k}(1),n,b}} as
\begin{equation}
\zeta_{l,\mc{U}_{b_k}(1),n,b}^{(j)} = \frac{1}{2} \, \left ( \, \zeta^{\set{b}(j)}_{l,\mc{U}_{b_k}(1),n,b} + \zeta^{\set{b_k}(j)}_{l,\mc{U}_{b_k}(1),n,b} \, \right ).
\label{zeta_update}
\end{equation}
The dual variable vector \me{\mbfa{\nu}^{\set{b_k}}}, which includes the stacked dual variables of the interference equality constraint at \ac{BS} \me{b_k}, are updated using the subgradient as
\begin{equation}\label{dual-sg-update}
\nu^{\set{b_k}(j)}_{l,k,n,b} = \nu^{\set{b_k}(j-1)}_{l,k,n,b} + \rho \, \left (\zeta^{\set{b_k}(j)}_{l,k,n,b} - \zeta^{(j)}_{l,k,n,b} \right ).
\end{equation}
The distributed precoder design using \ac{ADMM} approach is shown in Algorithm \ref{algo-3}.
{\allowdisplaybreaks 
\begin{algorithm}
\SetAlgoLined
\DontPrintSemicolon
\BlankLine
\SetKwInput{KwInit}{Initialize}
\KwIn{\me{a_k, \, Q_k, \, \mvec{H}{b,k,n},\; \fall b \in \mathcal{B}, \, \fall k \in \mathcal{U}, \fall n \in \mathcal{N}}}
\KwOut{\me{\mvec{m}{l,k,n}} and \me{\mvec{w}{l,k,n} \fall l \in \set{1,2,\dotsc,L}}}
\KwInit{\me{i=0} and the transmit precoders \me{{\mbf{m}}_{l,k,n}} randomly satisfying total power constraint \eqref{eqn-4.3}}
update \me{\mvec{w}{l,k,n}} with \eqref{eqn-10} and \me{\tilde{\mbf{u}}_{l,k,n}} with \eqref{eqn-8} \;
initialize the global interference vectors \me{\mbfa{\zeta}^{(0)} = \mbfa{0}^{\mathrm{T}}} \;
initialize the interference threshold \me{\mbfa{\nu}^{\set{b}(0)} \forall b \in \mc{B} = 0} \;
\ForEach{\ac{BS} \me{b \in \mc{B}}}{
	\Repeat{convergence or \me{i \geq I_{\max}}}{
		initialize \me{j=0} \;
		\Repeat{convergence or \me{j \geq J_{\max}}}{
			solve for \me{\mvec{m}{l,k,n}} and the local interference \me{\mbfa{\zeta}^{\set{b}}} using \eqref{eqn-dual-3} \;
			exchange \me{\mbfa{\zeta}^{\set{b}(j)}} among \acp{BS} in \me{\mc{B}} \;
			update dual variables in \me{\mbfa{\nu}^{\set{b}(j+1)}} using \eqref{dual-sg-update}\;
			update the consensus vector \me{\mbfa{\zeta}^{(j+1)}} using \eqref{zeta_update}\;
			\me{j = j + 1}\;
		}
		downlink precoded pilot transmission with \me{\mvec{m}{k,n}}\;
		update \me{\mvec{l}{k,n}} and notify to all \acp{BS} in \me{\mc{B}} using uplink precoded pilots as in \cite{komulainen2013effective}\;
		update \me{{\mbf{u}}_{l,k,n}} using \eqref{eqn-6.3} and \eqref{eqn-wsrm-expr} for \ac{SCA} or \me{\tilde{\epsilon}_{l,k,n}} using \eqref{eqn-mse-1.3} for \ac{MSE} approach \;
		$i = i + 1$ \;
	}
}
\caption{Distributed \ac{JSFRA} scheme using \ac{ADMM}}
\label{algo-3}
\end{algorithm}}

\begin{comment}
\subsubsection*{Convergence}
The convergence of the \ac{ADMM} method follows the same argument as the centralized algorithm if each distributed algorithm is allowed to converge to a stationary value for the fixed \ac{SCA} point. Since the subproblem solved at each \ac{BS} is convex, the \ac{ADMM} method converges to a stationary point \cite{boyd2011distributed} for the fixed \ac{SCA} value. The receive beamformers are updated along with the \ac{SCA} update of \me{\tilde{\mbf{u}}_{l,k,n}}. Combining the receiver update with the \ac{SCA} update improves the convergence speed due to the fact that the \ac{MMSE} receivers are optimal for the fixed transmit beamformers, providing monotonic increase in the objective function.
\end{comment}
