%We consider a downlink multi-cell \ac{MIMO} \ac{IBC} using \ac{OFDM} with \acl{MU} contending for space-frequency resources in a given scheduling instant. The problem is to determine the transmit precoders by the \acp{BS} in a coordinated approach to minimize the total number of backlogged packets in the \acp{BS}. The \ac{Q-WSRM} is a conventional design objective with the number of backlogged packets reflected in the corresponding weights, \textit{i.e}, the longer the queue size, the higher the priority. In contrast, we design the precoders jointly across the space-frequency resources by minimizing the total user queue deviations. The problem is nonconvex, and, therefore, we employ \ac{SCA} to solve the problem by a sequence of convex subproblems using the first order Taylor approximation. At first, we propose a centralized \ac{JSFRA} solution using two different formulations by employing the \ac{SCA} technique, namely the sum rate formulation and the \ac{MSE} reformulation. We then introduce distributed precoder designs using primal and \ac{ADMM} for the \ac{JSFRA} solutions. Finally, we propose a practical distributed iterative precoder design based on the \ac{MSE} reformulation approach by solving the \acl{KKT} conditions with closed form expressions. Numerical results are used to compare the proposed algorithms to the existing solutions.
%
%we address the queue minimizing downlink precoder design as a joint nonconvex optimization problem over space-frequency resources. We employ \ac{SCA} technique to solve the problem by a sequence of convex subproblems using inner approximations. Initially, we discuss the centralized \acl{JSFRA} solutions based on \ac{SCA} as well as by \acl{MSE} reformulation. Then we extend the distributed precoder design for the centralized schemes using primal and \ac{ADMM} method. Finally, we discuss the distributed precoder design problem by solving the \ac{KKT} expressions to obtain the closed form solutions for the transmit and the receive precoders. Numerical results are shown to compare them.
%
%We consider a downlink multi-cell \ac{MIMO} \ac{IBC} using \ac{OFDM} with \acl{MU} contending for space-frequency resources in a given scheduling instant. The problem is to design transmit precoders efficiently to minimize the total number of backlogged packets waiting at the coordinating \acp{BS}. Conventionally, the \ac{Q-WSRM} formulation with the number of backlogged packets as the corresponding weights is used to design the precoders. However, in the proposed \ac{JSFRA} scheme, the precoders are designed jointly across the space-frequency resources for all users by minimizing the total number of backlogged packets in each transmission instant, thereby performing user scheduling implicitly. Since the problem is nonconvex, we solve the \ac{JSFRA} formulation iteratively by solving an approximate convex subproblem in each iteration by updating the operating point. The \ac{JSFRA} formulation is solved using two different techniques, namely by the \ac{SINR} relaxation method and by the \ac{MSE} reformulation approach. We then discuss the distributed precoder designs using primal decomposition and the \ac{ADMM} for the proposed \ac{JSFRA} solutions. Finally, we propose a practical iterative precoder design by solving the \acl{KKT} expressions for the \ac{MSE} reformulation that requires minimal information exchange for each update. Numerical results are used to compare the proposed algorithms to the existing solutions.
We consider a downlink multi-cell \ac{MIMO} \ac{IBC} using \ac{OFDM} with \acl{MU} contending for space-frequency resources in a given scheduling instant. The problem is to design precoders efficiently to minimize the number of backlogged packets queuing in the coordinating \acp{BS}. Conventionally, the \ac{Q-WSRM} formulation with the number of backlogged packets as the corresponding weights is used to design the precoders. In contrast, we propose \ac{JSFRA} formulation, in which the precoders are designed jointly across the space-frequency resources for all users by minimizing the total number of backlogged packets in each transmission instant, thereby performing user scheduling implicitly. Since the problem is nonconvex, we use the combination of \ac{SCA} and \ac{AO} to handle nonconvex constraints in the \ac{JSFRA} formulation. In the first method, we approximate the \ac{SINR} by convex relaxations, while in the second approach, the equivalence between the \ac{SINR} and the \ac{MSE} is exploited. We then discuss the distributed approaches for the centralized algorithms using primal decomposition and \acl{ADMM}. Finally, we propose a more practical iterative precoder design by solving the \acl{KKT} expressions for the \ac{MSE} reformulation that requires minimal information exchange for each update. Numerical results are used to compare the proposed algorithms to the existing solutions.
%We consider a downlink multi-cell \ac{MIMO} \ac{IBC} using \ac{OFDM} with \acl{MU} contending for space-frequency resources in a given scheduling instant. The problem is to design transmit precoders efficiently to minimize the total number of backlogged packets queuing at the coordinating \acp{BS}. Conventionally, the \ac{Q-WSRM} formulation with the number of backlogged packets as the corresponding weights is used to design the precoders. Alternatively, we propose the \ac{JSFRA} scheme, which design the precoders jointly across the space-frequency resources for all users to minimize the number of backlogged packets in each transmission instant, thereby performing scheduling implicitly. Since the \ac{JSFRA} problem is nonconvex, we first propose two centralized iterative algorithms based on the combination of \ac{SCA} and the \ac{AO}. The proposed algorithms solve a sequence of convex problems obtained by fixing a subset of otpimization variables or by approximating the nonconvex constraints by the convex ones. In the first method we directly aprroximate the signal-to-interferenceplus-noise ratio (SINR) by convex formulations, while in the second one the equivalcen between rate function and mean squre error is exploited. Based on the centralized algorithms we then discuss the distributed precoder designs using primal decomposition and the alternating directions method of multipliers (ADMM). Finally, we also propose a more practical iterative precoder design by directly solving the Karush-Kuhn-Tucker system of equations for the MSE reformulation that is numerically shown to require minimal information exchange for each update. Numerical results are provided to demonstrate the superior performance of the proposed methods over the existing solutions.


%Since the problem is nonconvex, we solve the \ac{JSFRA} formulation iteratively by solving an approximate convex subproblem in each iteration by updating the operating point. The \ac{JSFRA} formulation is solved using two different techniques, namely by the \ac{SINR} relaxation method and by the \ac{MSE} reformulation approach. We then discuss the distributed precoder designs using primal decomposition and the \ac{ADMM} for the proposed \ac{JSFRA} solutions. Finally, we propose a practical iterative precoder design by solving the \acl{KKT} expressions for the \ac{MSE} reformulation that requires minimal information exchange for each update. Numerical results are used to compare the proposed algorithms to the existing solutions.


%Since the considered JSFRA  problem is nonconvex, we first propose two centralized iterative algorithms based on a combination of successive convex approximation and alternating optimization. The proposed algorithms solve a sequence of convex problems obtained by fixing a subset of otpimization variables or by approximating the nonconvex constraints by the convex ones. In the first method we directly aprroximate the signal-to-interferenceplus-noise ratio (SINR) by convex formulations, while in the second one the equivalcen between rate function and mean squre error is exploited. Based on the centralized algorithms we then discuss the distributed precoder designs using primal decomposition and the alternating directions method of multipliers (ADMM). Finally, we also propose a more practical iterative precoder design by directly solving the Karush-Kuhn-Tucker system of equations for the MSE reformulation that is numerically shown to require minimal information exchange for each update. Numerical results are provided to demonstrate the superior performance of the proposed methods over the existing solutions.